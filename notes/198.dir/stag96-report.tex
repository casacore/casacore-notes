%\documentstyle[11pt]{article}

%\title{Report of the AIPS++ Scientific \& Technical Advisory Group}

%\author{}

%\date{November 14, 1996}

%\maketitle

%\tableofcontents

\section{Introduction \& General comments}

The AIPS++ scientific and technical advisory group (membership is
included at the end of this document) met for the first time in
Socorro on 4, 5 and 6 November 1996. Presentations by various AIPS++
project members were interspersed with both public and private
discussion. In this document, we summarize the recomendations of the
advisory group; beginning with some rather general issues and then
proceeding to more specific issues for which feed-back was requested.

The AIPS++ project has clearly made substantial progress towards the
implementation of a powerful modern problem solving environment. The project
now needs to focus on making the transition from the
research/ development/ prototype stage to the production environment. A
comprehensive plan is needed for this phase of the project, with a
clear timetable for delivery of functionality and a realistic
indication of manpower requirements.

\section{Project Management}

Overall, we are happy with the current management of the AIPS++
project under Tim Cornwell.  Tim has the right combination of
scientific and software expertise, combined with a pragmatic approach
to software development, to get the job done.  However, more attention
needs to be paid to project definition.  In particular, (1) a
comprehensive development plan should be formulated, (2) the target
audiences and contents of the releases through 1998 need to be specified
in more detail, and (3) a date should be set (e.g. the mid-1997 release)
for completing the first stable version of the programming
interfaces.

We are concerned that the various application areas are currently
somewhat lacking in global focus. We can appreciate the difficulty of
beggining with a very extensive list of user requirements and then
translating these into a concrete development path which results in
full functionality in a finite time. Even so, this is exactly what is
necessary. A ``blueprint'' must be made for each application area
which sets out how to proceed, in an organized way, from zero
functionality to full functionality.  Some of the expertise to draw up
a realistic development path already exists within the AIPS++ group,
particularly in the areas of synthesis imaging and soon (we hope) in
the area of VLBI.  We suggest that more use might be made of external
``contact groups'' for the various application areas. Enlisting the
assistance of highly motivated and knowledgeable end--users might be
quite effective at focusing the development path.  The lack of global
development focus is especially evident in the area of Single Dish
applications. It seems necessary to us that a high level person carry
out a global system design in this area at the earliest possible date.

The lack of manpower available in all application areas is a major
concern. This needs to be addressed in the near future.


\section{Content and Audience of Releases}

There are a number of critical issues which must be addressed
concerning upcoming releases of the AIPS++ package. In particular, the
content and audience of each planned release must be well defined to
ensure that they will be effective. Furthermore, it is vital that each
release be ``billed'' in such a way that no unrealistic expectations
are generated. 

\subsection{January 1997 Release}

It is our view that the so--called ``$\beta$ release'' scheduled for
January 1997, must already be as bug--free as possible. In particular
the problems of memory management, and Glish--related crashes should be
solved, before even this limited distribution of the package is
undertaken. The functionality need not be excessive, but what is
present should work well. The audience of this release is presumed to
be only a very small number of ``friendly'' sites beyond the currently
participating consortium members. One of the primary purposes we
identify for this release is actually the definition of a concrete
deadline which should serve to focus the efforts of the group and lead
to a stable development environment at the earliest possible time.

Although we understand the increased logistics impact, we suggest that
both source code and documentation be included with the initial binary
release, even if there is no intention of supporting local building.
The likely recipients should be able to look more closely at the
workings of the package. The very limited functionality of this
release make it critical that prospective users have the ability to
get their data both in {\it and} out. The ability of the package to
write at least calibrated single--source UVFITS files should therefore
be given a high priority. We believe this would greatly enhance the
number of early users.

\subsection{Mid--1997 Release}

The release scheduled for mid--1997, which has been referred to as the
first ``full release'', might be better referred to as a ``Limited
Public Release''. The term ``full release'' seems to us to imply a
much greater degree of functionality than can be envisioned for this
early date. This first public offering of the package will, in our
view, be critical for defining the community perception of the
project. For this reason, we feel that it is important that a high
priority be given to further development of the user
interface(s). Simple ``auto--GUIs'' should be reinstated to enable
novice users to setup processing steps without the need for mastering
a verbose and non--intuitive syntax.  The command line interface should
also be carefully evaluated and if necessary repackaged to insure a
high degree of accessibility to the user community. The package needs
to ``feel'' good to new users. This property is more critical, in the
short term, than an enhanced functionality. During 1997, new users
will have more tolerance for a reduced functionality, than for the
frustrations which arise from an awkward interface.

At the same time it seems very desirable to have completed at least
one high level application package which thoroughly exercises all of the
major system components; including interactive image display and
graphics. Only then will it become clear what the ``standards'' (in
terms of user interface, documentation, performance, etc.) should
be for new applications within the system. These ``standards'' need to
be defined as soon as possible so that standard applications can be
generated within the extended user and programmer community.

The clear limitations of functionality of this release make it
important that data can be brought into, and removed from, the AIPS++
environment. A capability for both reading and writing multi--source
UVFITS seems essential in this regard. If it is deemed impossible to
reconcile the different table functionality of UVFITS with that of
AIPS++, then it would be sufficient to provide calibrated output.

\subsection{Subsequent Releases}

Based on the solid base of the mid--1997 release, increasing
functionality will presumably accompany each subsequent release. A
global plan to plot the development of all major processing components
which are currently envisaged should be made to help in planning for a
timely completion.

Planning for the early support of LINUX and NT platforms should receive a
moderately high priority.

\subsection{User Support \& Feedback} 

It is important that the target audience (astronomer/user or developer)
be understood before a release takes place.  Is the purpose of the
release to support outside development, or to provide end--users with
complete scientific packages to do serious work?  Is the purpose of a
release merely to involve the outside community and get feedback from
users to guide further development?  These decisions need to be made early
in the planning phase of a release to guide development.  The
announcement of a release should make it clear to prospective users what
type of user (end--user or developer) the release is targeted for.
Probably several releases over a span of 1--2 years will be required to
meet all goals for the baseline system and the emphasis of each planned
release should be made clear so that prospective users can decide when to
get involved.  We feel this is essential, not only to focus development,
but to avoid disappointing users who may expect a release to provide
something that it does not.

Further planning by the AIPS++ group is needed to prioritize
development for the next 1--2 years and specify the goals and contents
of each release.  Our impression is that the January 1997 release will
have limited impact outside the AIPS++ consortium sites and will serve
mainly to get a snapshot of the system out for testing.  The mid--1997
public release is unlikely to contain enough scientific applications
software to be interesting to most astronomers, but it will provide
the user community with an early look at the system and should provide
developers with the first stable and fairly complete version of the
development environment.  It is important that this version contain
one or two complete scientific applications to demonstrate that a
complete development environment exists, and to provide working
examples of typical AIPS++ applications.  A further release sometime
in 1998 should provide the first fairly comprehensive scientific
applications, as well as an update to the development environment.

Some mechanism for technical and user support should be in place by
the time of the mid--1997 release.  At a minimum this should consist
of a project email address for technical support and perhaps a hotline for
telephone support.  Additional resources may need to be identified for
platform support, preparation of distributions and patches, version
control, and technical and user support.

A list of user-related concerns which we believe require attention in
time for the mid--1997 release is given below.

\begin{itemize}
\item Provide GUIs for most functions.

\item Polish user interaction with Glish to have a better feel. 

\item Standardize the look and feel of different applications,
whether they are composed of Glish commands, part of the standard
packages or e.g.\ AipsView.

\item Provide context sensitive help.

\item Provide context sensitive defaults.

\item Provide tutorials for the (limited) advertised functionality.

\item Provide global and data-specific history logging and editing tools.

\item Provide means to save and reload parameter values.

\item Provide plotting capabilities with publication quality. This
includes control over fonts, sizes, weights and labeling.

\item Provide reasonable (perhaps tunable) memory usage and high 
performance (well within a factor of 2 compared to existing packages).
\end{itemize}

\subsection{Performance \& Testing}

It is especially important for a first package release to avoid
establishing a bad reputation which gets passed on by word--of--mouth
and which may take years to overcome.

There was significant concern that the performance associated with
some AIPS++ operations is unsatisfactorily slow.  Regrettably, few
hard numbers on the speed performance of AIPS++ code are available,
and the few that are, are not entirely encouraging. No systematic
performance evaluation of, for example, the synthesis package has been
performed. Such evaluation needs to be performed, and the code needs
to be tuned so that it is no worse than about a factor of two slower
than present packages, preferably much better. This problem was
discussed extensively with Cornwell, Glendenning and Garwood.
Clearly, the code can be sped up but it is unclear that manpower is
available to implement all of the changes needed on a reasonable
timescale.  The committee feels that performance is an important aspect
of AIPS++ and inadequate performance could
result in poor acceptance by the user community.

The memory requirements of AIPS++, both for a programmer (256~MB) and
a user (64~MB), tend to be larger than the workstations currently used
by most astronomers. Although hardware and compiler improvements may
diminish this concern over the next few years, the AIPS++ group needs
to continue to investigate ways to reduce memory bloat. Memory leaks
must be plugged.

The alpha testing of AIPS++ and the AIPS++ interface (Glish) has been
minimal so far, yet it is a very important step towards acceptance of
AIPS++ by the user community.  To date, it appears that virtually no
consortium astronomers have acted as friendly novice users or
``alpha'' testers. AIPS++ will not be a real system until it is in
production use on a daily basis within NRAO and the other consortium
sites.  New releases should be beta--tested within the
consortium for several months prior to any major
release, by having the software used on a daily basis for routine
data processing by the scientific and technical staff.
This is in addition to the alpha--testing performed by the programming
staff during development. We recommend the aggressive recruitment of
consortium scientific staff for in--house testing.

\section{AIPS++ Sub-Assemblies}

\subsection{Visualization}

AipsView has a reasonable overall functionality but has inadequate
publishable line graphics functionality (e.g.\ choice of fonts,
tiling, annotation). It is probably best to freeze development and
concentrate on the Glish/TK widgets and the image display library.

Completing the Glish/TK interface should have very high priority.
This should be in place for the January 1997 release.  It is not
completely clear whether it will yield adequate line--drawing
functionality from Glish. The group felt it was important to keep
searching for more capable substitutes for pgplot on a 1 to 2 year
timescale.

The image display library should receive fairly high priority, since
interactivity of DOs depends on it. Early 1998 might be a reasonable
target date for completion. 

The group did not feel that it was advisable to adopt Java in the
short term, but felt it important to keep track of commercial
developments in this area.

\subsection{Glish/User Interface}

Glish is one of the best features of AIPS++ and it provides a powerful
capability for object oriented programming at the level of interpreted
scripts. However, Glish as it currently stands, is not well suited as
a CLI, and something more approachable is needed to avoid turning off
users, especially novice users.  GUIs provide one way of addressing
the problem.  Both application GUIs and auto--generated GUIs for
controlling Glish level modules or objects will help.  If much
interactive use of the CLI is anticipated then a streamlined command
line interface to Glish is needed, similar to that provided by
existing astronomical data system shells and Unix shells.  It should
be easy to do simple, common operations without a lot of typing.
Addressing this shortcoming has a very high priority before the
mid--1997 public release. We wish to stress, however, that we are not
recommending construction of a complete layer on top of Glish.

There is also a need for real feed-back from novice users to evaluate
user satisfaction. It would be very fruitful for the developers to
spend some time with existing systems, like IDL, to gain insight into
successful CLI design.

Clearly it is critical that Glish be made robust against crashing.

Some suggestions of specific methods to reduce unnecessary typing are:
\begin{itemize}
\item shorter class names, 
\item minimum matching
\item default contexts (if you almost always type {\tt im.plot()}, let the
   {\tt im.} be default as indicated by a prompt)
\item  alternatives to the {\tt method()} syntax when there are no arguments
\item default (AIPSrc?) instantiation of most obvious objects
\end{itemize}

Specific goals for the mid--1997 release might be a number of standard
Glish closure scripts that bundle standard imager functionalities.
These should be accompanied with documentation, reliable failure
modes, and GUIs. The logging of processing history from both Glish
and DOs should also be enabled.

\subsection{GUIs}

There should be an auto--GUI or standard GUI for all DOs and all
``standard'' Glish scripts (preferably in time for the January 1997
release). 

We recommend development of simple GUIs created by DOs to enable
interaction with asynchronous processes (preferably in time for the
mid--1997 release).

Work should be done on developing a standardized GUI for major
applications. 

\subsection{Tasking}

A capability needs to be added to allow large jobs to execute in the
background (similar to ampersand in Unix).  This should include
provision for monitoring the progress of background jobs, interacting
with jobs which require input, aborting jobs, and passing data back
to the foreground context.

Ideally AIPS++ should not be tied to Glish as the only mechanism for
managing intermodule (distributed object) communications.  More powerful
messaging system standards are sure to be important in the future, e.g.\ 
CORBA.  The messaging subsystem should be abstracted and isolated so that
new messaging system technology can be substituted in the future.  In
addition to protecting AIPS++ from technological evolution in messaging
systems, this is important if AIPS++ is to be an open system capable of
communicating with external non--AIPS software.

\subsection{Parallelization}

Some modest level of effort needs to be devoted to the discussion
of parallelization, including the use of multi--threads, but this is
clearly a goal for a future release. Taking advantage of specific
software development talent on an as-needed basis is probably
realistic at this point.  

Within the modest overall priority of this topic, most attention
should be given to supporting parallelization within small
multi-processor machines, which are likely to become increasingly
common within the general user community.

\subsection{Single Dish Data Analysis}

Some progress is being made at last on the single dish data analysis
front.  The committee was shown a prototype application (SDCalc) which
had a good user interface for operations on single scans.  While we
felt this was a good prototyping experience we felt it lacked the
scope necessary to deal with multi-dimensional datasets and the high
data rates which will, in general, be needed.  The committee strongly
recommends a multi-dimensional approach which is more solidly based on
the standard AIPS++ MeasurementSet.  In particular, we envision users
will need sophisticated bulk--analysis tools to cope with large
datasets with several of the following dimensions: two spatial, one or
more frequency, one polarization, one beam and one time.  The
flexibility of the MeasurementSet can accommodate these needs more
easily than a FITS datacube approach.  A multi-dimensional user
interface approach will make this flood of data more comprehensible to
the astronomer.

One area of synergy which appears bypassed is the interface between
traditional single dish processing and the AipsView visualization
world.  For example, AipsView is capable of selecting and creating
spectra from datacubes. These spectra, or collections of spectra,
should be written into a MeasurementSet which could then be
manipulated by SDCalc.

There was substantial concern that the envisioned needs of the GBT
might not be met on an appropriate timescale.

\subsection{Help/Documentation}

All AIPS++ releases, starting with January 1997  should contain
full application oriented documentation to enable users to run AIPS++ 
effectively. This documentation should include fully--worked examples 
using a test data set shipped with the release, and a description of
the methodology being used by each application. The documentation should 
preferably be shipped with the software, so that network problems do
not get in the way of the user. There should be an extensive tutorial
on general AIPS++ and Glish use, including annotated scripts. Specific
requirements include:

\begin{itemize}
\item Module/Object/Method listings which are more hierarchical. All
the user now sees is a very flat listing of methods.

\item Context oriented help from Glish.

\item Complete postscript versions of all the web documents.

\item A search engine to enable program developers to find relevant code.
\end{itemize}

\subsection{Synthesis and VLBI processing}

Polarimetric self--calibration and NNLS imaging are the jewels of the
January release. This capacity is unique. A good implementation of
these is most important.  The speed of the processing, however, is a
major concern. Conventional self--calibration and deconvolution should
not be a factor of 5 slower than comparable systems in the January
release, and preferably no slower than a factor of 2 by the mid--1997
release. If these speed requirements can be met, then spectral--line,
multi--IF/multi--frequency processing should be addressed.  Mosaicing
and wide--field imaging are very desirable but should have lower priority
in both January and mid-1997 releases.

Extending the above synthesis processing to include the functionality
and user friendly interface of DIFMAP would be valuable for the VLBI
community, as well as being of wider appeal among synthesis imagers.
Once a better user interface has been produced, this should be
comparatively inexpensive in relation to the product that can be
delivered. DIFMAP should be used as a reference for performance.

Planning for VLBI processing is in its infancy within the project. The
current requirements and planning for VLBI processing do not appear to
be well integrated within the project. We appreciate the lack of
manpower currently commited. However, longer term planning for VLBI is
sorely needed.

\subsection{Measures}

The arrival of the measures toolbox was a welcome development. Full
support of VLBI and pulsar applications will be important additions
for the coming months. There is some concern regarding the need for
extensive retrofitting of code to introduce the toolkit at all the
necessary levels. Tools like this would ideally have been available in
the early stage of application development. At this moment several
application development projects can still profit from continuing
effort in this area, notably VLBI and interferometer calibration.

We can see good reason to make some of the functionality of this
system available at the user interface level in an early stage. Priority
might be given to making the unit system available, in particular the
use of physical constant and unit conversions from Glish. Furthermore,
coordinate transformation (e.g.\ from B1950 to J2000) would be
interesting as well as Doppler calculation (e.g.\ topocentric to LSR).

It is important that all physical constants, coordinate systems and
methods of conversion be fully documented with appropriate references.

\subsection{General Toolkit}

The AIPS++ library provides the programmer with a rich and versatile
toolkit. At present, the main pending developments in the Table
System part of the library appear to be ensuring that
features such as multi-user synchronization and locking are
implemented.

There are two critical items with regard to the libraries: Firstly, it is
vital that memory leaks be hunted down and eliminated. Otherwise
mysterious system crashes are sure to occur. Secondly, because there
appear to be misgivings that AIPS++ applications are slow compared
to the competition, every effort should be made to ensure that
the library classes are optimally designed for maximum speed.

The time may be nearing when the core functionality of the library
should be frozen, with a switch of priority to writing applications.

\section{STAG Membership}
\parindent=0cm

Robert Braun (NFRA) chair

Jayaram Chengalur (NCRA) not present for this meeting

Roger Foster (NRL)

Dennis Gannon (Univ. Indiana)

Walter Jaffe (Univ. Leiden)

Lee Mundy (Univ. Maryland)

Bob Sault (ATNF)

Lister Stavely-Smith (ATNF)

Dave Shone (NRAL)

Doug Tody (NOAO)

Huib Jan van Langevelde (JIVE)

Tony Willis (DRAO)

Al Wootten (NRAO)
%\end{document}
% LocalWords: 