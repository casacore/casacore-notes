
\section{Documenting Your Application}
\label{sec:documenting}
\subsection{Background}

Good documentation is vital for the success of code contributed to
AIPS++. Without documentation, your code will be neglected.  In
AIPS++, user documentation is written in latex using a set of special
purpose macros. These are defined in \htmlref{aips2help.sty}{197.aips2help}.
The example for the \htmlref{tableplotter}{tableplotter} is worth studying.

If at all possible, provide a global function for demonstration that
illustrates how the application is to be used. Place the demo function
inside the {\tt .g} file that contains the application code.  Try to
make the demonstration as self-contained as possible without
compromising the demonstration. Here is the demo function for
tableplotter:

\begin{verbatim}
  const tableplotterdemo:=function() {
    localtp:=tableplotter();
    col1:=tablecreatescalarcoldesc("N",0);
    col2:=tablecreatescalarcoldesc("N^2",0);
    td := tablecreatedesc (col1, col2);

    tb := table('squares', td, 100);
    tb.putcol("N", 1:100);
    tb.putcol("N^2", (1:100)^2);

    print localtp.settable(tb);
    print localtp.setx("N")
    print localtp.sety("N^2");
    print localtp.auto();
    localtp.x.label:="N";
    localtp.y.label:="N**2";
    localtp.title:="Squares";
    print localtp.show();
    print localtp.getxy();
    print localtp.plot();
    print tb.delete();
    print shell('rm -r squares');
  }
\end{verbatim}

Note that it is self-contained, constructing a table specially for
the demonstration. Note also that all public functions are invoked.
Remember that a good demonstration can often help a user penetrate
the fog of confusing or inadequate documentation.

\newcommand{\userrefman}{
\htmladdnormallink{"AIPS++ User Reference Manual"}{../../user/Refman/Refman.html}}
To support writting user based help for AIPS++ in LaTeX, 
we have developed an aips2help.sty file for generating stylistically proper
documentation.
While it is not necessary for you to write the documentation
using the aips2help.sty we urge you to use it.
If you use the aips2help.sty and the help templates, it will be possible to
parse the help input files and provide non-html-on-line help.

User help (for scripts and stand-alone commands) is written at
three levels:
\begin{itemize}
\item Module,
\item Object, and
\item Function/Method.
\end{itemize}
There are three template .help files (
\htmladdnormallink{template-module-help}{../../../code/install/codedevl/template-module-help}, 
\htmladdnormallink{template-object-help}{../../../code/install/codedevl/template-object-help}, 
\htmladdnormallink{template-function-help}{../../../code/install/codedevl/template-function-help})
 that should be used.  By convention
the files end in .help.  Each template file contains several items
that ought to be in the documentation.

To see what your .help file will look like in the \userrefman you need to run 
\texttt{help2tex} on your .help file.  \texttt{Help2tex} translates the .help
file into
a vanilla LaTeX used in the \userrefman.
You may run \texttt{latex2e} or \texttt{latex2html} on your .help file
without running it 
through \texttt{help2tex}, but the output will be 
different from what appears in the \userrefman.

\subsection{aips2help.sty\label{197:aips2help}}
Aips2help.sty defines several environments and commands.
\subsubsection{aips2help.sty Environments}
(Note: Preface the environment with
\verb!\begin{ahenvironment}!  and end the environment block with 
\verb!\end{ahenvironment}!.)
\begin{description}
\item[\{ahpackage\}\{one\}\{two\}] Defines a package. The first argument
is the package name the second argument is a short description of the package.
Packages contain usually just modules.
\item[\{ahmodule\}\{one\}\{two\}] Defines a module. The first argument
is the module name the second argument is a short description of the module.
Module may contain tools as well as functions.
\item[\{ahtool\}\{one\}\{two\}] Defines a tool. The first argument
is the tool name the second argument is a short description of the
tool.  Tools may contain function/methods.
\item[\{ahfunction\}\{one\}\{two\}] Defines a function/method. The first
argument is the function/method name, the second argument is a short
description of the function/method.  If used in an ahtool environment
the function is a "member function" of that tool.  If defined outside 
of an ahtool environment but inside an ahmodule environment it is a 
global function.
\item[\{ahconstructor\}\{one\}\{two\}] Defines a constructor.  The first
argument is the constructor name, the second argument is a short description
of the constructor.  Only makes sense if used inside an ahobject environment.
\item[\{ahaipsrc\}]Identifies what aipsrc variables may be used.  Use
\verb!\ahaddarg! to identify the aipsrc variables.
\item[\{ahdata\}]Identify public data for an object. Use \verb!\ahdatum! to
identify public data members.
\item[\{ahrecord\}]Identify members of a record.  Use \verb!\ahdatum! to
identify record members.
\item[\{ahdescription\}] Specifies a more complete desciption of module,
object, or function/method.
\item[\{ahargs\}] Defines an argument list, use \verb!\ahaddarg! to identify the
arguments.
\item[\{ahexample\}] Present an example of how to use the module, object,
function/method here.  You will need to enclose the actual example text
inside of a \verb!\begin{verbatim} \end{verbatim}! block.
\item[\{ahcomments\}] Put comments about the example in this environment.
\item[\{ahseealso\}] Put links to other things of interest here.  Use the
\verb!\ahlink! command to specify what else to look at.
\end{description}

\subsubsection{aips2help.sty Commands}

\begin{description}
\item[$\backslash$ahinclude\{one\}] Use this command to identify any files to include.
Most useful with glish.
%
\item[$\backslash$ahkeyword\{one\}\{two\}] Use this command to index a keyword with
the module, object, or function/method.  The second argument should
be left blank.
The $\backslash$ahkeyword are used to index the user documentation.  
 While any word(s) maybe used for keywords we suggest using nouns only.
%
\item[$\backslash$ahfuncs{}] This command will print a list of functions with their
corresponding short descriptions.  Used in the ahmodule or ahobject
environments.  
If used within the ahobject/ahtool environment it will print a list
of member functions and constructors for the tool.  If used with in the
ahmodule environment it will print the global functions for that
module.
%
\item[$\backslash$ahmethods{}]  This command will print a list of methods with their
corresponding short descriptions.  Used in the ahobject environment.
%
\item[$\backslash$ahobjs{}]  This command will print a list of objects with their
corresponding short descriptions.  Used in the ahmethod environment.
%
\item[$\backslash$ahlink\{one\}\{two\}] This command specifies a link between the
text specified in argument one and a label specified in argument two.  A typical use
of this command would be to link to another tool or function of the reference manual.
The reference manual uses the following scheme for labeling tools and functions,
\begin{itemize}
\item modulename:toolname
\item modulename:toolname.constructor (.constructor identifies it as a constructor)
\item modulename:toolname.function (.function identifies it as a global function)
\item modulename:functionname
\item modulename:toolname.functionname
\end{itemize}
An example for the calc constructor is
\begin{verbatim}
\ahlink{calc the constructor}{images:image.calc.constructor}
\end{verbatim}
\begin{verbatim}
for the global calc function
\ahlink{calc the function}{images:image.calc.function}
\end{verbatim}
Used inside the ahseealso environment.
%
\item[$\backslash$ahaddarg[in | out | inout] \{one\}\{two\}\{three\}\{four\}] This command is used inside
the ahargs environment.  The name is specfied in argument one, a description
is specified in argument two, the default value is specified in argument
three, and the allowed values are specfied in argument four.  
\item[$\backslash$ahaddarg\{one\}\{two\}\{three\}\{four\}] This command is
used inside the ahrecord and ahdata environments. The name is specfied in argument one, a description
is specified in argument two, the default value is specified in argument
three, and the allowed values are specfied in argument four.  The optional argument specified in []
identifies the argument as either an input, output, or input and output.
%
\item[$\backslash$ahreturns\{one\}] This command tells what is returned from a function
or method.
\end{description}


\subsection{Modules}
After writing module
documentation you need to include the module in the existing package .help
file adding your module with the line:\\
\begin{verbatim}
\input{mypackage.help}
\end{verbatim}

\label{197:module_example}
This example of module documentation comes from
\htmladdnormallink{code/trial/apps/table/table.help}{../../../code/trial/apps/table/table.help}.
\begin{verbatim}
\begin{ahmodule}{table}{Glish interface to table system}

\ahinclude{"table.g"}

\begin{ahdescription}
table allows creation of table objects inside Glish. The resulting
objects can be operated on in various ways:
\begin{itemize}
\item Creation of new tables,
\item Opening, deletion, cloning, copying of existing tables,
\item Set and put of table information strings,
\item Get and put of table cells, columns and keywords,
\item Iteration by subtables,
\item Access via table rows,
\item Browsing of tables,
\item Printing of a summary of a table,
\end{itemize}
In addition this module contains a number of global functions
related to tables.
\end{ahdescription}

\begin{ahexample}
\begin{verbatim}
include "table.g"
vis:=table("3C273XC1.MS", readonly=T);
vis.summary();
uvw:=vis.getcol("UVW");
spw:=table("3C273XC1.MS/SPECTRAL_WINDOW", readonly=T);
freq:=spw.getcell("REFERENCE_FREQUENCY", 1);
uvw*:=(1.420E9/freq);
vis.putcol("UVW", uvw);
vis.close();
\end{verbatim}
\verb!\end{verbatim}!
\begin{verbatim}
\end{ahexample}
\begin{ahseealso}
\ahlink{tablerow}{tablerow.label}
\ahlink{tableiterator}{tableiterator.label}
\end{ahseealso}

\ahobjs{}
\ahfuncs{}

.... Lots of stuff deleted

\end{ahmodule}
\end{verbatim}
\subsection{Object}

\label{197:example_object}
This example of object documentation comes from
\htmladdnormallink{code/trial/apps/table/table.help}{../../../code/trial/apps/table/table.help}.
\begin{verbatim}
\begin{ahobject}{table}{table object}

\begin{ahconstructor}{table}{Construct table object}
\begin{ahargs}
\ahaddarg{tablename}{Name of table on disk}{F}{Bool}
\ahaddarg{tabledesc}{Table descriptor}{F}{Bool}
\ahaddarg{nrow}{Number of rows}{0}{Int}
\ahaddarg{readonly}{Open Read-only?}{F}{Bool}
\ahaddarg{ack}{Acknowledge creations, etc}{T}{Bool}
\ahaddarg{tablehandler}{Table handler to be used}{gtable}{Any tableserver}
\ahaddarg{tablelogger}{logger to be used}{logger}{Any logger object}
\ahreturns{object}
\end{ahargs}

\ahfuncs{}

\begin{ahexample}
\begin{verbatim}
table1:=table("3C273XC1.MS");
table2:=table("3C273XC1.new.MS", table1.getdesc());
\end{verbatim}
\verb!\end{verbatim}!
\begin{verbatim}
\end{ahexample}
\begin{ahcomments}
The first line opens an existing table 3C273XC1.MS, the second creates another
table using the table description of the first table, but no rows are written.
\end{ahcomments}
\end{ahconstructor}

\begin{ahfunction}{ok}{Is the table object ok?}
\ahreturns{Bool}
\end{ahfunction}

....Lots of stuff deleted....

\end{ahobject}
\end{verbatim}

\subsection{Function/Method}
A function belongs to a module, if a function is defined inside an
object it is considered a method of that object. 
\label{197doc:function_example}
This example of function documentation comes from
\htmladdnormallink{code/trial/apps/table/table.help}{../../../code/trial/apps/table/table.help}.
\begin{verbatim}
\begin{ahfunction}{tablecommand}{Execute a tablecommand}
\begin{ahargs}
\ahaddarg{comm}{Command string}{}{Any valid table command}
\end{ahargs}
\begin{ahexample}
\begin{verbatim}
print tablecommand('SELECT FROM table.in WHERE column1*column2$>$10 ORDERBY 
column1 GIVING table.out');
\end{verbatim}
\verb!\end{verbatim}!
\begin{verbatim}
\end{ahexample}
\ahreturns{Bool}

\begin{ahcomments}
The grammar for the command string is SQL-like and is defined
in TableGram.l and explained in Tables.h.
Between SELECT and FROM you can give some column names (separated by
commas). Then the output table will contain those columns only.
If no column names are given, all columns will be selected.
E.g.:   SELECT column1, column2,column3 FROM table.in GIVING table.out

The WHERE part (like column1*column2$>$10) contains an arbitrary expression.
Functions (like sin, max, ceil) are supported. Only scalar columns
(or keywords) can be used in the expression. Complex numbers must
be given as, for example, 3.4+4i (similar to glish).
With some extra syntax (not explained here) it is even possible
to use keywords from other tables in the expression.
Rows obeying the WHERE expression will be selected.

Sorting can be done with the ORDERBY clause. You can give there any
number of (scalar) columns separated by commas. You can use a mix of
ascending (is default) and descending.
E.g.:   ORDERBY column1 DESC, column2, column3 ASC, column4 DESC

The GIVING clause defines the output table containing the
requested rows and columns. It can be used as any other table.

Each part (column selection, expression, sorting, giving) is optional.
\end{ahcomments}
\end{ahfunction}
\end{verbatim}
\subsection{Help System Details}
Where do we put help files?  Put them where the code is.  All help files should
end in a .help extension.  Also, if you use any encapsilated postscript files,
put them in that directory too.  If you are adding a new object or module, you
will have to update
the appropriate .help file for the module or project.  Currently we have the
following project files, aips.help, synthesis.help and nrao.help.  All are
found in
trial/scripts.  Module files are likely to be scattered through out the code
tree (use the following to help you track them down: cd /aips++/daily/code;
find . -name module.help -print)

The .help files are used for two purpose, one generate standard LaTeX for a 
"printed" reference manual (Refman) and the other to generate glish help atoms
for use by the help function in glish.

To generate the LaTeX document, there is a special helpsys target in the 
code/doc/user makefile.  It searches the active code tree, (trial, aips, 
synthesis, and nrao) for all files ending in a .help, .ps, and .eps extension.
Once the documents are collected, help2tex is run on the package helpfiles
and .htex files are produced.  These .htex files are compared with the existing
ones in docs/user/helpfiles and if different updated.  The .ps and .eps files
are also moved to this directory if necessary.  Help atoms file for glish are
generated if necessary at this time by help2tex.  The standard .latex rules
take over after all the copying is done.

We encourage your to check out your help file by doing the following,

\begin{verbatim}
gmake myfile.dvi
\end{verbatim}

The gmake runs help2tex on the file myfile.help 
and then runs latex on the resulting ".htex" file.
It's important to make sure your .help file runs through the process clean. If
it fails any documentation that follows won't be included in the Refman.  If
have relative-outside links in your .help file or links to other parts of the
Refman,
they will be hi-lighted but the hyper-links will fail when built in the 
"programmer" tree.  This happens because of three things:
\begin{enumerate}
\item The docs tree has a different number of directories compared to the
programmers code tree,
\item The external documents won't be in your tree, and
\item You are only building a small fraction of the reference manual and it's
entirely out of context.
\end{enumerate}
Currently checking links to external documents in the aips++ docs tree,
requires
specifying a DOCSROOT and building the Refman and the external document you 
need to check against (gmake docsys DOCSROOT=/mydocs).  Note for gmake docsys
to work, you need to ai and au your help file, cd code/doc/user and then run
the gmake.
