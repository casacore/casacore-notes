\subsection{Adding Notes and Memos}\label{215:addnotes}
\textit{Contributed by Wim Brouw, ATNF}\newline
It took me a while to insert a Note in the system and considering the
difficulty others have had, I make the following points:

\begin{enumerate}
\item Select a note number (I just take the next number available, but maybe
   there is somebody to ask?) - say nnn (e.g. 225)
\item Create (check in) directory code/doc/notes/nnn.dir
\textit{(If it's there then others will know the number is taken.)}
\item Put the body of your latex Note (without embracing document information
   like documentstyle, title, \verb+\begin{document}+ etc) in nnn.dir (e.g.
   ai -l mynote.latex)
\item Create, (in code/doc/notes itself) a file called nnn.latex containing the
   document information, and which includes mynote.latex (without a
   directory!)
   Example:
\begin{verbatim}
\documentclass[11pt]{article}
\usepackage{html, epsf}
%%-----------------------------------------------------------------------------

\begin{document}

\title{NOTE 224 -- AIPS++ Least Squares background}
  

\author{Wim Brouw}

\date{22 January 1999}

\maketitle
%%---------------------------------------------------------------------------
\begin{htmlonly}
\htmladdnormallink{A postscript version of this note is available 
(124kB).}{../224.ps}
\end{htmlonly}

\tableofcontents
      
\input{lsq.latex}
\end{document}
\end{verbatim}

\item Add your note number to the code/doc/notes/makefile (just do like the
   other notes, e.g.:
\begin{verbatim}
224 := -split 0
\end{verbatim}

\item Test if all ok, by making a Postscript version:
   gmake nnn.ps\newline
   And an html version:\newline
   gmake nnn\newline

\item If all is ok, then add your Note to the Notes index:
\begin{verbatim}
   code/doc/notes/notes.dir/nindex.tex
   E.g:
\item[224]
\htmladdnormallink{\textit{AIPS++ Least Squares background}}{../224/224.html}
\linebreak  1999/01/25 Brouw
\end{verbatim}

\end{enumerate}


\textit{Note: Other notes, AIPS++ documents can include your note or parts of your note
by adding the following line to the makefile}
\begin{verbatim}
TIROOT := $(word 1, $(AIPSPATH))/code/doc
EXTRA_TEXINPUTS := $(TIROOT)/memos/111.dir:$(TIROOT)/notes/156.dir
/notes/196.dir
\end{verbatim}
\textit{you then use the} \verb+\input{file}+
\textit{ to include it the other document.
The HowTos (code/doc/reference/HowTos.latex) is a good example.}
