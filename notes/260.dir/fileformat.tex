
%%\externallabels{../../user/Refman}{../../user/Refman/labels.pl}

\section{Introduction}
The Casacore Table System (CacTS) is an RDBMS-like system to store data in
tables consisting of a number of columns and rows. A
table and each column can have an associated set of keywords
 to define global data (like
subtables or the unit of a column containing numeric data).
Also called TAOSS (Tabular Array-Oriented Storage System)

Besides the usual scalar data, a table column or keyword can hold
N-dimensional arrays.
The data type of a scalar or array can be Bool, Char, uChar, Short,
uShort, Int, uInt, Float, Double, Complex, DComplex, or
String. Furthermore a keyword can be an Int64,

The keywords are meant for small amounts of data. The bulk data will
be stored in columns.
There is a strict distinction between the logical and physical
model for the data columns.
The table description defines the logical model, while data managers
(storage managers and virtual data managers) implement the physical
model.

On disk a CaCTS table is a directory containing
a number of files to hold data and meta data. The formats of these
files are quite varying and can be quite complicated.
\\Furthermore a table can have a zero or more subtables that are
usually stored as a subdirectory of the main table. A good example of
this is the casacore MeasurementSet (see note 229).

A CaCTS table can be a reference to one or more other tables which is
comparable to a view in a relational data base.
CaCTS supports a few types of tables.
\begin{itemize}
\item
A PlainTable is the basic table type. It contains all data.
\item
A RefTable is a referencing table. It only references rows
in another table. Usually it is the result of a select, sort, or
iteration of another table.
The subtables of a RefTable are the same as those of its parent; it
references them.
\item
A ConcatTable is the virtual concatenation of tables with an identical
structure (identical columns). It only references the tables to be
concatenated.
The subtables of a ConcatTable can be concatenated at will. It means
that a subtable can be the subtable of the first table or a
concatenation of the subtable in all tables. 
\end{itemize}

All table types consist of a directory with the name of the table and
several files. The following two files are always present.
A PlainTable can contain several more files.
\begin{itemize}
\item
\texttt{table.dat} is a binary file holding the table definition. The
first part of its contents is the same for all table types. The
remainder is type specific.
\item
\texttt{table.info} is a text file containing a brief
description of the table.
\end{itemize}

This document describes the format of all files supported natively by CaCTS.
CaCTS can use arbitrary storage managers dynamically loaded
from a shared library. The file formats of such third party storage
managers are not described in this document.

CaCTS is designed such that it is fully backward compatible. Even
tables from the very first days can be read back despite the fact that
the format has changed considerably.
\\CaCTS understands both little and big endian data
representations. At table creation time it is decided which one to use.


\section{AipsIO}
CacTS makes heavily use of the AipsIO class to store its meta
data. Therefore a brief description of the AipsIO format is given.

AipsIO is basically a mechanism to write a C++ object into a
stream of bytes and to read it back. The format consists of a header followed
by the object's data. It can be nested arbitrarily deeply.
\\The header contains the following fields:

\vspace{0.15in}
\begin{tabular}{|l|l|} \hline
  Data Type & Description \\ \hline\hline
  uInt & Object length \\
  String & Object type \\
  uInt & Version \\
  \hline
\end{tabular}
\vspace{0.15in}

The version can be used to make the software fully backward compatible
in case an object changes over time. This feature is used a lot by CaCTS.
\\The object data are stored after the header as a stream of
bytes. The data are unaligned in the stream, thus, say, an integer
does not need to start on a multiple of 4 bytes.
\\The length of an object is the total length; it includes the length
of the header and the length of possibly nested objects. Of course, each nested
object has its own header.
\\The following data types are natively supported. Other data types
(classes) can be supported by writing the appropriate shift functions
(a la iostream).

\vspace{0.15in}
\begin{tabular}{|l|p{12cm}|} \hline
  Bool & 1 byte \\
  Char and uChar & 1 byte \\
  Short and uShort & 2 bytes \\
  Int and uInt & 4 bytes \\
  Int64 and uInt64 & 8 bytes \\
  Float & 4 bytes IEEE \\
  Double & 8 bytes IEEE \\
  Complex & 2 Floats \\
  DComplex & 2 Doubles \\
  String & uInt length followed by its characters (can be length 0) \\
  C-array & uInt length followed by its elements (can be length 0).
  \tabularnewline & A bool array is compressed to bits. \\
  \hline
\end{tabular}
\vspace{0.15in}

AipsIO can store the data in little or big endian order. In fact,
it can also handle old VAX/VMS and IBM/360 data formats.
\\CaCTS only uses the canonical AipsIO format (which is big endian).

\subsection{Array objects}
Templated Array$<$T$>$ objects are used quite heavily in casacore. In
AipsIO they are stored as follows.

\vspace{0.15in}
\begin{tabular}{|l|p{13cm}|} \hline
  header & Array (version 3) \\
  uInt & number of dimensions \\
  ndim Int & shape \\
  ndim Int & origin (only for version $<$ 3) \\
  T & all array values. Bools are compressed to bits. \\
  \hline
\end{tabular}
\vspace{0.15in}

\begin {itemize}
\item Casacore Array objects hold their data in Fortran order (fastest
  varying axis first). The shape and data are also stored that way.
\item In principle the template parameter can be of any
  type. However, CaCTS only uses Arrays of the standard types.
\item Until 1995 the Array class supported an origin. It has been
  removed because it was a
  source of errors. Theoretically an origin can still be present in
  Array objects in very old tables. If so, it is read and silently discarded.
\item Before RTTI was fully supported in C++, casacore used its
  own RTTI functionality. At that time the name of the template
  parameter was made part of the AipsIO header, for example
  \texttt{Array<Int>}. After the standard C++ RTTI was used, this was
  not done anymore and an array of any type has the same header name
  \texttt{Array}. When reading an older Array object from AipsIO, the
  template part in the name is still recognized, but not used anymore.
\end {itemize}

\subsection{IPosition object}
An IPosition object defines the shape of an Array object.
It will be stored in AipsIO as follows.

\vspace{0.15in}
\begin{tabular}{|l|p{13cm}|} \hline
  header & IPosition (version 1 or 2) \\
  uInt & number of dimensions \\
  ndim Int & shape (for version 1) \\
  ndim Int64 & shape (for version 2) \\
  \hline
\end{tabular}
\vspace{0.15in}

Until 2009 a shape element was represented as a 32-bit integer,
thereafter as a 64-bit integer. To be as forward compatible as
possible, an IPosition is still stored as version 1 if all elements in
the shape have a value fitting in a 32-bit integer.
 
\subsection{TableRecord object}
A TableRecord object holds a set of keyword/value pairs. They are used
in CaCTS to represent the keyword set attached to the table and each column.
\\A value can be a scalar or array of one the standard data types including
complex and string values. Furthermore a value can be a TableRecord in
itself; the nesting can be arbitrarily deep. Finally a value can be
a Table object which is used to hold the subtables of a table.

A TableRecord is described by a RecordDesc object that also
contains the descriptions of possible nested records. A TableRecord
can have a fixed or variable format. In CaCTS only variable fomat
TableRecords are used.

In the very early days of CaCTS a keyword set was stored in a very
different way. Reading back such keyword sets is still supported in
the code.

A TableRecord is stored in AipsIO as follows:

\vspace{0.15in}
\begin{tabular}{|l|p{13cm}|} \hline
  Data Type & Description \\ \hline\hline
  AipsIO header & TableRecord (version 1) \\
  RecordDesc & The TableRecord description \\
  Int & record type (CaCTS always uses variable format records) \\
  any & All values are stored consecutevily.
  Their data types are defined in the RecordDesc.
   \tabularnewline & Scalars and arrays are stored in the standard AipsIO way.
   \tabularnewline & For a nested TableRecord only the values are
   stored, because its description is part of the record
   description. However, a nested empty TableRecord is stored as such.
  \tabularnewline & If a value is a Table, only the table name is
  stored. The name is tried to be made relative to the
  name of the parent table. In that way moving a table has no effect
  its subtable names. A relative name is prefixed with
  ././ if the full parent table name has to be used as
  directory, while the prefix ./ is used if the directory name of
  the parent table has to be used. Note that in all practical cases
  (like a MeasurementSet) ././ is used as prefix. \\
  \hline
\end{tabular}
\vspace{0.15in}

The RecordDesc is stored as follows:

\vspace{0.15in}
\begin{tabular}{|l|p{13cm}|} \hline
  Data Type & Description \\ \hline\hline
  AipsIO header & RecordDesc (version 2) \\
  uInt & number of keywords \\
  String & keyword name \\
  Int & data type (from DataType.h) \\
  Specific & specific info for a value containing an array, Table, or Record.
  \tabularnewline & - shape for an array (shape [-1] means it is not
  fixed). A shape is stored as a vector of Int values.
  \tabularnewline & - TableDesc name for a Table (empty name means it is not fixed).
  \tabularnewline & - description for a Record. \\
  String & comment to explain the meaning of a keyword (not in version 1) \\
  \hline
\end{tabular}
\vspace{0.15in}

The latter 4 fields are repeated for each keyword.

\section{PlainTable}
A PlainTable object represent a basic CaCTS table with its own
keywords, columns, and associated data managers. It is a directory
containing several files named \texttt{table.*}.
\\Besides the files \texttt{table.dat} and \texttt{table.info} a
PlainTable consists of the following files.
\begin{itemize}
\item
\texttt{table.lock} is used to facilitate concurrent access
to a table. Locks are acquired and released for this file. Furthermore
it contains information telling another process if a table has changed.
\item
The files \texttt{table.f<i>*} are used by storage managers to store
the data. The storage managers are numbered 0..n and use the number as
the suffix \texttt{i} in the file name. Note that the numbers do not
need to be consecutive. This can happen in case a storage manager has
been deleted or a virtual data manager is used.
\end{itemize}


\subsection{General file table.dat}
This file is the main table file. It contains the table definition,
the keywords and their values, and the binding of columns to data
managers. It is basically a single AipsIO object containing the
following fields.

\vspace{0.15in}
\begin{tabular}{|l|l|} \hline
  Data Type & Description \\ \hline\hline
  AipsIO header & Table (version 2) \\
  uInt & number of table rows \\
  uInt & endianness (0=little, 1=big) \\
  String & table type (PlainTable) \\
  TableDesc & table description and table keyword set \\
  TableRecord & table keyword set (in version 1 only) \\
  ColumnSet & column data manager info and column keyword sets \\
  \hline
\end{tabular}
\vspace{0.15in}

Note that the set of table and column keywords is part of the table
description.
Only in the very first CaCTS version (before 1995), the table
keyword set was stored separately. For backward compatibility it is
still possible to read such a keyword set, but it is not stored
anymore.

It is instructive to use the UNIX \texttt{strings} command on a
table.dat file. It shows all the strings mentioned in this section.


\subsubsection{Table Description}
The TableDesc object contains the table description defining 
the logical view of a table. It contains
the name, data type, and some extra info of each column. Furthermore
it contains the keyword set attached to the table and each column.
Finally it contains a private keyword set that is used by the
TableDesc object to hold some extra meta data.

\vspace{0.15in}
\begin{tabular}{|l|p{13cm}|} \hline
  Data Type & Description \\ \hline\hline
  AipsIO header & TableDesc (version 2) \\
  String & name of table description \\
  String & version info of table description \\
  String & comment about table description\\
  TableRecord & table keyword set \\
  TableRecord & private table keyword set (not in version 1) \\
  uInt & number of columns \\
  ColumnDesc & info of each column \\
  \hline
\end{tabular}
\vspace{0.15in}

The info of each column is stored in a standard way.

\vspace{0.15in}
\begin{tabular}{|l|p{13cm}|} \hline
  Data Type & Description \\ \hline\hline
  uInt & ColumnDesc version (=1) \\
  String & Type of column using the name of the object containing the
  description. It can have one of the following values:
  \tabularnewline & - ScalarColumnDesc$<$T
  \tabularnewline & - ArrayColumnDesc$<$T
  \tabularnewline & - ScalarRecordColumnDesc
  \tabularnewline & - SubTableDesc
  \tabularnewline & where T is a string defining the type. It can be:
  Bool, uChar, Short, uShort, Int, uInt, float, double, Complex, DComplex,
  and String. \\
  \hline
  uInt & BaseColumnDesc version (=1) \\
  String & column name \\
  String & comment \\
  String & default data manager type \\
  String & default data manager group \\
  Int & data type (like TpInt from DataType.h>. It must match the type
  in the column type string. \\
  Int & column options
  \tabularnewline & 1 = direct array; meant for small fixed sized
  arrays for which a storage manager can decide to put it directly.
  \tabularnewline & 2 = undefined value can exist (not used).
  \tabularnewline & 4 = array has a fixed shape. \\
  uInt & array dimensionality (-1 means scalar) \\
  IPosition & fixed array shape (only for columns containing arrays) \\
  uInt & maximum length of a string value (0 = no maximum) \\
  TableRecord & column keyword set \\
  \hline
  \multicolumn{2}{|c|}{For a column containing scalars} \\
  \hline
  uInt & ScalarColumnDesc version (=1) \\
  T & default value (stored according to the data type) \\
  \hline
  \multicolumn{2}{|c|}{For a column containing arrays} \\
  \hline
  uInt & ArrayColumnDesc version (=1) \\
  Bool & dummy flag (always False); is present for backward compatibility \\
  \hline
  \multicolumn{2}{|c|}{For a column containing records} \\
  \hline
  uInt & RecordScalarColumnDesc version (=1) \\
  \hline
  \multicolumn{2}{|c|}{For a column containing subtables} \\
  \hline
  uInt & SubTableDesc version (=1) \\
  String & TableDesc name \\
  Bool & Flag telling how the description of the subtable is found.
  \tabularnewline & True = table description is by name, thus in a
  file with that name.
  \tabularnewline & False = table description is in the next field. \\
  TableDesc & the description of the subtable (only if flag=False) \\
  \hline
\end{tabular}
\vspace{0.15in}

\begin{itemize}
\item A column containing records makes it possible to store arbitrary
  data in a column. It is handed as an AipsIO byte stream (vector of
  uChar) to the data managers.
\item A column containing subtables is supported in the description, but it
  has never been fully implemented in CaCTS, so it cannot be used. This
  feature is only used in a test program.
\end{itemize}

\subsubsection{ColumnSet}

\subsection{Info file table.info}
This file is a text file giving some brief info about the table. It is
read and written by class \texttt{TableInfo}.
\\It contains the following lines:
\begin{itemize}
\item The table type like \texttt{Type = MeasurementSet}
\item The table subtype like\texttt{SubType =}
\item Zero or more lines giving some more explanation.
\end{itemize}

\subsection{Lock file table.lock}
CaCTS supports concurrent access to a table. One writer or multiple
readers can be active at the same time. A lock needs to be acquired to
access the table, although there is a mode to bypass locking when
reading. See \htmladdnormallink{note 256}{../256.html}
for more info about how locking is done.

The first few bytes in the lock file are used by the concurrency
mechanism of CaCTS to manage a read or write lock (on the entire table).
\begin{itemize}
\item The first byte is used to acquire/release a lock using the system's
  file locking mechanism (\texttt{fcntl}). The lock can be shared (for
  read) or exclusive (for write).
\item The second byte is used to tell other processes that the table
  is opened. This is done by creating a shared lock on it.
\item The third byte is used to tell other processes that a process holds
  a permanent lock on the table. It is also using a shared lock.
\item The fourth till sixth byte are used for special internal lock purposes.
\end{itemize}

The lock file contains two data structures to assist processes in
acquiring locks and synchronizing their internal data structures with
table data changes made in another process.
All data are written in big-endian format.
\begin{itemize}
\item A list of process identifications is kept in the first
  260 bytes of the file. The list contains the host and process ids of
  processes wanting to acquire a  lock, but cannot because
  another process is holding the lock.
  \\The list consists of 4-byte integers, where the first value contains
  the number of processes in the list. The list can contain up  to 32 processes.
\item Info telling if and how table data have changed. It is written
  when a write lock is released. It is used by a process
  acquiring a lock to synchronize its internal data objects.
  \\The info is written as an AipsIO stream described below. Bytes
  260-263 of the lock file contain the length of the info
  stream. Thereafter the AipsIO stream is written. 
\end{itemize}

\vspace{0.15in}
\begin{tabular}{|l|p{13cm}|} \hline
  Data Type & Description \\ \hline\hline
  AipsIO header & sync (version 1) \\
  uInt & number of table rows \\
  uInt & number of table columns \\
  uInt & modify counter \\
  uInt & table change counter. The counter is incremented if the
  table.dat file has changed (e.g. by adding a keyword). \\
  Block$<$uint$>$ & change counter per data manager. A counter is
  incremented if that data manager has written data. \\
  \hline
\end{tabular}
\vspace{0.15in}

A process acquiring a lock detects if the main table file or a data
manager file has changed by comparing its change counters with the
ones in the lock file.

\subsection{Storage Managers}

\subsubsection{StandardStMan}
- header block
- index
- data

\subsubsection{IncrementalStMan}

\subsubsection{TiledStMan}
- index
- cell, column, shape stman

\subsubsection {AipsIOStMan}

\subsection{Virtual Column Engines}

\section{RefTable}

\section{ConcatTable}
