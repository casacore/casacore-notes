\documentclass{article}

\pagestyle{plain}

\newcommand{\subsectionC}[4]{
  {\bfseries \vspace{-\baselineskip} \vspace{-\baselineskip}

   \begin{longtable}[l]{lrl}
    #1 -- &Constructor & \kill\\
          & Package & #2\\
          & Module & #3\\
          & Tool & #4\\
    \end{longtable}

  }
  \flushleft\vspace{\baselineskip}
}

\newcommand{\subsectionF}[3]{
  {\bfseries\large \vspace{-\baselineskip} \vspace{-\baselineskip}
   \begin{longtable}[l]{lrl}
    #1 -- &Function & \kill\\
      & Package & #2\\
      & Module  &#3\\
   \end{longtable}

  }
  \flushleft\vspace{\baselineskip}
}

\newcommand{\subsectionM}[4]{
  {\bfseries \vspace{-\baselineskip} \vspace{-\baselineskip}
   \begin{longtable}[l]{lrl}
     \thesubsection #1 -- & Tool & \kill\\
     &Package  &#2\\
     &Module  &#3\\
     &Tool  &#4\\
  \end{longtable}

  }
  \flushleft\vspace{\baselineskip}
}

\newcommand{\subsectionO}[3]{
  {\bfseries \large \vspace{-\baselineskip} \vspace{-\baselineskip}

   \begin{longtable}[l]{lrl}
     \thesubsection #1 -- & Tool & \kill\\
     & Package & #2\\
     & Module  &#3\\
   \end{longtable}
  }
  \flushleft\vspace{\baselineskip}
}

%%%%%%%%%%%%%%%%%%%%%%%%%%%%%%%%%%%%%%%%%%%%%%%%%%%%%%%%%%%%%%%%%%%%%%%%%%%%%
%%
%% Generic .help file definitions
%%
\newcommand{\aipspp}{\textsf{AIPS++}}
\newcommand{\glish}{\emph{Glish}}
\newcommand{\av}{\textit{AipsView}}
\newcommand{\tool}               {{\tt tool}}
\newcommand{\tools}              {{\tt tools}}
\newcommand{\toolfunction}       {{\tt tool function}}
\newcommand{\toolfunctions}      {{\tt tool functions}}
\newcommand{\constructor}        {{\tt constructor}}
\newcommand{\constructors}       {{\tt constructors}}
\newcommand{\fits}               {{FITS}}
\newcommand{\uvfits}             {{UVFITS}}
%
\newcommand{\viewer}             {Viewer}
\newcommand{\regionmanager}      {Regionmanager}
\newcommand{\toolmanager}        {Toolmanager}
%%
%%%%%%%%%%%%%%%%%%%%%%%%%%%%%%%%%%%%%%%%%%%%%%%%%%%%%%%%%%%%%%%%%%%%%%%%%%%%%
%% module  font
\newcommand{\mf} {\tt}

%% glish code font
\newcommand{\gcf} {\tt}

%% specific tool font
\newcommand{\stf} {\bf}

%% specific file font
\newcommand{\sff} {\em}
   
%% specific string font
\newcommand{\ssf} {\em}

%% GUI item font
\newcommand{\gif} {\sf}

%% Unix application
\newcommand{\uaf} {\sl}

%%%%%%%%%%%%%%%%%%%%%%%%%%%%%%%%%%%%%%%%%%%%%%%%%%%%%%%%%%%%%%%%%%%%%%%%%%%%%%
%%
%% The following definitions are used by NEBK
%% in a range of   image related .help files

%% specific tool function font
\newcommand{\stff} {\tt}
%% specific tool function argument font
\newcommand{\stfaf} {\tt}
%% glish variable font
\newcommand{\gvf} {\tt}
%% code font
\newcommand{\cf} {\tt}
%%%%%%%%%%%%%%%%%%%%%%%%%%%%%%%%%%%%%%%%%%%%%%%%%%%%%%%%%%%%%%%%%%%%%%%%%%%%%%%%
 
%Generic
 
%% Specific
\newcommand{\imagefile}          {{\tt image file}}
\newcommand{\imagefiles}         {{\tt image files}}
\newcommand{\imagetool}          {{\tt image tool}}
\newcommand{\imagetools}         {{\tt image tools}}   
\newcommand{\imagetoolfunction}  {{\tt image tool function}}
\newcommand{\imagetoolfunctions} {{\tt image tool functions}}
%%
\newcommand{\region}             {{\tt region-of-interest}}
\newcommand{\regions}             {{\tt regions-of-interest}}
\newcommand{\Regions}             {{\tt Regions-of-interest}}
\newcommand{\regionmask}             {{\tt region mask}}
\newcommand{\pixelmask}             {{\tt pixel mask}}
\newcommand{\pixelmasks}             {{\tt pixel masks}}
\newcommand{\Pixelmasks}             {{\tt Pixel masks}}
%%
%%%%%%%%%%%%%%%%%%%%%%%%%%%%%%%%%%%%%%%%%%%%%%%%%%%%%%%%%%%%%%%%%%%%%%%%%%%%%%%%



\usepackage[colorlinks]{hyperref}
\usepackage{epsf}
\setlength{\textwidth}{6.5in}

\usepackage{titlesec}

\setcounter{secnumdepth}{4}

\titleformat{\paragraph}
{\normalfont\normalsize\bfseries}{\theparagraph}{1em}{}
\titlespacing*{\paragraph}
{0pt}{3.25ex plus 1ex minus .2ex}{1.5ex plus .2ex}

\setlength{\evensidemargin}{0.0in}
\setlength{\oddsidemargin}{0.0in}
\parskip=5truept
\newcommand{\nc}{$N_c$}
\newcommand{\nr}{$N_r$}
\newcommand{\nf}{$N_f$}
\newcommand{\nfs}{$N_f^*$}
\newcommand{\nl}{$N_l$}
\newcommand{\ncat}{$N_{cat}$}
\newcommand{\na}{$N_{ant}$}
\newcommand{\defline}[1]{\cline{1-5}
\multicolumn{5}{|l|}{#1} \\
\cline{1-5}}
\newcommand{\definetable}[2]
{
	\vfill\newpage
	\subsection{#1}
        \vspace{0.15in}
        \small
	\begin{tabular}{|l|p{1.25in}|l|p{.9in}|p{1.4in}|}
	\hline
	\multicolumn{5}{|c|}{\bf #1}\\ 
	\cline{1-5}
        \multicolumn{1}{|c|}{Name}&\multicolumn{1}{|c|}{Format}&
        \multicolumn{1}{|c|}{Units}&\multicolumn{1}{|c|}{Measure}&
        \multicolumn{1}{|c|}{Comments}\\
        \cline{1-5}
        #2
        \hline
	\end{tabular}
}


\begin{document}

\title{MeasurementSet VLBI extensions}

\author{Kettenis M.M.}
\date{November, 2023}
\maketitle

\ifpdf
\else
\href{265.pdf}{A pdf version of this note is available.}
\fi

\tableofcontents 



\section{Introduction}

The MeasurementSet (MS) defines a format in which interferometer
visibilities and single-dish data are stored.  It is implemented in
software packages using casacore code.  Version 2.0 (Note 229; Kemball
and Wieringa 2000) of the MeasurementSet has been in use since 2000 in
different packages and at different telescopes.

Recent developments to add VLBI capabilities to CASA revealed some
gaps in the metadata provided by MS V2.0.  To fill these gaps
additional tables have been defined and implemented in casacore.
These extension tables are filled by the code that implements the
conversion from FITS-IDI to MS.  This follows the existing practice
established by the code that converts from ASDM to MS that is provided
in CASA where several ASDM-specific tables are stored in MS form with
names wit an ASDM\_ prefix.

The following new tables are defined:

\begin{itemize}

\item{{\bf GAIN\_CURVE:} Antenna gain curves}

\item{{\bf PHASE\_CAL:} Phase calibration measurements}

\item{{\bf EARTH\_ORIENTATION:} Earth Orientation Parameters}

\end{itemize}


\section{VLBI metadata tables}

This section contains a description of the VLBI metadata tables.
These tables are stored as subtables of the MS and like the standard
subtables these are stored as keywords of the MS.  All these subtables
are optional.

\definetable{GAIN\_CURVE: Antenna gain curves}
{
\defline{\bf Columns}
\defline{\bf Keys}
ANTENNA\_ID & Int        &      &       & Antenna id.\\
FEED\_ID    & Int        &      &       & Feed id.\\
SPECTRAL\_WINDOW\_ID & Int &    &       & Spectral window id.\\
TIME        & Double     & s    & EPOCH & Interval midpoint \\
INTERVAL    & Double     & s    &       & Time interval\\
\defline{\em Data description}
TYPE        & String     &      &       & Gain curve type \\
NUM\_POLY   & Int        &      &       & \# Series order\\
\defline{\em Data}
GAIN        & Float(\nr, num\_poly) & & & Gain\\
SENSITIVITY & Float(\nr) & K/Jy &       & Sensitivity\\
}

\begin{description}

\item[Notes:] This sub-table contains antenna gain curves.\\
  For further discussion of this table see casacore
  \htmladdnormallink{Issue
    1029}{https://github.com/casacore/casacore/issues/1039}.

\item[ANTENNA\_ID] Antenna identifier, as indexed by ANTENNA$n$ from MAIN.

\item[FEED\_ID] Feed identifier, as indexed from FEED$n$ in MAIN. A
  value of -1 indicates the row is valid for all feeds.

\item[SPECTRAL\_WINDOW\_ID] Spectral window identifier. A value of -1
  indicates the row is valid for all spectral windows.

\item[TIME] Mid-point of the time interval over which the data in the
  row are valid. Required to use the same TIME Measure reference as in
  MAIN.

\item[INTERVAL] Time interval.

\item[TYPE] Gain curve type. Reserved keywords include:
  ("POWER(EL)" - Power as a function of elevation; "POWER(ZA)" - Power
  as a function of zenith angle; "VOLTAGE(EL)" - Voltage as a function
  of elevation; "VOLTAGE(ZA)" - Voltage as a function of zenith
  angle).

\item[NUM\_POLY] Series order for the GAIN column (number of terms in
  the gain curve polynomial).

\item[GAIN] Coefficients of the polynomial that describes the (power
  or voltage) gain.

\item[SENSITIVITY] Sensitivity of the antenna expressed in K/Jy. This
  is what AIPS calls ``DPFU''.
  
\end{description}

\definetable{PHASE\_CAL: Phase calibration measurements}
{
\defline{\bf Columns}
\defline{\bf Keys}
ANTENNA\_ID & Int         &      &       & Antenna id.\\
FEED\_ID    & Int         &      &       & Feed id.\\
SPECTRAL\_WINDOW\_ID & Int &     &       & Spectral window id.\\
TIME        & Double      & s    & EPOCH & Interval midpoint \\
INTERVAL    & Double      & s    &       & Time interval\\
\defline{\em Data description}
NUM\_TONES  & Int         &      &       & \# phase calibration tones\\
TONE\_FREQUENCY & Double(num\_tones) & Hz & & Tone frequencies\\
\defline{\em Data}
PHASE\_CAL  & Float(\nr, num\_tones) & & & Phase-Cal tone measurements\\
CABLE\_CAL  & Double(\nr) & s    &       & Cable delay\\
}
\begin{description}

\item[Notes:] This sub-table contains signal chain phase calibration
  measurements.\\
  For further discussion of this table see casacore
  \htmladdnormallink{Issue
    1165}{https://github.com/casacore/casacore/issues/1165}.

\item[ANTENNA\_ID] Antenna identifier, as indexed by ANTENNA$n$ from MAIN.

\item[FEED\_ID] Feed identifier, as indexed from FEED$n$ in MAIN. A
  value of -1 indicates the row is valid for all feeds.

\item[SPECTRAL\_WINDOW\_ID] Spectral window identifier. A value of -1
  indicates the row is valid for all spectral windows.

\item[TIME] Mid-point of the time interval over which the data in the
  row are valid. Required to use the same TIME Measure reference as in
  MAIN.

\item[INTERVAL] Time interval.

\item[NUM\_TONES] Number of phase-cal tones that are measured. This
  number may vary by antenna, and may vary by spectral window as well,
  especially if spectral windows of varying widths are supported.

\item[TONE\_FREQUENCY] The sky frequencies of each measured phase-cal tone.

\item[PHASE\_CAL] Phase calibration measurements. These are provided
  as complex values that represent both the phase and amplitude for a
  measured phase-cal tone. Measurements are provided as a
  two-dimensional array such that separate measurements can be
  provided for each receptor of a feed (so separate values for each
  polarization) for each of the measured tones.

\item[CABLE\_CAL] Cable calibration measurement. This is a measurement
  of the delay in the cable that provides the reference signal to the
  receiver. There should be only a single reference signal per feed
  (even if that feed has multiple receptors) so this is provided as a
  simple scalar.

\end{description}

\definetable{EARTH\_ORIENTATION: Earth orientation parameters}
{
\defline{\bf Columns}
\defline{\bf Keys}
TIME        & Double    & s    & EPOCH & Time-stamp for values\\
OBSERVATION\_ID & Int   &      &       & Points to OBSERVATION table\\
\defline{\em Data}
UT1\_UTC    & Double    & s    &       & UT1 - UTC\\
PM          & Double(2) & rad  &       & Celestial pole position\\
TYPE        & String    &      &       & Type of EOP value\\
}
\begin{description}

\item[Notes:] This sub-table contains the earth orientation parameters
  that were used to generate the correlator model.\\
  For further discussion of this table see casacore
  \htmladdnormallink{Issue
    1307}{https://github.com/casacore/casacore/issues/1307}.

\item[TIME] Timestamp of the EOP values; this is expected to be the
  UTC time although some sources use TAI instead.

\item[OBSERVATION\_ID] Observation identifier (see the OBSERVATION table)

\item[UT1\_UTC] Difference between UT1 and UTC (UT1 - UTC) in seconds.

\item[PM] Position of celestial pole; these are provided as X and Y
  components (in that order) in radians.

\item[TYPE] Type of the EOP value; Reserved keywords include:
  ("PREDICTED" - for predicted parameter values; "PRELIMINARY" - for
  measured parameter values that are subject to further revision in
  the future; "FINAL" - for final parameter values).  If the type is
  not know this should be an empty string.

\end{description}


\section{References}

\noindent Kemball, A.J., Wieringa, M.H., 2000, casacore Note 229.

\end{document}
