\documentclass{article}

\pagestyle{plain}

\newcommand{\subsectionC}[4]{
  {\bfseries \vspace{-\baselineskip} \vspace{-\baselineskip}

   \begin{longtable}[l]{lrl}
    #1 -- &Constructor & \kill\\
          & Package & #2\\
          & Module & #3\\
          & Tool & #4\\
    \end{longtable}

  }
  \flushleft\vspace{\baselineskip}
}

\newcommand{\subsectionF}[3]{
  {\bfseries\large \vspace{-\baselineskip} \vspace{-\baselineskip}
   \begin{longtable}[l]{lrl}
    #1 -- &Function & \kill\\
      & Package & #2\\
      & Module  &#3\\
   \end{longtable}

  }
  \flushleft\vspace{\baselineskip}
}

\newcommand{\subsectionM}[4]{
  {\bfseries \vspace{-\baselineskip} \vspace{-\baselineskip}
   \begin{longtable}[l]{lrl}
     \thesubsection #1 -- & Tool & \kill\\
     &Package  &#2\\
     &Module  &#3\\
     &Tool  &#4\\
  \end{longtable}

  }
  \flushleft\vspace{\baselineskip}
}

\newcommand{\subsectionO}[3]{
  {\bfseries \large \vspace{-\baselineskip} \vspace{-\baselineskip}

   \begin{longtable}[l]{lrl}
     \thesubsection #1 -- & Tool & \kill\\
     & Package & #2\\
     & Module  &#3\\
   \end{longtable}
  }
  \flushleft\vspace{\baselineskip}
}

%%%%%%%%%%%%%%%%%%%%%%%%%%%%%%%%%%%%%%%%%%%%%%%%%%%%%%%%%%%%%%%%%%%%%%%%%%%%%
%%
%% Generic .help file definitions
%%
\newcommand{\aipspp}{\textsf{AIPS++}}
\newcommand{\glish}{\emph{Glish}}
\newcommand{\av}{\textit{AipsView}}
\newcommand{\tool}               {{\tt tool}}
\newcommand{\tools}              {{\tt tools}}
\newcommand{\toolfunction}       {{\tt tool function}}
\newcommand{\toolfunctions}      {{\tt tool functions}}
\newcommand{\constructor}        {{\tt constructor}}
\newcommand{\constructors}       {{\tt constructors}}
\newcommand{\fits}               {{FITS}}
\newcommand{\uvfits}             {{UVFITS}}
%
\newcommand{\viewer}             {Viewer}
\newcommand{\regionmanager}      {Regionmanager}
\newcommand{\toolmanager}        {Toolmanager}
%%
%%%%%%%%%%%%%%%%%%%%%%%%%%%%%%%%%%%%%%%%%%%%%%%%%%%%%%%%%%%%%%%%%%%%%%%%%%%%%
%% module  font
\newcommand{\mf} {\tt}

%% glish code font
\newcommand{\gcf} {\tt}

%% specific tool font
\newcommand{\stf} {\bf}

%% specific file font
\newcommand{\sff} {\em}
   
%% specific string font
\newcommand{\ssf} {\em}

%% GUI item font
\newcommand{\gif} {\sf}

%% Unix application
\newcommand{\uaf} {\sl}

%%%%%%%%%%%%%%%%%%%%%%%%%%%%%%%%%%%%%%%%%%%%%%%%%%%%%%%%%%%%%%%%%%%%%%%%%%%%%%
%%
%% The following definitions are used by NEBK
%% in a range of   image related .help files

%% specific tool function font
\newcommand{\stff} {\tt}
%% specific tool function argument font
\newcommand{\stfaf} {\tt}
%% glish variable font
\newcommand{\gvf} {\tt}
%% code font
\newcommand{\cf} {\tt}
%%%%%%%%%%%%%%%%%%%%%%%%%%%%%%%%%%%%%%%%%%%%%%%%%%%%%%%%%%%%%%%%%%%%%%%%%%%%%%%%
 
%Generic
 
%% Specific
\newcommand{\imagefile}          {{\tt image file}}
\newcommand{\imagefiles}         {{\tt image files}}
\newcommand{\imagetool}          {{\tt image tool}}
\newcommand{\imagetools}         {{\tt image tools}}   
\newcommand{\imagetoolfunction}  {{\tt image tool function}}
\newcommand{\imagetoolfunctions} {{\tt image tool functions}}
%%
\newcommand{\region}             {{\tt region-of-interest}}
\newcommand{\regions}             {{\tt regions-of-interest}}
\newcommand{\Regions}             {{\tt Regions-of-interest}}
\newcommand{\regionmask}             {{\tt region mask}}
\newcommand{\pixelmask}             {{\tt pixel mask}}
\newcommand{\pixelmasks}             {{\tt pixel masks}}
\newcommand{\Pixelmasks}             {{\tt Pixel masks}}
%%
%%%%%%%%%%%%%%%%%%%%%%%%%%%%%%%%%%%%%%%%%%%%%%%%%%%%%%%%%%%%%%%%%%%%%%%%%%%%%%%%



\usepackage[colorlinks]{hyperref}
\usepackage{epsf}
\setlength{\textwidth}{6.5in}
\setlength{\evensidemargin}{0.0in}
\setlength{\oddsidemargin}{0.0in}
\parskip=5truept
\newcommand{\nc}{$N_c$}
\newcommand{\nr}{$N_r$}
\newcommand{\nf}{$N_f$}
\newcommand{\nfs}{$N_f^*$}
\newcommand{\nl}{$N_l$}
\newcommand{\ncat}{$N_{cat}$}
\newcommand{\defline}[1]{\cline{1-5}
\multicolumn{5}{|l|}{#1} \\
\cline{1-5}}
\newcommand{\definetable}[2]
{
	\vfill\newpage
	\subsection{#1}
        \vspace{0.15in}
        \small
	\begin{tabular}{|l|p{1.25in}|l|p{.9in}|p{1.4in}|}
	\hline
	\multicolumn{5}{|c|}{\bf #1}\\ 
	\cline{1-5}
        \multicolumn{1}{|c|}{Name}&\multicolumn{1}{|c|}{Format}&
        \multicolumn{1}{|c|}{Units}&\multicolumn{1}{|c|}{Measure}&
        \multicolumn{1}{|c|}{Comments}\\
        \cline{1-5}
        #2
        \hline
	\end{tabular}
}
\begin{document}

\title{MeasurementSet definition version 3.0}
\author{eds.}
\date{January, 2018}
\maketitle

\ifpdf
\else
\href{264.pdf}{A pdf version of this note is available.}
\fi

\tableofcontents 
\section{Summary} 

This note describes a revised MeasurementSet definition which extends
that previously given in casa/casacore Note 229(Kemball and Wieringa
2000).  The new definition is designed to support features that were
felt missing from v2.0 or tedious to deal with, provide support for
VLBI processing and non dish arrays.

\section{Introduction}

The MeasurementSet (MS) defines the format in which visibility and
single-dish data are stored for packages using casa/casacore code.


\begin{itemize}

\item{{\bf VLBI data reduction:} From v2.0 added an optional subtable INTERFEROMETER_MODEL}

\item{{\bf Synthesis calibration:} The need for calibrations table as
    part the MeasurementSet has been felt by several telescopes and
    those using v2.0 (e.g ALMA and EVLA) have been using non standard
    tables to carry calibration type information.}

\item{{\bf Single-dish processing:} FLOAT_DATA is going to be replaced
    by DATA unifying it with interferometer. It is left to the column
    implementation to store floats efficiently rather than have
    multiple columns defined for data types }

\item{{\bf Explicit keys} One of the major complaints of ms v2.0 is
    the use of row ids in subtables as implicit keys. In v3.0 we are
    moving to explicit keys; e.g ANTENNA subtable will have a
    column ANTENNA_ID so that when making a subset MS re-indexing is
    not needed in the main table.}

\item{{Data,weight, flag versioning} Non standard columns (e.g
    CORRECTED_DATA) or kludges have been used by casa (for e.g) to
    deal with having version of data, flag or weight. These 3 columns
    are having the same shapes in every row of an msv3.0.  Whenever
    changes are made example flagging or calibration and a new version
    is needed. A new triplet of columns will be made. It is left to
    software implementation to do the smart thing (not making
    unnecessary copies). E.g if flagging only is done and need to be
    saved in a new version of FLAG column, the new version of DATA an
    WEIGHT are going to be just virtual columns pointing back to the
    previous version.

\end{itemize}

This is an opportune time to revise the MeasurementSet, as an
increased level of synthesis development is underway. Later revisions
will have a broader impact on existing code.

The design philosophy underlying this MS definition is summarized in
terms of the following objectives:

\begin{itemize}

\item{{\bf Incremental change:} The changes proposed here are designed
to be as incremental as possible and no extensive re-design has been
attempted. The scientific benefit of each modification has been weighed
against the scope of the proposed change to the MS design.}

\item{{\bf Compatibility:} Compatibility between single-dish and
synthesis data has been retained within one basic MS definition. The
proposed VLBI extensions are constrained to be compatible with both
the basic synthesis format and across existing VLBI networks and
correlators.}

\item{{\bf Separation of information:} A fundamental distinction is
made between a priori information which is known at the time of
observation or shortly thereafter, and calibration information
subsequently derived in post-processing. The MS definition is
primarily designed to encompass a priori information. The format of
calibration tables in given elsewhere.}

\item{{\bf Storage:} A future document will define standard Data
Managers for each column, which are recommended but not
required. Attention has been paid to the physical file sizes implied
by the MS definition, particularly for larger datasets, and storage
managers for a standard compressed MS format will be discussed in the
same document.}

\item{{\bf Combining measurement sets:} The MS definition has been
revised to facilitate the combining of diverse observational datasets
within one MeasurementSet if required, in a manner that is compatible
with the calibration system. This does not hold the implication that
all observations to be processed jointly require that the underlying
MeasurementSets be combined. Applications will support the ability to
process groups of MeasurementSets. However, the facility to combine
data needs to be provided in the MeasurementSet design.

 The combining of different observations within one MeasurementSet
is permitted subject to the conventions of Section 3.3, which
separates data primarily by OBSERVATION\_ID, but also by
PROCESSOR\_ID and ARRAY\_ID. Cross-calibration between observations
is however subject to the capabilities of the calibration
system. Specialized inter-conversion of calibration information may be
necessary if the calibration in separate observations is sufficiently
disjoint. For example, transferring calibration between frequencies
sufficiently far apart will likely be performed in an intermediate
step, using external utilities.

 An application is envisaged to combine data from one or more
separate MeasurementSets, creating a new output MS by copying the
input data row by row, and renumbering the indices and interleaving
data as required. No sort order is prescribed for a MeasurementSet in
general, but it is expected that data sorted in time order will be
most useful to the broadest range of applications.}

\end{itemize}

Specific principles adopted in the design are given below:

\begin{itemize}

\item{{\bf Signal path:} The MeasurementSet provides a format to
represent data from a generic radio-telescope or interferometer. Along
with the basic observed data, for which a limited set of accepted
types are specified in the MAIN table section, there are associated
data characterizing the state of the instrument as a whole. These
include: i) abstracted antenna properties of components in a generic
telescope, such as feeds and spectral windows, which serve also to
label the output data; ii) external information, such as flagging,
history or weather data; and iii) instrument-specific back-end data
which may be difficult to represent in a completely generic form. The
final category includes intermediate data and state information from
devices such as correlators, radiometers, spectrometers or pulsar
timers, amongst others, where they do not overlap with abstracted
antenna properties defined elsewhere. This state information may be
used in computing or initializing calibration corrections. Thus the MS
represents the signal path and state of the instrument, using as
generic an interface as possible, but allowing specialization where
appropriate. This conforms closely with the overall calibration
model.}

\item{{\bf Use of Measures:} Some columns in the MeasurementSet
require coordinate and unit specification. This is done in a manner
compatible with the AIPS++ Measures system. Measure frame information
is implicit in the underlying MS data. Row-based measures are avoided
wherever possible due to the overhead this would often impose on the
data reduction system through frequent coordinate conversion to a
common reference frame. Thus, column-based measures are the default
unless otherwise noted. The only place in which a row-based Measure
reference is currently allowed is the frequency axis in the
SPECTRAL\_WINDOW sub-table, where it is supplied in order to allow an
efficient representation of Doppler tracking. Column-based Measures of
a specific type (e.g. EPOCH), should have a common reference across
the MS as a whole, unless there is a compelling reason otherwise. This
requirement is necessary for efficiency, and in minimizing
transformations when combining diverse MS datasets. No standard
reference is enforced for any given Measure type; these should be
chosen prudently. Units are also required to be column-based, unless a
row-based Measure is allowed for a given column. Recommended units are
specified for each Measure column. Access to the MS is assumed to take
place through the MS access classes. TableMeasures will be used
wherever practical in the MS access classes.}

\item{{\bf Relative indexing:} All indices for antennas, feeds,
spectral windows, or related quantities are assumed to start at zero.
Thus for direct indices into sub-tables, such as SPECTRAL\_WINDOW\_ID,
a value ID=$n$ maps to row=$n$. MS indexing is zero-based in C++, and
one-based in Glish, as before.}

\item{{\bf Sub-table completeness:} Not all data represented in
associated sub-tables are assumed to be present in MAIN, although this
is encouraged. For example, extra antennas may appear in the ANTENNA
sub-table, even if they have no associated data in MAIN.}

\item{{\bf Blanking and defaults:} Measured data are blanked by
setting associated Boolean flag data, and magic value blanking is not
used. For integer values constrained to be non-negative, or sub-table
indices, however, a value of -1 will generally denote an unset
value. All required columns should be filled with suitable defaults if
not actually used.}

\item{{\bf Times and intervals:} For quantities which may vary with
time, an INTERVAL of zero implies a constant value with no
time-dependence, while a negative INTERVAL implies that the value is
valid until re-defined. The latter case accommodates values that are
time-stamped but have an undefined or unknown period of validity.}

\item{{\bf Non-standard columns:} The naming convention for
non-standard columns remains the same as in MS definition v1.0,
namely: i) general, non-supported columns start with the prefix
NS\_\$SITE\_, as in NS\_NRAO\_WHATEVER; and ii) columns supported by a
consortium site start with the prefix \$SITE\_, as in
NRAO\_WHATEVER. Standard columns, such as those described in this
definition, are not permitted to use either of the above two prefix
forms. Non-standard columns may not be defined to store data already
stored in standard columns. In addition, applications are not required
to support non-standard columns.}

\item{{\bf Archiving:} MeasurementSet data are assumed to be archived
in an external format such as FITS or HDF (Hierarchical Data Format)
for example, in as lossless a format as possible. This format will be
specified separately. Changes to the MS format will not be made
without a compelling scientific reason, but the format will, of
necessity, evolve over time. The archiving application has the
responsibility to restore the data to the latest format. It is deemed
too complex for the underlying MS access code, and data reduction
code, to be made capable of recognizing multiple MS revisions at the
C++ level. A utility is envisaged, however, to convert MS v1.0 data to
the MS v2.0 format.}

\end{itemize}

Suggestions regarding this revision of the MS definition have been
contributed broadly from within the aips++ project. No specific
attribution is given for each change but the accompanying text
reflects the rationale behind the modification and includes relevant
points that were raised in public discussion.

\section{Summary of changes}

This section contains a description of the changes proposed for each
table in the MS definition. A full definition of the v2.0 MS format is
given in Section 5.

\subsection{MAIN table}

The fundamental assumption underlying the MAIN table is that it
contains interferometer or single-dish data, represented in
cross-power or autocorrelation form. The composite key is
baseline-based and auxiliary antenna-based data, that are consequently
not fully functionally dependent on the baseline-based key, are stored
in some of the associated sub-tables. As the latter data are often
time-variable, the design admits multi-key lookups in these tables, as
is already assumed in v1.0 (eg. SYSCAL table). In keeping with
realistic databases, the design of the MAIN table is not fully
normalized. The term is used here in the conventional database sense
(see for example, Date 1986). Adding indices to MAIN to avoid
sub-table lookups involves a trade-off between the anticipated
increase in the physical file size (which is storage manager
dependent), and the expected size of the sub-table and frequency with
which the auxiliary data need to be accessed in time-critical
applications. It also depends on whether the index is commonly used in
data selection. Note also that auxiliary data may be sampled on
different intervals, which is an important consideration in deciding
which antenna-based information should be placed in the same
sub-table. These factors have been balanced subject to the following
guidelines: i) storage manager independence has been retained
(i.e. the worst case is assumed regarding extra MAIN indices; ii)
single-key lookups in sub-tables are strongly favoured; and iii) it is
assumed that small sub-tables will be held in cache during MS access.

The changes proposed for the MAIN table are presented below:

\subsubsection{MAIN keywords}

It is proposed that the DEFAULT\_CALIBRATION keyword be removed, as it
is no longer used. 

The following additional keywords are proposed:

\begin{tabbing}
SORT\_COLUMNS\quad\quad  \= String\quad\quad  \=
(Indices over which MS is sorted)\quad\quad \= (Reqd.) \kill
MS\_VERSION \> Float \> (MS format revision) \> (Reqd.) \\
SORT\_COLUMNS  \> String  \> (Indices over which MS is sorted) \> (Opt.) \\
SORT\_ORDER \> String \> ('ASCENDING' or 'DESCENDING') \> (Opt.) \\
\end{tabbing}

 Keywords SORT\_COLUMNS and SORT\_ORDER are used to provide
information on the sort-order of the underlying MS.

\subsubsection{Non-standard MAIN data}

Three non-standard data types have been suggested for accommodation
within the standard MS definition, namely: (i) lag data; (ii)
triple-product data, and (iii) FLOAT\_DATA with an arbitrary axis other
than frequency.

The need to store lag data is motivated by the output format of XF
correlators and the requirements for data loading and inspection in
this case. Storage of triple-product data is required for optical
interferometry processing but is also useful for the implementation of
incoherent VLBI fringe-fitting algorithms. The requirement for
non-standard FLOAT\_DATA axes is concerned with more flexible
single-dish data formats.

Lag data can be incorporated by allowing an optional MAIN column:

\begin{tabbing}
LAG\_DATA\quad\quad \= Complex(NUM\_CORR,NUM\_LAG) \quad\quad \= 
(Lag function) \quad\quad \= (Opt.) \\
\end{tabbing}

A LAG\_ID index (via DATA\_DESC\_ID) points to a LAG table containing
the details of the lag spectrum, including range. Lag data are not
expected to be calibrated directly and would be supported primarily as
an interim format in an anticipated data reformatting sequence. The
routine storing of both lag and transformed spectral data in one MS
would often lead to prohibitive disk space requirements, and is not
anticipated here. Further information regarding the joint storing of
lag functions and frequency spectra in the same MS can be found in the
MAIN table description in Section 5.

Triple-product data can be incorporated in the MS definition by adding
additional antenna labels such as:

\begin{tabbing}
FEED3\quad\quad\quad\quad  \= Int\quad\quad  \=
(Feed on ANTENNA3)\quad\quad \= (Opt.)\quad\quad \kill
ANTENNA3 \> Int \> (Third antenna) \> (Opt.) \\
FEED3    \> Int  \> (Feed on ANTENNA3) \> (Opt.) \\
\end{tabbing}

The UVW column is similarly affected. The triple-product data would
be expected to be formed for matching time intervals, so other columns
such as SOURCE\_ID would be held to be common. The triple-product data
would be stored in the DATA column and the presence of a valid
ANTENNA3 column would label the data to be of this type. It is
stressed that triple-product and lag data will not have nearly equal
standing in the calibration system, as compared to conventional
frequency spectra.

Support for an arbitrary axis in the FLOAT\_DATA column, by
generalizing the SPECTRAL\_WINDOW\_ID index, is more difficult to
accommodate, and is best approached in the same manner as the lag
data, by enumerating special cases with separate indices such as
LAG\_ID. The lag and triple-product data formats are still
fundamentally visibility data and may also share at least some of the
broader synthesis infrastructure. Their impact on the sub-table
formats is also limited. This is not true for arbitrary data types in
the MS such as may result if the axis type in the FLOAT\_DATA
column was allowed to vary arbitrarily. 

 Accommodating non-standard columns (such as LAG\_DATA) in this way is
contingent on the requirement that they do not permit the
representation of identical data in multiple equivalent formats. This
would represent a significant application overhead and would affect
the global structure of the MS as a whole. Most importantly, this
would limit joint processing of data from separate instruments
(eg. single-dish and interferometry data).


\subsubsection{DATA labeling}

At present, polarization labeling for each correlator is stored in the
SPECTRAL\_WINDOW table, although such information is not properly
associated with frequency. Consequently, it is proposed that
polarization information be stored in a separate POLARIZATION
sub-table, indexed by POLARIZATION\_ID. However, rather than add
POLARIZATION\_ID to MAIN as an additional index, it is proposed that
SPECTRAL\_WINDOW\_ID be replaced by DATA\_DESC\_ID, which points to a
DATA\_DESCRIPTION sub-table containing SPECTRAL\_WINDOW\_ID and
POLARIZATION\_ID pairs. This structure has the specific advantage that
the DATA array shape is completely determined by DATA\_DESC\_ID, which
is an important consideration in tiling and uv-data access. The
DATA\_DESC sub-table will be small, so that secondary look-ups to
determine SPECTRAL\_WINDOW\_ID from DATA\_DESC\_ID can take place in
memory and will not represent a significant overhead.

Thus, SPECTRAL\_WINDOW\_ID in MAIN will be replaced by DATA\_DESC\_ID,
as follows:

\begin{tabbing}
DATA\_DESC\_ID\quad\quad \= Int\quad\quad \= (Data description index)
 \quad\quad \= (Reqd.) \\
\end{tabbing}

\subsubsection{Processor information}

As discussed in Section 2, provision needs to be made for the
characterization of the state of the specific back-end device used to
produce the final data. Such devices include correlators, radiometers,
spectrometers and pulsar timers, amongst others. These device data are not
permitted to duplicate general antenna properties represented
elsewhere, but are invariably highly instrument-specific. It is
proposed that the existing CORRELATOR\_ID index be replaced and
generalized as:

\begin{tabbing}
PROCESSOR\_ID\quad\quad \= Int\quad\quad \= (Processor identifier)
 \quad\quad \= (Reqd.) \\
\end{tabbing} 

This acts as an index into a PROCESSOR sub-table, which similarly
replaces the CORRELATOR sub-table, as described in Section 4.0.4.
This scheme accommodates parameter and status information associated
with each processor, generally by look-up into secondary sub-tables
containing this information using a subset of the MAIN key. For
single-dish data, however, the processor switching phase is
sufficiently fundamental to warrant inclusion as an optional MAIN
column, both for access efficiency and data labeling:

\begin{tabbing}
PHASE\_ID\quad\quad \= Int\quad\quad \= (Switching phase identifier)
 \quad\quad \= (Opt.) \\
\end{tabbing} 

\subsubsection{State information}

An associated STATE table is proposed to store combined information
relating to active reference signals or loads, as used in phase-switching, sub-scan information and a characterization of the current observing mode. This information is not fully represented in the MS at present, and is required for a more complete integration of single-dish processing in a common framework. An index in MAIN would be required as follows:

\begin{tabbing}
STATE\_ID\quad\quad \= Int\quad\quad \= (State identifier)
 \quad\quad \= (Reqd.) \\
\end{tabbing} 


\subsubsection{Pulsar information}

Synthesis correlation of pulsar data may involve the use of a pulsar gate,
which is a Boolean matrix in frequency and pulsar phase. The use of the
pulsar gate is either selected for a given baseline or not. The gate
may change with frequency group or source. No baseline dependence
seems likely but the gate could change with time. A general solution
is to implement a new MAIN table column as follows:

\begin{tabbing}
PULSAR\_GATE\_ID\quad\quad \= Int\quad\quad \= (Pulsar gate identifier)
 \quad\quad \= (Opt.) \\
\end{tabbing}

This is an index into the PULSAR\_GATE table containing the gate mask.
It is also proposed that the PULSAR\_ID column be absorbed in the
SOURCE table (via SOURCE\_ID), and removed from MAIN.  In this
revision of the MS definition, the PULSAR\_BIN column is made
optional.

\subsubsection{Baseline reference antenna}

For VLBI data from baseline-based correlators, record needs to be kept
of which antenna was used as the reference on each baseline during
correlation. The antenna sequence may be switched by the data filler
to preserve an ascending baseline numbering scheme. The reference
antenna needs to be known when shifting spectra in frequency to a
common velocity frame (eg. LSR). An optional column is proposed for
this purpose:

\begin{tabbing}
BASELINE\_REF\quad\quad \= Bool\quad\quad \= (True for ANTENNA1)
 \quad\quad \= (Opt.) \\
\end{tabbing}

\subsubsection{Flag data}

 The MS definition has been revised to recognize two distinct forms of
flagging, both of which need to be supported, namely: i) flagging most
easily expressed as global commands (eg. (ANTENNA1=3) AND (TIME $>$
12h03 UT)); and ii) highly time-variable flagging of a more random
nature, such as that resulting from mean-level clipping. In addition,
the flagging system recognizes two further requirements, namely that
point-by-point flags stored with the MS MAIN data are more efficient
than global flag commands when data are accessed, and secondly that
flags should be arbitrarily reversible in both representations.

 To this end, the multi-level Boolean column FLAG\_HISTORY in MAIN is
retained, but renamed to FLAG\_CATEGORY. It contains a record of
point-by-point flags associated with each MS row in separate
categories. The cumulative effect is stored in column FLAG, for all
categories combined. Categories may be arbitrarily defined by editing
applications, but several standard categories are reserved. This
includes a special category, FLAG\_CMD, which contains the cumulative
effect of flag (or un-flag) commands stored in the FLAG\_CMD
table. These commands are stored in the FLAG\_CMD table with
associated identifiers, including flag reason and level. The flag
reason indicates the type of flagging and the flag level the revision
of flagging to which the command belongs. Flag revisions are stored in
the point-by-point flags in FLAG\_CATEGORY by creating new, related
categories to store separate revisions of the same category if
required.

 Thus point-by-point or global flags can be reversed individually or
jointly. Both representations are scientifically necessary, and
supported. If the global flags are updated, it is the responsibility
of the flagging application to carry out the required one-time update
of the point-by-point flags in MAIN.

\subsubsection{Auto-correlation and single-dish data}

Single-dish data are represented by (ANTENNA1={\it n}, ANTENNA2= {\it
n}), and autocorrelation data by the secondary condition (FEED1={\it
m}, FEED2={\it m}). MeasurementSets are permitted containing a mixture
of FLOAT\_DATA and DATA columns. For autocorrelation data, however, in
any given row the FLOAT\_DATA column will be checked first, and, if
absent, the DATA column will be used.

\subsubsection{Hypercube indices}

A standardized naming convention for hypercube indices is proposed as
$<$column$>$\_HYPERCUBE\_ID (eg. DATA\_HYPERCUBE\_ID), where they are
specifically required by Tiled Storage Managers, as implemented in the
Table system.

\subsubsection{Video point}

 An optional data column is proposed in MAIN to store the video
point. This is preserved on some systems, and is required for a
complete transform between the frequency and lag domains. 

\begin{tabbing} 
VIDEO\_POINT\quad\quad \= Complex(NUM\_CORR)\quad\quad 
\= (Video point) \quad\quad \= (Opt.) \\
\end{tabbing}

 This representation is more transparent than encoding the information
in the DATA array, which leads to display and data selection
inconsistencies.

\subsubsection{Time centroid}

The TIME column in MAIN is defined as the mid-point of the nominal
sampling interval, as specified in the column, INTERVAL. The EXPOSURE
column, in contrast to INTERVAL, defines the effective integration
time, including the effects of missing data. The centroid of the
effective data interval is not represented in the current MS
definition, but does contain distinct information. It is added here as
a new non-key attribute:

\begin{tabbing}
TIME\_CENTROID\quad\quad \= Double\quad\quad \= (Centroid of
exposure) \quad\quad \= (Reqd.) \\
\end{tabbing}

\subsection{ANTENNA table}

The changes proposed for the ANTENNA table are presented below:

\subsubsection{ARRAY identification}

It is proposed that the antennas retain the same global numbering
scheme across the MS as a whole, and not be allowed to vary
independently for each ARRAY\_ID. Thus ARRAY\_ID is no longer a
secondary antenna index, but does retain its primary purpose to separate
antennas in the MAIN table for calibration purposes. There are
advantages to enforcing a global antenna numbering scheme,
particularly when antennas are shared between subarrays and may thus
share certain calibration information.

\subsubsection{Antenna type}

A new column is proposed to distinguish between tracking stations and
standard antennas (in which orbiting antennas are included):

\begin{tabbing}
TYPE\quad\quad \= String\quad\quad \= 
(Enumerated antenna type (eg. TRACKING-STN))
 \quad\quad \= (Reqd.) \\
\end{tabbing}

\subsubsection{Orbital elements}

Mean Keplerian orbital elements need to be stored for orbiting
antennas, in the following proposed format:

\begin{tabbing}
MEAN\_ORBIT\quad\quad \= Double(6)\quad\quad \= (Mean Keplerian elements)
 \quad\quad \= (Opt.) \\
\end{tabbing}

\subsubsection{Phased array identifier}

It is proposed that the PHASED\_ARRAY\_ID column be made optional.

\subsubsection{Row flag}

It is proposed that a flag column be added to validate or invalidate
the contents of the row, as follows:

\begin{tabbing}
FLAG\_ROW\quad\quad \= Bool\quad\quad \= (Row flag)\quad\quad \= (Reqd.)\\
\end{tabbing}

\subsection{ARRAY table}

It is proposed that multiple datasets be combined within one MS
through the use of the following indices in the MAIN table:

\begin{itemize}

\item{{\bf OBSERVATION\_ID:} This separates distinct observations
taken with different instruments (eg. MERLIN, VLBA, VLA), or observations
with the same instrument at different epochs.}

\item{{\bf PROCESSOR\_ID:} Information obtained via this index
separates data from separate processors, such as separate correlators,
as may occur for combined VLBI datasets (eg. MK4, VLBA).}

\item{{\bf ARRAY\_ID:} This index indentifies subarrays, such as those
defined by separate groups of antennas observing independently in
source or frequency, which cannot necessarily be calibrated
together. Subarrays can be transient and highly variable, as in
geodetic VLBI observations, or more directly associated with the clear
sub-division of an interferometer array (eg. Fixed Q-band and C-band
subarrays during a VLA observing run). Subarrays may also be assigned
by users in an arbitrary manner to label separate parts of an
array. This may bear no direct relation to the observing
schedule. This interpretation is supported here with the proviso that
antennas which cannot be calibrated together are not assigned to the
same subarray. The index ARRAY\_ID is not numbered relative to
OBSERVATION\_ID.}

\end{itemize}

In this context the ARRAY table can be absorbed in the OBSERVATION
table and it is proposed here that it be removed from the MS
definition.  A new, optional SCAN\_SUMMARY table is proposed to store
overall array, observation and scan summary information.

\subsection{FEED table}

It is proposed that the ARRAY\_ID column be removed from the FEED
table, for reasons discussed in Section 3.2.1.  The format of the BEAM
sub-table (indexed on BEAM\_ID) is deferred for later specification.
In addition, it is proposed that the PHASED\_FEED\_ID be made optional.

A new column is proposed to store the focus length, as:

\begin{tabbing}
FOCUS\_LENGTH\quad\quad \= Double\quad\quad \= 
(Focus length) \quad\quad \= (Opt.) \\ 
\end{tabbing}


\subsection{FIELD table}

Changes applicable to the FIELD table are discussed in this section.

\subsubsection{Direction information}

With the definition of a POINTING sub-table (see Section 4.0.1), the
POINTING\_DIR and POINTING\_DIR\_RATE columns can be removed from the
FIELD table. 

It is also proposed that the DELAY\_DIR\_RATE, PHASE\_DIR\_RATE and
REFERENCE\_DIR\_RATE columns be replaced by extending the DELAY\_DIR,
PHASE\_DIR and REFERENCE\_DIR columns to dimension (2, NUM\_POLY+1) to
represent higher order terms. This allows the representation of more
specialized correlation modes. The polynomial order is added as a new
column:

\begin{tabbing}
NUM\_POLY\quad\quad \= Int\quad\quad \= (Polynomial order)
 \quad\quad \= (Reqd.) \\
\end{tabbing}

\subsubsection{Ephemeris identification}

A separate identifier is proposed to identify the ephemeris data
associated with time-variable FIELD position data, as obtained for
moving sources. These data are stored in a separate EPHEMERIS
table. This has the added advantage of identifying multiple revisions
of the ephemeris, which may vary between observation and final
reduction. The proposed new column is:

\begin{tabbing} 
EPHEMERIS\_ID\quad\quad  \= Int\quad\quad\quad\quad \= 
(Ephemeris identifier)\quad\quad\quad\quad \= (Opt.) \\
\end{tabbing}

\subsubsection{Indexing}

It is further proposed that the FIELD\_ID column be removed and that
the table be directly indexed using the FIELD\_ID value in MAIN. A
row flag is proposed to invalidate the entry, as follows:

\begin{tabbing}
FLAG\_ROW\quad\quad  \= Bool\quad\quad \= (Row flag)\quad\quad \= (Reqd.)\\
\end{tabbing}

\subsection{FLAG\_CMD}

As discussed in Section 3.1.8, this table is intended to hold all
global flag commands applied to the MS. Their cumulative effect is
reflected in the MAIN column FLAG\_CATEGORY, in the category FLAG\_CMD,
and therefore further in FLAG and FLAG\_ROW. In this proposal, the
FLAG\_CMD records are represented as character strings, using the MS
column names in a limited syntax to represent flagging operations,
each with associated labeling information:

\begin{tabbing} 
FLAG\_COMMAND\quad\quad\quad\quad \= String\quad\quad \= (Flagging command)
\quad\quad\quad\quad \= (Reqd.) \kill
TIME \> Double \> (Midpoint of interval) \> (Reqd.) \\
INTERVAL \> Double \> (Time interval) \> (Reqd.) \\
TYPE \> String \> (FLAG or UNFLAG) \> (Reqd.) \\
REASON \> String \> (Reason) \> (Reqd.) \\
LEVEL  \> Int    \> (Flag level) \> (Reqd.) \\
SEVERITY \> Int \> (Flag severity) \> (Reqd.) \\
APPLIED \> Bool \> (Flag reflected in MAIN ?) \> (Reqd.) \\
COMMAND \> String \> (Flagging command) \> (Reqd.) \\
\end{tabbing}

\subsection{OBSERVATION table}

It is proposed that the CORR\_SCHEDULE field be removed (see Section
4.0.4), which is accessed via the PROCESSOR sub-table. The telescope
name, which was previously stored in the ARRAY sub-table, is proposed
to be recorded as:

\begin{tabbing} 
TELESCOPE\_NAME\quad\quad \= String\quad\quad \= (Telescope name (eg WSRT))
 \quad\quad \= (Reqd.) \\
\end{tabbing}

The observation date is required as:

\begin{tabbing} 
TIME\_RANGE\quad\quad \= Double(2)\quad\quad \= 
(Start and stop times of observation)
 \quad\quad \= (Reqd.) \\
\end{tabbing}

A new column is proposed to identify the type of the unmodified
schedule summary is SCHEDULE, namely:

\begin{tabbing} 
SCHEDULE\_TYPE\quad\quad \= String\quad\quad \= 
(Schedule type (eg. VEX)) \quad\quad \= (Reqd.) \\
\end{tabbing}

A field is proposed to store the log file recorded during observations, as:

\begin{tabbing}
LOG\quad\quad \= String(*)\quad\quad \= (Observing log)
 \quad\quad \= (Opt.) \\
\end{tabbing}

A field is proposed to store the target release date for the project:

\begin{tabbing}
RELEASE\_DATE\quad\quad \= Double\quad\quad \= (Release date)
 \quad\quad \= (Opt.) \\
\end{tabbing}

A row flag is proposed to invalidate the entry as:

\begin{tabbing}
FLAG\_ROW\quad\quad \= Bool\quad\quad \= (Row flag)\quad\quad \= (Reqd.)\\
\end{tabbing}

\subsection{OBS\_LOG table}

It is proposed that this table be renamed as HISTORY, and that its
primary function be to act as a processing log. New columns
are proposed as follows:

\begin{tabbing}
PRIORITY\quad\quad\quad \= String\quad\quad \= 
(Message priority (eg. DEBUGGING, WARN, NORMAL or SEVERE)
 \quad\quad \= (Reqd.) \\
ORIGIN             \> String  \> 
(Source code location from which message originates) \> (Reqd.) \\
OBJECT\_ID         \> String  \>
(Originating ObjectID, if available) \> (Reqd.) \\
APPLICATION        \> String \>
(Application name) \> (Reqd.) \\
\end{tabbing}

User notes will be identified by a specific ORIGIN string.

This is also a useful table to store a record of the Glish commands
and parameters used to invoke the application. This allows parts of
the data reduction to be repeated with minimal effort. An automatic
recording mechanism needs to be invoked by individual methods or
applications for this purpose. This information can be represented in
a string format as:

\begin{tabbing}
CLI\_COMMAND\quad\quad \= String(*)\quad\quad \= 
(Glish parameter or command sequence)
 \quad\quad \= (Reqd.) \\
APP\_PARAMS \> String(*) \> (Application parameters) \> (Reqd.)
\end{tabbing}

 It is recognized that a mechanism to log all ad hoc CLI commands
affecting a MeasurementSet is impractical, but this field should be
filled by all standard applications.

\subsection{SOURCE table}

The changes proposed for the SOURCE table are given below:

\subsubsection{Source model}

Information on the detailed source model needs to be specified at
several points in calibration and imaging, and is properly selected at
that time. A field is added here to record the assignment of a
component source model for each SOURCE\_ID, as this is made during
calibration. This assigned source model will never be used in an
application without explicit notification or confirmation, and is not
intended as a default source model.

\begin{tabbing}
SOURCE\_MODEL\quad\quad \= TableRecord\quad\quad \= 
(Assigned CSM) \quad\quad \= (Opt.) \\
\end{tabbing}

\subsubsection{Spectral line information}

Information on the systemic velocity and rest frequency of spectral
lines observed towards individual sources is required in various
stages of line calibration, particularly for Doppler tracking of
individual transitions in software. The indices SOURCE\_ID and
SPECTRAL\_WINDOW\_ID are a good composite key for this information and
it is proposed that they be added to the SOURCE table. A general
extension for multiple spectral lines per spectral window is proposed
as follows:

\begin{tabbing}
REST\_FREQUENCY\quad\quad \= Double(NUM\_LINES)\quad\quad \=
(Rest frequency for this transition)\quad\quad\quad\quad\quad \= (Reqd.) \kill
NUM\_LINES \> Int \> 
(No. of transitions for this spectral window) \> (Reqd.) \\
TRANSITION           \> String(NUM\_LINES)  \> 
(Transition name) \> (Reqd.) \\
REST\_FREQUENCY       \> Double(NUM\_LINES) \>
(Rest frequency for this transition) \> (Reqd.) \\
SYSVEL               \> Double(NUM\_LINES) \>
(Systemic velocity) \> (Reqd.) \\
\end{tabbing}

\subsubsection{Pulsar information}

 The PULSAR\_ID is moved from the MAIN table to the SOURCE table, as
proposed in Section 3.1.6. This points to the PULSAR table, containing
pulsar source information.

\subsection{SPECTRAL\_WINDOW table}

As discussed in Section 3.2.1, it is proposed that the ARRAY\_ID key be
removed.  In addition, the polarization labeling data CORR\_TYPE,
CORR\_PRODUCT and NUM\_COR are moved to the POLARIZATION sub-table, as
discussed in Section 3.1.3.

\subsubsection{Name}

A spectral window name, is proposed as:

\begin{tabbing}
NAME\quad\quad \= String\quad\quad \= 
(Spectral window name) \quad\quad\quad \= (Reqd.) \\
\end{tabbing}


\subsubsection{Frequency groups}

It is important to identify which SPECTRAL\_WINDOW\_ID's are
associated for calibration purposes as part of a broader frequency
group.  This is necessary in certain calibration applications,
including multi-band VLBI fringe-fitting. Two new columns are
added for this purpose:

\begin{tabbing}
FREQ\_GROUP\_NAME\quad\quad \= String\quad\quad \=
(Frequency group name)\quad\quad \= (Reqd.) \kill
FREQ\_GROUP \> Int \> (Frequency group) \> (Reqd.) \\
FREQ\_GROUP\_NAME \> String \> (Frequency group name) \> (Reqd.) \\
\end{tabbing}

\subsubsection{Doppler tracking}

The REST\_FREQUENCY column is replaced by a DOPPLER\_ID field
reflecting the Doppler tracking parameters used for the spectral
window. These data are stored in a separate DOPPLER table which
references the line parameters defined in Section 3.9.2.

\begin{tabbing}
DOPPLER\_ID\quad\quad \= Int\quad\quad \= 
(Doppler identifier) \quad\quad\quad\quad\quad \= (Opt.) \\
\end{tabbing}

\subsubsection{Sideband and baseband-converter information}

Two new spectral window attributes are added to record baseband
converter and sideband information:

\begin{tabbing}
NET\_SIDEBAND\quad\quad \= Int\quad\quad \= 
(Upper or lower) \quad\quad\quad\quad\quad \= (Reqd.) \\
BBC\_NO           \> Int  \> 
(Baseband converter no.) \> (Opt.) \\
BBC\_SIDEBAND     \> Int \>
(BBC sideband) \> (Opt.) \\
\end{tabbing}

\subsubsection{Receiver information}

An additional index is proposed to reference receiver information:

\begin{tabbing}
RECEIVER\_ID\quad\quad \= Int\quad\quad \= 
(Receiver identifier) \quad\quad\quad\quad\quad \= (Opt.) \\
\end{tabbing}

This points to an optional RECEIVER sub-table.

\subsubsection{Associated spectral windows}

Individual spectral windows may hold a unique relationship to others,
such as that between channel-zero data and the associated, unaveraged
spectrum. Other may have a subset or equivalence relationship, and it
is very useful to record this information. Additional columns for this
purpose are proposed as follows:

\begin{tabbing}
ASSOC\_SPW\_ID\quad\quad \= Int(*)\quad\quad \= 
(Associated spw\_id.) \quad\quad\quad\quad\quad \= (Opt.) \\
ASSOC\_NATURE           \> String(*)  \> 
(Nature of association) \> (Opt.) \\
\end{tabbing}

\subsubsection{Frequency information}

The nominal channel spacing, actual spectral resolution and effective
noise bandwidth are distinct, and it is proposed that they be
decoupled by adding new columns:

\begin{tabbing}
CHAN\_WIDTH\quad\quad \= Double(*)\quad\quad \= 
(Nominal channel width) \quad\quad\quad\quad\quad \= (Reqd.) \\
EFFECTIVE\_BW \> Double(*) \> (Effective bandwidth) \> (Reqd.)\\
\end{tabbing}

The RESOLUTION column then denotes the actual spectral resolution.  A
new column is proposed to allow a row-based Frequency Meaure
reference, as:

\begin{tabbing}
MEAS\_FREQ\_REF\quad\quad \= Int\quad\quad \= (Frequency ref.)\quad\quad 
 \= (Reqd.) \\
\end{tabbing}

\subsubsection{Row flag}

It is proposed that a flag column be added to validate or invalidate
the contents of the row, as follows:

\begin{tabbing}
FLAG\_ROW\quad\quad \= Bool\quad\quad \= (Row flag)\quad\quad \= (Reqd.)\\
\end{tabbing}

\subsection{SYSCAL table}

As discussed in Section 3.2.1 it is proposed that the ARRAY\_ID key be
removed.  Further specific changes to the SYSCAL table are listed
below:

\subsubsection{Number of receptors}

This information is redundant as it is already specified in the FEED
sub-table, and it is proposed that NUM\_RECEPTORS be removed.

\subsubsection{Measured temperatures}

Optional antenna temperature fields are proposed as follows, where
\nr=NUM\_RECEPTORS and \nf=NUM\_CHAN:

\begin{tabbing}
TANT\_TSYS\_SPECTRUM\quad\quad \= Float(\nr,\nf)\quad\quad \=
(${{T_{ant}}\over{T_{sys}}}$ spectrum)
\quad\quad\quad\quad\quad\quad\quad\quad\quad 
\= (Opt.) \kill
TANT \> Float(\nr) \> (Antenna temperature $T_{ant}$) \> (Opt.) \\
TSKY \> Float(\nr) \> (Sky temperature) \> (Opt.) \\
TANT\_SPECTRUM   \> Float(\nr,\nf)  \> 
(Antenna temperature spectrum) \> (Opt.) \\
TSKY\_SPECTRUM \> Float(\nr,\nf) \> (Sky temp. spectrum) \> (Opt.) \\
TANT\_TSYS     \> Float(\nr) \>
(${{T_{ant}}\over{T_{sys}}}$) \> (Opt.) \\
TANT\_TSYS\_SPECTRUM     \> Float(\nr,\nf) \>
(${{T_{ant}}\over{T_{sys}}}$ spectrum) \> (Opt.) \\
TANT\_FLAG     \> Bool \>
(Flag for TANT) \> (Opt.) \\
TSKY\_FLAG \> Bool \> (Flag for TSKY) \> (Opt.) \\
TANT\_TSYS\_FLAG     \> Bool \>
(Flag for TANT\_TSYS) \> (Opt.) \\
\end{tabbing}

 Calibration measurements in bands other than those for which there
are recorded data, are accomodated by creating additional spectral
window identifiers in the SPECTRAL\_WINDOW table, and recording the
data in the same format in the SYSCAL table. This allows simultaneous
wide- and narrow-band calibration data to be recorded in a uniform manner.

 Existing calibration quantities in the SYCAL table, including
PHASE\_DIFF, TCAL, TRX and TSYS, and their associated flags, are made
optional, on an equal footing with the other entries.


\subsubsection{Antenna pointing}

There is a key requirement to know where the antenna and feed are
pointed at the time of the calibration measurement. In synthesis,
these data may be taken while interferometer data are not being
recorded for example, but are nonetheless useful in a priori amplitude
calibration. This information is obtained by lookup in the POINTING
sub-table (see Section 4.0.1).

\subsection{WEATHER table}

As in previous sections it is proposed that the ARRAY\_ID qualifier
for ANTENNA\_ID be removed (see Section 3.2.1). Time- and
direction-dependent ionospheric electron column density data (such as
that derived from GPS receiver data) is proposed to be stored in a
separate TEC sub-table. The existing data columns are made optional,
and associated flags are added:

\begin{tabbing}
IONOS\_ELECTRON\_FLAG\quad\quad \= Bool\quad\quad \= 
(Flag for IONOS\_ELECTRON) \quad\quad \= (Opt.) \\
H2O\_FLAG \> Bool \> (Flag for H2O) \> (Opt.) \\
IONOS\_ELECTRON\_FLAG \> Bool \> (Flag for IONOS\_ELECTRON) \> (Opt.) \\
PRESSURE\_FLAG \> Bool \> (Flag for PRESSURE) \> (Opt.) \\
REL\_HUMIDITY\_FLAG \> Bool \> (Flag for REL\_HUMIDITY) \> (Opt.) \\
TEMPERATURE\_FLAG \> Bool \> (Flag for TEMPERATURE) \> (Opt.) \\
WIND\_DIRECTION\_FLAG \> Bool \> (Flag for WIND\_DIRECTION) \> (Opt.) \\
WIND\_SPEED\_FLAG \> Bool \> (Flag for WIND\_SPEED) \> (Opt.) \\
\end{tabbing}

A new column set is proposed to record the dew point:

\begin{tabbing}
DEW\_POINT\_FLAG\quad\quad   \= Bool\quad\quad  \= 
(Flag for DEW\_POINT)\quad\quad \= (Opt.) \kill
DEW\_POINT \> Float \>  (Dew point temperature) \> (Opt.) \\
DEW\_POINT\_FLAG   \> Bool  \> (Flag for DEW\_POINT) \> (Opt.) \\
\end{tabbing}


\section{New sub-tables}

New sub-tables added to MS v2.0 are included in this Section.

\subsubsection{Antenna-based pointing (POINTING)}

The current MS definition associates a pointing direction with the
FIELD\_ID. In general, however, the pointing direction is
antenna-based and may be time-variable and distinct from the
correlated field position. This is best modeled by introducing a
POINTING sub-table indexed from the MAIN table by the key (TIME,
ANTENNA), and containing absolute pointing information.

The format for the POINTING table is given later but includes support
for tracking arcs on the sky, and storing a priori pointing offset
information. The POINTING table indicates the direction the optical
axis of the telescope is pointing; feed offsets are stored in the FEED
table.

\subsubsection{State information (STATE)}

The STATE table is used to store information concerning reference
signals or loads, sub-scan information and a description of the
current observing mode, as these fields constitute a set of related
information. Sub-scans are used here to denote component sequences of
standard observing patterns. Observing modes are envisaged to be
represented by standard reserved keywords or phrases. This provides a
means of linking the observing strategy and the data; this is a
central requirement in guiding automated data reduction procedures,
and also provides greater user information.

\subsubsection{Digital samplers (SAMPLER)}

Data streams which have in common an antenna, feed and receiver/LO
chain may be independently associated with digital samplers with
time-variable quantization schemes (eg. 1- versus 2-bit sampling) and
sampler voltage levels. For calibration purposes this does not
constitute a separate feed but does require separate digital
correction. This is certainly important in VLBI. A separate SAMPLER
sub-table is proposed to address this, indexed via MAIN
table columns (TIME, ANTENNA, FEED, SPECTRAL\_WINDOW\_ID).

\subsubsection{Processor information (PROCESSOR)}

The back-end processor is identified through the PROCESSOR sub-table,
which contains information concerning the processor type (see Section
3.1.4), and indices for time-dependent and time-independent mode
information. A pass number is used to distinguish multiple passes for
back-end processors for which this is possible (e.g. VLBI
correlators). Each back-end or processor type includes a family of
associated tables. These are formatted using common column names where
appropriate, and accessed using classes that encapsulate the different
information unique to each instrument-specific data behind a common
interface. This family of tables will be specified in the future for
individual instruments and processor types. It is noted, however, that
the optional PHASE\_ID index in MAIN indexes directly into
processor-specific phase information included in this family of
tables. An illustration for generic interferomter correlators is given
below:

The correlation parameters associated with an individual uv-record are
particularly important in VLBI calibration but have more general
application in broader synthesis development. This correlator
information has four separate components, which are indexed from the
PROCESSOR sub-table (which has primary key PROCESSOR\_ID):

\begin{itemize}
\item{{\bf Time-invariant correlation parameters}

This includes software revision numbers and geometrical constants
adopted by the correlator model. We use a TYPE\_ID index to point to
this information, which is stored in a separate $subtype\_$CORRELATOR
table.  The correlator sub-type (e.g. MK4, JIVE) defines the prefix of
the correlator model, geometry and mode tables discussed below.  }

\item{{\bf Time-dependent correlator parameters}

For synthesis data, these include correlator model values and model
polynomials as a function of antenna or baseline, time, spectral
window, polarization and correlator identifier. Note that there is no
one-to-one mapping between a given uv-record and a correlator model
record, as a model update may occur within an accumulation
interval, and an indexed lookup is always required to locate the
associated correlator model record. The time-dependent data also
include geometric information common to the instrument as a whole such as
polar motion, time information and the solar system ephemerides. These
data are indexed on time and correlator identifier. We propose 
correlator model and correlator geometry tables to hold these two
types of time-dependent data respectively. Each table has a composite
key including TYPE\_ID and no additional MAIN table columns are
required.

Synthesis correlators differ significantly in their architecture and
reference frames. For example, the VLBI MK3 correlator is a
baseline-based, lag correlator (XF), while the VLBA correlator is an
antenna-based FX correlator. Both also use different celestial and
interferometer reference frames. This poses difficulties for the
representation of correlator mode, geometry and model information in a
common format. Translation between different frames may not be exact
or reversible in this case, and reference to the original correlator
model data is sometimes required. The primary purpose of correlator
record-keeping is to provide parameters for visibility corrections
required by correlator digital signal processing effects, and to
provide total model information for geodetic or astrometric
applications.

To accommodate these correlator differences, and other significant
differences between synthesis and single-dish correlators, it is
proposed that correlator sub-type (eg. MK4, GBT, VLBA; as given in the
PROCESSOR sub-table) act as a global key for correlator information,
which resides in separate tables (eg. MK3\_CORRELATOR\_MODEL,
MK4\_CORRELATOR\_GEOMETRY), but is accessed by uniform MS methods to return
common global data such as total model or correlator correction
information. The format of these sub-tables needs to form part of a
global MS definition, with uniform column names used wherever
possible. The same subset of MAIN indices are used to access the
correlator sub-tables, in conjunction with TYPE\_ID. The alternative
representation of diverse correlator information in a single, common
table will lead to a proliferation of optional columns, which will
obscure the underlying physical correlator model in most cases.}


\item{{\bf Correlator mode information}

Within a given correlator run, separate subarrays may be correlated in
different modes which are distinguished by different setup parameters
(eg. FFT size, number of lags, spectral averaging) but which share the
same global correlator model. This can be varied independently of
TYPE\_ID and a secondary index MODE\_ID is
proposed for this purpose. Separate correlator mode sub-tables are
required to represent this information (eg. MK3\_CORRELATOR\_MODE), as
discussed in the previous item.}

\item{{\bf Correlator pass information}

Redundant data are often produced by multiple VLBI correlator
passes. Traditionally, no record is kept of the correlation pass for
each uv-record and redundant data may be rejected by other criteria,
such as comparing weights on a record-by-record basis. Retaining
correlator pass information adds flexibility to the rejection of
redundant data, allows more versatile data selection and simplifies
the filling of VLBI data. A separate table is proposed to store
correlator pass information (including correlator job number), which
is indexed on the pass number and time. The correlator pass
information is numbered relative to each correlator job identifier.}

\end{itemize}

\subsubsection{Frequency offsets (FREQ\_OFFSET)}

Time-variable frequency tracking at each antenna is properly modeled
by incorporating an antenna-based frequency offset for each antenna in
each uv-record. This retains the same SPECTRAL\_WINDOW\_ID for
calibration purposes, if appropriate, but accurately reflects
frequency changes throughout the observations. These data are indexed
by the MAIN table columns (TIME, ANTENNA$n$, FEED,
SPECTRAL\_WINDOW\_ID).

\subsubsection{Orbiting VLBI (TRACKING\_STN)}

In orbiting VLBI observations, each antenna can be thought of as
consisting of both an orbiting component and a ground-based tracking
station component, with different calibration errors associated with
each. Both require entries in the ANTENNA table and can in principle
vary independently. A new TRACKING\_STN table is proposed to fully
support calibration and data selection in this case, indexed by MAIN
columns (TIME, ANTENNA).

\subsubsection{DATA labeling (DATA\_DESCRIPTION)}

A new table is added to hold the data description information, as
discussed in Section 3.1.3. As described in the same Section,
polarization information is stored in a POLARIZATION sub-table.

\subsubsection{Doppler tracking (DOPPLER)}
A new doppler tracking table is added to store source-based doppler
tracking information for each SPECTRAL\_WINDOW, as discussed in
Section 3.10.2.

In spectral-line observing, both the on-line telescope and back-end
processor control systems and the off-line data reduction package
require information specifying the Doppler tracking parameters used
during observing or required for post-processing. There are many
different possibilities for how Doppler tracking is implemented in
practice; these depend on the instrument and science goals of the
observing. The MSv2 data format attempts to represent this information
in a sufficiently general format that is not instrument-dependent, and
covers all Doppler tracking strategies.

In general, Doppler tracking parameters can be sub-divided into those
that are dependent on the astronomical source or field, and those that
capture the frequency tracking strategy used at the telescope and/or
back-end processor. The source-based parameters include:

\begin{description}

\item{(a) Celestial coordinates of the source or field position used
for Doppler tracking.}

\item{(b) Properties of the associated transitions $(1..N_{lines})$.
   \begin{description}
      \item{(i) Transition name (e.g. CO J=2-1, HI etc.).}
      \item{(ii) Systemic velocity $(V_{sys}).$}
      \item{(iii) Rest frequency $(\nu_0).$}
   \end{description}}

\item{(c) Options for frequency tracking during observing include:
   \begin{description}
      \item{(i) No tracking on-line; shift spectra in post-processing.}
      \item{(ii) Coarse tracking on-line; refine shift of spectra in post-processing.}
      \item{(iii) Fine tracking on-line; no refinement in post-processing.}
      \item{(iv) Shift spectra in post-processing to a new line or new frame, not tracked during observing, but lying in the observed frequency band.}
   \end{description}}
\end{description}

In MSv2, the source properties (a,b) are recorded in the SOURCE
sub-table, which is accordingly not optional if Doppler tracking is
employed. The transition information is SPECTRAL\_WINDOW\_ID dependent, as
reflected in the SOURCE sub-table key. The SPECTRAL\_WINDOW\_ID is part of the
primary key in the MAIN table, via DATA\_DESC\_ID, and can accordingly
be used in any sub-table key without introducing a circular
dependency. If only the field center is tracked, then a SOURCE entry
for the field center should be added, named accordingly. The
transition list in the SOURCE table should include only those
transitions of scientific interest to the observer, not all possible
transitions in the frequency band. With increasing observing
bandwidths, multiple transition entries will become more common
however.

The frequency tracking strategies (c) are recorded in the
SPECTRAL\_WINDOW, FREQ\_OFFSET and DOPPLER sub-tables. An underlying
principle is that spectral windows which can be calibrated together
should share a common SPECTRAL\_WINDOW\_ID wherever possible. For
minimal or no on-line tracking (c-i,ii), the frequency in the
SPECTRAL\_WINDOW table should be labelled as topocentric, and will be
fixed. Option (c-ii) is accommodated by adding supplementary offset in
the FREQ\_OFFSET sub-table as required. This table can also be used to
accomodate antenna-based and other frequency offsets in a flexible
manner. Both (c-i,ii) have DOPPLER\_ID = -1.

For (c-iii), the frequency axis can be labelling in a frame in which
it is constant, to within meaningful scientific precision. In this
case, the DOPPLER\_ID index points to the DOPPLER entry describing the
transition information used in the on-line tracking. This can be
source-dependent, as the DOPPLER key includes a SOURCE\_ID. For
Doppler tracking in software (c-iv), the DOPPLER\_ID key should be
adjusted to reflect the resultant frequency shift, in the same way
as described for on-line Doppler tracking.

\subsubsection{Scan summary information}

An optional new table is proposed to store summary information for
each observation, array and scan, as outlined in Section 3.3. This
table may need to be re-generated if data are flagged or removed,
however. It will provide useful user information, and may in addition
be used by applications if current.

\subsubsection{Associated tables (CAL\_TABLES and SORTED\_TABLES)}

Links to associated sorted reference MeasurementSets, and calibration
tables are provided in these two MS sub-tables.

\section{MS v3.0 layout}

There is a MAIN table containing a number of data columns and keys
into various subtables. There is at most one of each subtable. The
subtables are stored as keywords of the MS, and all defined sub-tables
are tabulated below.  Optional sub-tables are shown in italics.

\begin{tabular}{|l|p{1.5in}|p{1.6in}|} \hline
\multicolumn{3}{|c|}{\bf Subtables}\\ \cline{1-3}
\multicolumn{1}{|c|}{Table}&\multicolumn{1}{|c|}{Contents}&\multicolumn{1}{|c|}{Keys}\\ \cline{1-3}
ANTENNA& Antenna characteristics & ANTENNA\_ID \\
DATA\_DESCRIPTION & Data description & DATA\_DESC\_ID\\
({\it DOPPLER}) & Doppler tracking & DOPPLER\_ID, SOURCE\_ID \\
FEED  &          Feed characteristics & FEED\_ID, ANTENNA\_ID, TIME, SPECTRAL\_WINDOW\_ID \\
FIELD  &         Field position & FIELD\_ID  \\
FLAG\_CMD  &  Flag commands & TIME \\
({\it FREQ\_OFFSET}) & Frequency offset information & FEED\_ID, ANTENNA$n$, FEED\_ID, TIME, SPECTRAL\_WINDOW\_ID \\
HISTORY  & History information & OBSERVATION\_ID, TIME \\
OBSERVATION  &Observer, Schedule, etc & OBSERVATION\_ID \\ 
POINTING & Pointing information & ANTENNA\_ID, TIME \\
POLARIZATION &Polarization setup & POLARIZATION\_ID \\
PROCESSOR & Processor information & PROCESSOR\_ID  \\
({\it SOURCE})  &        Source information & SOURCE\_ID, SPECTRAL\_WINDOW\_ID, TIME \\
SPECTRAL\_WINDOW  &     Spectral window setups & SPECTRAL\_WINDOW\_ID \\
STATE   & State information & STATE\_ID \\
({\it SYSCAL}) &         System calibration characteristics & FEED\_ID, ANTENNA\_ID, TIME, SPECTRAL\_WINDOW\_ID \\
({\it WEATHER})  &       Weather info for each antenna & ANTENNA\_ID, TIME \\
\hline
\end{tabular}

\par
Note that there are two types of subtables. For the first, simpler
type, the key (ID) is the row number in the subtable. 
Examples are FIELD, SPECTRAL\_WINDOW, OBSERVATION
and PROCESSOR. For the second, the key is a collection of parameters,
usually including TIME. Examples are FEED, SOURCE, SYSCAL, WEATHER.

Note that all optional columns are indicated in italics and in parentheses.

\definetable{MAIN table: Data, coordinates and flags}
{
\defline{\bf Columns}
\defline{\em Keywords}
MS\_VERSION   & Float  & & & MS format version \\
{\it (SORT\_COLUMNS)} & String & & & Sort columns \\  
{\it (SORT\_ORDER)}   & String & & & Sort order \\
\defline{\em Key}
TIME  &         Double  &      s   &     EPOCH    & Integration midpoint \\
{\it (TIME\_EXTRA\_PREC)} & Double & s       &     & extraTIME precision \\
ANTENNA1  &      Int    &          &              & First antenna \\
ANTENNA2  &      Int    &          &              & Second antenna \\
{\it (ANTENNA3)} & Int    &          &              & Third antenna \\
FEED1  &      Int       &          &              & Feed on ANTENNA1 \\
FEED2  &      Int       &          &              & Feed on ANTENNA2 \\
{\it (FEED3)} & Int       &          &              & Feed on ANTENNA3 \\
DATA\_DESC\_ID  &  Int   &          &        & Data desc. id.\\
PROCESSOR\_ID  & Int   &          &        & Processor id.\\
{\it (PHASE\_ID)} & Int &               &         & Phase id.\\
FIELD\_ID  &      Int   &          &        & Field id.\\
\defline{\em Non-key attributes}
INTERVAL  &      Double &       s  &      & Sampling interval \\
EXPOSURE  &      Double &       s  &      & The effective integration time \\
TIME\_CENTROID & Double &       s  &    EPOCH & Time centroid \\
{\it (PULSAR\_BIN)} & Int &          &      & Pulsar bin number \\
{\it (PULSAR\_GATE\_ID)} & Int &         &      & Pulsar gate id. \\   
SCAN\_NUMBER  &   Int   &          &      & Scan number  \\ 
ARRAY\_ID  &      Int   &          &      & Subarray number\\
OBSERVATION\_ID  & Int   &          &     & Observation id.\\
STATE\_ID & Int   &          &              & State id.\\
{\it (BASELINE\_REF)} &  Bool  &     &      & Reference antenna \\
UVW       &   Double(3) &      m   & UVW  & UVW coordinates \\
{\it (UVW2)} &  Double(3) &      m   & UVW  & UVW (baseline 2) \\
}


\definetable{MAIN table: continued}
{
\defline{\em Data}
{\it (DATA)}    &    Complex(\nc, \nf) &  & & Complex visibility
matrix (synthesis arrays)\\
{\it (FLOAT\_DATA)} & Float(\nc, \nf) & & & Float data matrix
(single dish) \\
{\it (VIDEO\_POINT)} & Complex(\nc) & & & Video point \\
{\it (LAG\_DATA)} & Complex(\nc, \nl) & & & Correlation function \\
SIGMA   &    Float(\nc) &  & & Estimated rms noise for single channel \\ 
{\it (SIGMA\_SPECTRUM)}   &    Float(\nc, \nfs) &  & & Estimated rms noise \\ 
WEIGHT   &    Float(\nc) &  & & Weight for whole data matrix \\ 
{\it (WEIGHT\_SPECTRUM)}   &    Float(\nc, \nfs) &  & & Weight for each channel\\ 
\defline{\em Flag information}
FLAG    &    Bool(\nc, \nfs) &  &   &  Cumulative data flags \\ 
FLAG\_CATEGORY & Bool(\nc, \nfs, \ncat) & & & Flag categories \\
FLAG\_ROW &    Bool  &  &   &  The row flag \\ 
}
\begin{description}
\item[Notes:]

\item Note that \nl = number of lags, \nc = number of correlators, 
\nf = number of frequency channels, and \ncat = number of flag categories.

\item[MS\_VERSION] The MeasurementSet format revision number,
expressed as $major\_revision.minor\_revision$. This version is 2.0.

\item[SORT\_COLUMNS] Sort indices, in the form "$index_1,\ index_2\
...$, for the underlying MS. A string containing "NONE" reflects no
sort order.  An example might be SORT\_COLUMNS="TIME ANTENNA1
ANTENNA2", to indicate sorting in in time-baseline order.

\item[SORT\_ORDER] Sort order as either "ASCENDING" or "DESCENDING".

\item[TIME] Mid-point (not centroid) of data interval.

\item[TIME\_EXTRA\_PREC] Extra time precision.

\item[ANTENNA$n$] Antenna number ($\geq 0$), and a direct index into
the ANTENNA sub-table $rownr$. For $n > 2$, triple-product data
are implied.

\item[FEED$n$] Feed number ($\geq 0$). For $n > 2$, triple-product
data are implied.

\item[DATA\_DESC\_ID] Data description identifier ($\geq 0$), and
a direct index into the DATA\_DESCRIPTION sub-table $rownr$.

\item[PROCESSOR\_ID] Processor indentifier ($\geq 0$), and a direct
index into the PROCESSOR sub-table $rownr$.

\item[PHASE\_ID] Switching phase identifier ($\geq 0$)

\item[FIELD\_ID] Field identifier ($\geq 0$).

\item[INTERVAL] Data sampling interval. This is the nominal data
interval and does not include the effects of bad data or partial
integration.

\item[EXPOSURE] Effective data interval, including bad data and
partial averaging.

\item[PULSAR\_BIN] Pulsar bin number for the data record. Pulsar data
may be measured for a limited number of pulse phase bins. The pulse
phase bins are described in the PULSAR sub-table and indexed by this
bin number.

\item[PULSAR\_GATE\_ID] Pulsar gate identifier ($\geq 0)$, and a direct
index into the PULSAR\_GATE sub-table $rownr$. 

\item[SCAN\_NUMBER] Arbitrary scan number to identify data taken in
the same logical scan. Not required to be unique.

\item[ARRAY\_ID] Subarray identifier $(\geq 0)$, which identifies data
in separate subarrays, as defined in Section 3.3.

\item[OBSERVATION\_ID] Observation identifier $(\geq 0)$, which identifies
data from separate observations, as defined in Section 3.3.

\item[STATE\_ID] State identifier $(\geq 0)$, as defined in Section 3.1.5.

\item[BASELINE\_REF] Flag to indicate the original correlator
reference antenna for baseline-based correlators (True for ANTENNA1;
False for ANTENNA2).

\item[UVW] $uvw$ coordinates for the baseline from ANTENNE2 to ANTENNA1,
   i.e. the
   baseline is equal to the difference POSITION2 - POSITION1. The UVW given
   are for the TIME\_CENTROID, and correspond in general to the reference type
   for the PHASE\_DIR of the relevant field. I.e. J2000 if the phase reference
   direction is given in J2000 coordinates. However, any known reference is
   valid.
   Note that the choice of baseline direction and UVW definition ($W$ towards
   source direction; $V$ in plane through source and system's pole; $U$ in
   direction of increasing longitude coordinate) also determines the sign of
   the phase of the recorded data.

\item[UVW2] $uvw$ coordinates for the baseline from ANTENNE3 to ANTENNA1
   (triple-product data only), i.e. the
   baseline is equal to the difference POSITION3 - POSITION1. The UVW given
   are for the TIME\_CENTROID, and correspond in general to the reference type
   for the PHASE\_DIR of the relevant field. I.e. J2000 if the phase reference
   direction is given in J2000 coordinates. However, any known reference is
   valid.
   Note that the choice of baseline direction and UVW definition ($W$ towards
   source direction; $V$ in plane through source and system's pole; $U$ in
   direction of increasing longitude coordinate) also determines the sign of
   the phase of the recorded data.

\item[DATA, FLOAT\_DATA, LAG\_DATA] At least one of these columns
should be present in a given MeasurementSet. In special cases one or
more could be present (e.g., single dish data used in synthesis
imaging or a mix of auto and crosscorrelations on a multi-feed single
dish). If only correlation functions are stored in the MS, then \nfs\
is the maximum number of lags (\nl) specified in the LAG table for
this LAG\_ID.  If both correlation functions and frequency spectra are
stored in the same MS, then \nfs\ is the number of frequency channels,
and the weight information refers to the frequency spectra only. The
units for these columns (eg. 'Jy') specify whether the data are in
flux density units or correlation coefficients.

\item[VIDEO\_POINT] The video point for the spectrum, to allow the
full reverse transform.

\item[SIGMA] The estimated rms noise for a single channel, for
each correlator.

\item[SIGMA\_SPECTRUM] The estimated rms noise for each channel.

\item[WEIGHT] The weight for the whole data matrix for each
correlator, as assigned by the correlator or processor.

\item[WEIGHT\_SPECTRUM] The weight for each channel in the data
matrix, as assigned by the correlator or processor. The weight
spectrum should be used in preference to the WEIGHT, when available.

\item[FLAG] An array of Boolean values with the same shape as DATA
(see the DATA item above) representing the cumulative flags applying
to this data matrix, as specified in FLAG\_CATEGORY. Data are flagged
bad if the FLAG array element is True.

\item[FLAG\_CATEGORY] An array of flag matrices with the same shape as
DATA, but indexed by category. The category identifiers are specified
by a keyword CATEGORY, containing an array of string identifiers,
attached to the FLAG\_CATEGORY column and thus shared by all rows in
the MeasurementSet. The cumulative effect of these flags is reflected
in column FLAG. Data are flagged bad if the FLAG array element is
True. See Section 3.1.8 for further details.

\item[FLAG\_ROW] True if the entire row is flagged.

\end{description}

\definetable{ANTENNA: Antenna characteristics}
{
\defline{\bf Columns}
\defline{\em Data}
NAME &       String &  & &  Antenna name \\
STATION &        String & & &  Station name \\
TYPE  &  String &     &      & Antenna type \\
MOUNT  & String &  & &   Mount type:alt-az, equatorial, X-Y, orbiting, bizarre \\
POSITION &   Double(3) &  m   &     POSITION &   Antenna X,Y,Z phase reference positions \\
OFFSET & Double(3)  &   m &     POSITION & Axes offset of mount to FEED REFERENCE point \\
DISH\_DIAMETER &  Double &  m  &      & Diameter of dish\\
{\it (ORBIT\_ID)}    &    Int & & &  Orbit id. \\
{\it (MEAN\_ORBIT)}  &    Double(6) & & & Mean Keplerian elements\\
{\it (PHASED\_ARRAY\_ID)} & Int   &  & &   Phased array id.\\
\defline{Flag information}
FLAG\_ROW & Bool & & & Row flag\\
}
\begin{description}

\item[Notes:] This sub-table contains the global antenna properties
for each antenna in the MS. It is indexed directly from MAIN via
ANTENNA$n$.

\item[NAME] Antenna name (e.g. "NRAO\_140")

\item[STATION] Station name (e.g. "GREENBANK")

\item[TYPE] Antenna type. Reserved keywords include:
("GROUND-BASED" - conventional antennas; "SPACE-BASED" - orbiting
antennas; "TRACKING-STN" - tracking stations).

\item[MOUNT] Mount type of the antenna. Reserved keywords include:
("EQUATORIAL" - equatorial mount; "ALT-AZ" - azimuth-elevation mount; "X-Y"
- x-y mount; "SPACE-HALCA" - specific orientation model.)

\item[POSITION] In a right-handed frame, X towards the intersection of
the equator and the Greenwich meridian, Z towards the pole. The exact
frame should be specified in the MEASURE\_REFERENCE keyword (ITRF or
WGS84). The reference point is the point on the az or ha axis closest
to the el or dec axis.

\item[OFFSET] Axes offset of mount to feed reference point.

\item[DISH\_DIAMETER] Nominal diameter of dish, as opposed to the
effective diameter.

\item[ORBIT\_ID] Orbit identifier. Index used in ORBIT sub-table if
ANTENNA\_TYPE is "SPACE\_BASED".

\item[MEAN\_ORBIT] Mean Keplerian orbital elements, using the
standard convention (Flatters 1998): 
\begin{itemize}
   \item{{\bf 0:} Semi-major axis of orbit ($a$) in $m$.}
   \item{{\bf 1:} Ellipticity of orbit ($e$).}
   \item{{\bf 2:} Inclination of orbit to the celestial equator ($i$) 
                  in $deg.$}
   \item{{\bf 3:} Right ascension of the ascending node ($\Omega$) in $deg.$}
   \item{{\bf 4:} Argument of perigee ($\omega$) in $deg.$}
   \item{{\bf 5:} Mean anomaly ($M$) in $deg.$}
\end{itemize}

\item[PHASED\_ARRAY\_ID] Phased array identifier. Points to a
PHASED\_ARRAY sub-table which points back to multiple entries in the
ANTENNA sub-table and contains information on how they are combined.

\item[FLAG\_ROW] Boolean flag to indicate the validity of this
entry. Set to True for an invalid row. This does not imply any
flagging of the data in MAIN, but is necessary as the ANTENNA index in
MAIN points directly into the ANTENNA sub-table. Thus FLAG\_ROW can be
used to delete an antenna entry without re-ordering the ANTENNA
indices throughout the MS.

\end{description}

\definetable{DATA\_DESCRIPTION: Data description table}{
\defline{\bf Columns}
\defline{\em Data}
SPECTRAL\_WINDOW\_ID & Int & & & Spectral window id.\\
POLARIZATION\_ID & Int & & & Polarization id.\\
{\it (LAG\_ID)} & Int & & & Lag fn. id.\\
\defline{\em Flags}
FLAG\_ROW & Bool & & & Row flag.\\
}
\begin{description}

\item[Notes:] This table define the shape of the associated DATA array
in MAIN, and in indexed directly by DATA\_DESC\_ID.

\item[SPECTRAL\_WINDOW\_ID] Spectral window identifier.

\item[POLARIZATION\_ID] Polarization identifier ($\geq 0$); direct
index into the POLARIZATION sub-table.

\item[LAG\_ID] Lag function identifier ($\geq 0$), and a direct index
into the LAG sub-table $rownr$.

\item[FLAG\_ROW] True if the row does not contain valid data; does not
imply flagging in MAIN.

\end{description}


\definetable{DOPPLER: Doppler tracking information}{
\defline{\bf Columns}
\defline{\em Key}
DOPPLER\_ID  & Int & & & Doppler tracking id.\\
SOURCE\_ID  & Int & & & Source id.\\
\defline{\em Data}
TRANSITION\_ID & Int & & & Transition id.\\
VELDEF & Double & m/s & Doppler & Velocity definition of Doppler shift.\\
}
\begin{description}

\item[Notes:] This sub-table contains frame information 
for different Doppler tracking modes. It is indexed from the
SPECTRAL\_WINDOW\_ID sub-table (with SOURCE\_ID as a secondary index)
and thus allows the specification of a source-dependent Doppler
tracking reference for each SPECTRAL\_WINDOW. This model allows
multiple possible transitions per source per spectral window, but only
one reference at any given time.

\item[DOPPLER\_ID] Doppler identifier, as used in the SPECTRAL\_WINDOW
sub-table.

\item[SOURCE\_ID] Source identifier (as used in the SOURCE sub-table).

\item[TRANSITION\_ID] This index selects the appropriate line from the
list of transitions stored for each SOURCE\_ID in the SOURCE table.

\item[VELDEF] Velocity definition of the Doppler shift, e.g., RADIO or
OPTICAL velocity in m/s.

\end{description}



\definetable{FEED: Feed characteristics}{
\defline{\bf Columns}
\defline{\em Key}
ANTENNA\_ID &     Int &    &      & Antenna id\\
FEED\_ID &        Int &    &      & Feed id\\
SPECTRAL\_WINDOW\_ID &   Int &    &      & Spectral window id.\\
TIME &   Double  &      s  & EPOCH & Interval midpoint \\
INTERVAL &       Double  &      s     &       & Time interval\\
\defline{\em Data description}
NUM\_RECEPTORS &  Int &    &      & \# receptors on this feed \\
\defline{\em Data}
BEAM\_ID &        Int    &    &      & Beam model\\
BEAM\_OFFSET &    Double(2, NUM\_RECEPTORS)    &     rad & DIRECTION & Beam position offset (on sky but in antenna reference frame).\\
{\it (FOCUS\_LENGTH)} &   Double &    m       &       & Focus length \\
{\it (PHASED\_FEED\_ID)} & Int   &    &      & Phased feed\\
POLARIZATION\_TYPE &       String (NUM\_RECEPTORS) &    &      & Type of polarization to which a given RECEPTOR responds.\\
POL\_RESPONSE &  Complex (NUM\_RECEPTORS, NUM\_RECEPTORS) &    &      & Feed polzn. response\\
POSITION &    Double(3)  & m  &         POSITION & Position of feed relative to feed reference position for this antenna\\
RECEPTOR\_ANGLE &    Double (NUM\_RECEPTORS) &    rad &    & The reference angle for polarization.\\
}
\begin{description}

\item[Notes:] A feed is a collecting element on an antenna, such as a
single horn, that shares joint physical properties and makes sense to
calibrate as a single entity. It is an abstraction of a generic
antenna feed and is considered to have one or more RECEPTORs that
respond to different polarization states. A FEED may have a
time-variable beam and polarization response. 
Feeds are numbered from 0 on each separate antenna for each
SPECTRAL\_WINDOW\_ID. Consequently, FEED\_ID should be non-zero only
in the case of feed arrays, i.e. multiple, simultaneous beams on the
sky at the same frequency and polarization.

\item[ANTENNA\_ID] Antenna number, as indexed from ANTENNA$n$ in
MAIN.

\item[FEED\_ID] Feed identifier, as indexed from FEED$n$ in MAIN.

\item[SPECTRAL\_WINDOW\_ID] Spectral window identifier. A value of -1
indicates the row is valid for all spectral windows.

\item[TIME] Mid-point of time interval for which the feed parameters
in this row are valid. The same Measure reference used for the TIME
column in MAIN must be used.

\item[INTERVAL] Time interval. 

\item[NUM\_RECEPTORS] Number of receptors on this feed. See
POLARIZATION\_TYPE for further information.

\item[BEAM\_ID] Beam identifier. Points to an optional BEAM sub-table
defining the primary beam and polarization response for this FEED. A
value of -1 indicates that no associated beam response is defined.

\item[BEAM\_OFFSET] Beam position offset, as defined on the sky but in
the antenna reference frame.

\item[FOCUS\_LENGTH] Focus length. As defined along the optical axis
of the antenna.

\item[PHASED\_FEED\_ID] Phased feed identifier. Points to a
PHASED\_FEED sub-table which in turn points back to multiple entries
in the FEED table, and specifies the manner in which they are
combined. 

\item[POLARIZATION\_TYPE] Polarization type to which each receptor
responds (e.g. "R","L","X" or "Y"). This is the receptor polarization
type as recorded in the final correlated data (e.g. "RR"); i.e. as
measured after all polarization combiners.

\item[POL\_RESPONSE] Polarization response at the center of the beam
for this feed. Expressed in a linearly polarized basis $({\bf \vec
e_x}, {\bf \vec e_y})$ using the IEEE convention.

\item[POSITION] Offset of feed relative to the feed reference position
for this antenna (see ANTENNA sub-table).

\item[RECEPTOR\_ANGLE] Polarization reference angle. Converts into
parallactic angle in the sky domain.

\end{description}

\definetable{FIELD: Field positions for each source}{
\defline{\bf Columns}
\defline{\em Key}
\defline{\em Data}  
NAME &       String &    &      & Name of field \\
CODE &       String &    &      & Special characteristics of field \\
TIME &   Double  &      s  & EPOCH & Time origin for the directions and rates\\
NUM\_POLY & Int & & & Series order \\
DELAY\_DIR    &      Double(2, NUM\_POLY+1)  &      rad   &     DIRECTION & Direction of delay center. \\
PHASE\_DIR    &      Double(2, NUM\_POLY+1)  &      rad   &     DIRECTION & Phase center. \\
REFERENCE\_DIR &      Double(2, NUM\_POLY+1)  &      rad   &     DIRECTION & Reference center \\
SOURCE\_ID &      Int    &    &      & Index in Source table\\
{\it (EPHEMERIS\_ID)} &   Int    &    &      & Ephemeris id.\\
\defline{Flags}
FLAG\_ROW & Bool & & & Row flag\\
}
\begin{description}

\item[Notes:] The FIELD table defines a field position on the sky. For
interferometers, this is the correlated field position. For single dishes,
this is the nominal pointing direction.

\item[NAME] Field name; user specified.

\item[CODE] Field code indicating special characteristics of the
field; user specified.

\item[TIME] Time reference for the directions and rates. Required to 
use the same TIME Measure reference as in MAIN.

\item[NUM\_POLY] Series order for the *\_DIR columns.

\item[DELAY\_DIR] Direction of delay center; can be expressed as a
polynomial in time. Final result converted to the defined Direction
Measure type.

\item[PHASE\_DIR] Direction of phase center; can be expressed as a
polynomial in time. Final result converted to the defined Direction
Measure type.

\item[REFERENCE\_DIR] Reference center; can be expressed as a
polynomial in time. Final result converted to the defined Direction
Measure type. Used in single-dish to record the associated reference
direction if position-switching has already been applied. For
interferometric data, this is the original correlated field center,
and may equal DELAY\_DIR or PHASE\_DIR.

\item[SOURCE\_ID] Points to an entry in the optional SOURCE subtable, a
value of $-1$ indicates there is no corresponding source defined.

\item[EPHEMERIS\_ID] Points to an entry in the EPHEMERIS sub-table,
which defines the ephemeris used to compute the field position. Useful
for moving, near-field objects, where the ephemeris may be revised over time.

\item[FLAG\_ROW] True if data in this row are invalid, else False. Does not
imply flagging in MAIN.

\end{description}

\definetable{FLAG\_CMD: Flag commands}{
\defline{\bf Columns}
\defline{\em Key}
TIME      &    Double   &   s  &  EPOCH & Mid-point of interval \\
INTERVAL  &    Double   &   s  &        & Time interval \\
\defline{\em Data}  
TYPE      &    String   &      &        & FLAG or UNFLAG\\
REASON    &    String   &      &        & Flag reason\\
LEVEL     &    Int      &      &        & Flag level\\
SEVERITY  &    Int      &      &        & Severity code\\
APPLIED   &    Bool     &      &        & True if applied in MAIN\\
COMMAND   & String  &      &        & Flag command\\
}
\begin{description}

\item[Notes:] The FLAG\_CMD sub-table defines global flagging commands
which apply to the data in MAIN, as described in Section 3.1.8.

\item[TIME] Mid-point of the time interval to which this flagging
command applies. Required to use the same TIME Measure reference as
used in MAIN.

\item[INTERVAL] Time interval.

\item[TYPE] Type of flag command, representing either a flagging
("FLAG") or un-flagging ("UNFLAG") operation.

\item[REASON] Flag reason; user specified.

\item[LEVEL] Flag level $(\geq 0)$; reflects different revisions of
flags which have the same REASON.

\item[SEVERITY] Severity code for the flag, on a scale of 0-10 in
order of increasing severity; user specified.

\item[APPLIED] True if this flag has been applied to MAIN, and update
in FLAG\_CATEGORY and FLAG. False if this flag has not been applied to
MAIN.

\item[COMMAND] Global flag command, expressed in the standard
syntax for data selection, as adopted within the project as a whole.

\end{description}

\definetable{FREQ\_OFFSET: Frequency offset information}{
\defline{\bf Columns}
\defline{\em Key}
ANTENNA1  &   Int  & & & Antenna 1.\\
ANTENNA2  &   Int  & & & Antenna 2.\\
FEED\_ID     &   Int  & & & Feed id.\\
SPECTRAL\_WINDOW\_ID & Int & & & Spectral window id.\\
TIME & Double & s & EPOCH & Interval midpoint\\
INTERVAL & Double & s & & Time interval\\
\defline{\em Data}
OFFSET   &  Double & Hz &  & Frequency offset\\
}
\begin{description}

\item[Notes:] The table contains frequency offset information, to be
added directly to the defined frequency labeling in the
SPECTRAL\_WINDOW sub-table as a Measure offset. This allows bands with
small, time-variable, ad hoc frequency offsets to be labeled as the same
SPECTRAL\_WINDOW\_ID, and calibrated together if required.

\item[ANTENNA$n$] Antenna identifier, as indexed from ANTENNA$n$ in MAIN.

\item[FEED\_ID] Antenna identifier, as indexed from FEED$n$ in MAIN.

\item[SPECTRAL\_WINDOW\_ID] Spectral window identifier.

\item[TIME] Mid-point of the time interval for which this offset is valid.
Required to use the same TIME Measure reference as used in MAIN.

\item[INTERVAL] Time interval.

\item[OFFSET] Frequency offset to be added to the frequency axis for
this spectral window, as defined in the SPECTRAL\_WINDOW
sub-table. Required to have the same Frequency Measure reference as
CHAN\_FREQ in that table.

\end{description}

\definetable{HISTORY: History information}{
\defline{\bf Columns}
\defline{\em Key}
TIME & Double & s &              EPOCH  & Time-stamp for message\\
OBSERVATION\_ID  & Int   &          &        & Points to OBSERVATION table\\
\defline{\em Data}  
MESSAGE  & String &    &      & Log message\\
PRIORITY & String & & & Message priority\\
ORIGIN   & String & & & Code origin\\
OBJECT\_ID & String & & & Originating ObjectID \\
APPLICATION & String & & & Application name\\
CLI\_COMMAND & String(*) & & & CLI command sequence \\
APP\_PARAMS   & String(*) & & & Application paramters\\
}
\begin{description}

\item[Notes:] This sub-table contains associated history information
for the MS.

\item[TIME] Time-stamp for the history record. Required to have the
same TIME Measure reference as used in MAIN.

\item[OBSERVATION\_ID] Observation identifier (see the OBSERVATION table)

\item[MESSAGE] Log message.

\item[PRIORITY] Message priority, with allowed types: ("DEBUGGING",
"WARN", "NORMAL", or "SEVERE").

\item[ORIGIN] Source code origin from which message originated.

\item[OBJECT\_ID] Originating ObjectID, if available, else blank.

\item[APPLICATION] Application name.

\item[CLI\_COMMAND] CLI command sequence invoking the application.

\item[APP\_PARAMS] Application parameter values, in the adopted
project-wide format.
\end{description}

\definetable{OBSERVATION: Observation information}{
\defline{\bf Columns}
\defline{\em Data}  
TELESCOPE\_NAME &      String &    &      & Telescope name\\
TIME\_RANGE       &     Double(2) & s & EPOCH & Start, end times\\
OBSERVER &       String &    &      & Name of observer(s)\\
LOG &   String(*) & & & Observing log \\
SCHEDULE\_TYPE &  String & & & Schedule type\\
SCHEDULE &       String(*) &    &      & Project schedule\\
PROJECT &               String &    &      & Project identification string.\\
RELEASE\_DATE &         Double &  s  & EPOCH & Target release date\\
\defline{\em Flags}
FLAG\_ROW  & Bool & & & Row flag.\\
}

\begin{description}

\item[Notes:] This table contains information specifying the observing
instrument or epoch. See the discussion in Section 3.3 for details. It
is indexed directly from MAIN via OBSERVATION\_ID.

\item[TELESCOPE\_NAME] Telescope name (e.g. "WSRT" or "VLBA").

\item[TIME\_RANGE] The start and end times of the overall observing
period spanned by the actual recorded data in MAIN. Required to use
the same TIME Measure reference as in MAIN.

\item[OBSERVER] The name(s) of the observer(s).

\item[LOG] The observing log, as supplied by the telescope or instrument.

\item[SCHEDULE\_TYPE] The schedule type, with current reserved types
("VLBA-CRD", "VEX", "WSRT", "ATNF").

\item[SCHEDULE] Unmodified schedule file, of the type specified, and
as used by the instrument.

\item[PROJECT] Project code (e.g. "BD46")

\item[RELEASE\_DATE] Project release date. This is the date on which
the data may become public.

\item[FLAG\_ROW] Row flag. True if data in this row is invalid, but
does not imply any flagging in MAIN.
\end{description}

\definetable{POINTING: Antenna pointing information}{
\defline{\bf Columns}
\defline{\em Key}
ANTENNA\_ID  &  Int & & & Antenna id.\\
TIME         &  Double & s & EPOCH & Interval midpoint\\
INTERVAL     &  Double & s &       & Time interval\\
\defline{\em Data}
NAME     &   String & & & Pointing position desc.\\
NUM\_POLY & Int & & & Series order \\
TIME\_ORIGIN & Double & s & EPOCH & Origin for the polynomial\\
DIRECTION & Double(2, NUM\_POLY+1) & rad & DIRECTION & 
  Antenna pointing direction\\
TARGET & Double(2, NUM\_POLY+1) & rad & DIRECTION & Target direction \\
{\it (POINTING\_OFFSET)} & Double(2, NUM\_POLY+1) & rad & DIRECTION &
  A priori pointing correction \\
{\it (SOURCE\_OFFSET)} & Double(2, NUM\_POLY+1) & rad & DIRECTION & Offset from source\\
{\it (ENCODER)} & Double(2) & rad & DIRECTION & Encoder values\\
{\it (POINTING\_MODEL\_ID)} & Int & & & Pointing model id. \\
TRACKING & Bool & & & True if on-position \\
{\it(ON\_SOURCE)} & Bool & & & True if on-source\\
{\it(OVER\_THE\_TOP)} & Bool & & & True if over the top\\
}
\begin{description}

\item[Notes:] This table contains information concerning the primary
pointing direction of each antenna as a function of time. Note that
the pointing offsets for inidividual feeds on a given antenna are
specified in the FEED sub-table with respect to this pointing direction.

\item[ANTENNA\_ID] Antenna identifier, as specified by ANTENNA$n$ in MAIN.

\item[TIME] Mid-point of the time interval for which the information
in this row is valid. Required to use the same TIME Measure reference
as in MAIN.

\item[INTERVAL] Time interval.

\item[NAME] Pointing direction name; user specified.

\item[NUM\_POLY] Series order for the polynomial expressions in
DIRECTION and POINTING\_OFFSET.

\item[TIME\_ORIGIN] Time origin for the polynomial expansions.

\item[DIRECTION] Antenna pointing direction, optionally expressed as
polynomial coefficients. The final result is interpreted as a
Direction Measure using the specified Measure reference. 

\item[TARGET] Target pointing direction, optionally expressed as
polynomial coefficients. The final result is interpreted as a
Direction Measure using the specified Measure reference. This is the
true expected position of the source, including all coordinate
corrections such as precession, nutation etc.

\item[POINTING\_OFFSET] The a priori pointing corrections applied by
the telescope in pointing to the DIRECTION position, optionally
expressed as polynomial coefficients. The final result is interpreted
as a Direction Measure using the specified Measure reference.

\item[SOURCE\_OFFSET] The commanded offset from the source position,
if offset pointing is being used.

\item[ENCODER] The current encoder values on the primary axes of the
mount type for the antenna, expressed as a Direction Measure.

\item[TRACKING] True if tracking the nominal pointing position.

\item[ON-SOURCE] True if the nominal pointing direction coincides with
the source, i.e. offset-pointing is not being used.

\item[OVER-THE-TOP] True if the antenna was driven to this position
"over the top" (az-el mount).
\end{description}

\definetable{POLARIZATION: Polarization setup information}{
\defline{\bf Columns}
\defline{\em Data description columns}
NUM\_CORR   &     Int  &           &        & \# correlations\\
\defline{\em Data}  
CORR\_TYPE & Int(NUM\_CORR) & & & Polarization of correlation \\
CORR\_PRODUCT &  Int(2, NUM\_CORR) &  & & Receptor cross-products \\  
\defline{\em Flags}
FLAG\_ROW & Bool & & & Row flag\\
}
\begin{description}

\item[Notes:] This table defines the polarization labeling of the DATA
array in MAIN, and is directly indexed from the DATA\_DESCRIPTION
table via POLARIZATION\_ID.

\item[NUM\_CORR] The number of correlation polarization products. For
example, for (RR) this value would be 1, for (RR, LL) it would be 2,
and for (XX,YY,XY,YX) it would be 4, etc.

\item[CORR\_TYPE] An integer for each correlation product indicating
the Stokes type as defined in the Stokes class enumeration.

\item[CORR\_PRODUCT] Pair of integers for each correlation product,
specifying the receptors from which the signal originated.  The
receptor polarization is defined in the POLARIZATION\_TYPE column in
the FEED table. An example would be (0,0), (0,1), (1,0), (1,1) to
specify all correlations between two receptors.

\item[FLAG\_ROW] Row flag. True is the data in this row are not valid,
but does not imply the flagging of any DATA in MAIN.

\end{description}

\definetable{PROCESSOR: Processor information}{
\defline{\bf Columns}
\defline{\em Data}
TYPE            & String & & & Processor type \\
SUB\_TYPE & String & & & Processor sub-type\\
TYPE\_ID  & Int & & & Processor type id.\\
MODE\_ID  & Int & & & Processor mode id.\\
{\it (PASS\_ID)}  & Int & & & Processor pass number\\
\defline{\em Flags}
FLAG\_ROW & Bool & & & Row flag\\
}
\begin{description}

\item[Notes:] This table holds summary information for the back-end
processing device used to generate the basic data in the MAIN
table. Such devices include correlators, radiometers, spectrometers,
pulsar-timers, amongst others. See Section 4.0.4 for further details.

\item[TYPE] Processor type; reserved keywords include ("CORRELATOR" -
interferometric correlator; "SPECTROMETER" - single-dish correlator;
"RADIOMETER" - generic detector/integrator; "PULSAR-TIMER" - pulsar
timing device).

\item[SUB\_TYPE] Processor sub-type, e.g. "GBT" or "JIVE".

\item[TYPE\_ID] Index used in a specialized sub-table named as
$subtype\_type$, which contains time-independent processor information
applicable to the current data record (e.g. a JIVE\_CORRELATOR
sub-table).  Time-dependent information for each device family is
contained in other tables, dependent on the device type.

\item[MODE\_ID] Index used in a specialized sub-table named as
$subtype\_type\_mode$, containing information on the processor mode
applicable to the current data record. (e.g. a GBT\_SPECTROMETER\_MODE
sub-table).

\item[PASS\_ID] Pass identifier; this is used to distinguish data records
produced by multiple passes through the same device, where this is
possible (e.g. VLBI correlators). Used as an index into the associated
table containing pass information.

\item[FLAG\_ROW] Row flag. True if data in the row is not valid, but
does not imply flagging in MAIN.

\end{description}

\definetable{SOURCE: Source information}{
\defline{\bf Columns}
\defline{\em Key}
SOURCE\_ID &      Int &    &      & Source id\\
TIME &   Double & s  &   EPOCH  & Midpoint of time for which this set of parameters is accurate\\
INTERVAL &       Double & s &               & Interval\\
SPECTRAL\_WINDOW\_ID &   Int &    &      & Spectral Window id\\
\defline{\em Data description}
NUM\_LINES &  Int & & & Number of spectral lines\\
\defline{\em Data}  
NAME &       String &    &      & Name of source as given during observations\\
CALIBRATION\_GROUP &      Int &    &      & \# grouping for calibration purpose\\
CODE &       String &    &      & Special characteristics of source, e.g. Bandpass calibrator\\
DIRECTION &       Double(2) &  rad  &    DIRECTION &        Direction (e.g. RA, DEC) \\
{\it (POSITION)}  & Double(3) &   m  & POSITION & Position (e.g. for solar system objects) \\
PROPER\_MOTION &  Double(2) &  rad/s & &  Proper motion \\
{\it (TRANSITION)} & String(NUM\_LINES) & & & Transition name\\
{\it (REST\_FREQUENCY)} & Double(NUM\_LINES) & Hz & FREQUENCY & Line rest frequency\\
{\it (SYSVEL)}   & Double(NUM\_LINES)  &  m/s  & RADIAL VELOCITY & Systemic velocity at reference \\
{\it (SOURCE\_MODEL)} & TableRecord & & & Default csm \\
{\it (PULSAR\_ID)} & Int & & & Pulsar id. \\
}
\begin{description}

\item[Notes:] This table contains time-variable source information,
optionally associated with a given FIELD\_ID.

\item[SOURCE\_ID] Source identifier ($\geq 0)$, as specified in the
FIELD sub-table.

\item[TIME] Mid-point of the time interval for which the data in this
row is valid. Required to use the same TIME Measure reference as in
MAIN.

\item[INTERVAL] Time interval.

\item[SPECTRAL\_WINDOW\_ID] Spectral window identifier. A -1 indicates
that the row is valid for all spectral windows.

\item[NUM\_LINES] Number of spectral line transitions associated with
this source and spectral window id. combination.

\item[NAME] Source name; user specified.

\item[CALIBRATION\_GROUP] Calibration group number to which this source
belongs; user specified.

\item[CODE] Source code, used to describe any special characteristics
f the source, such as the nature of a calibrator. Reserved keyword,
including ("BANDPASS CAL").

\item[DIRECTION] Source direction at this TIME.

\item[POSITION] Source position $(x,y,z)$ at this TIME (for near-field
objects).

\item[PROPER\_MOTION] Source proper motion at this TIME.

\item[TRANSITION] Transition names applicable for this spectral window
(e.g. "v=1, J=1-0, SiO").

\item[REST\_FREQUENCY] Rest frequencies for the transitions.

\item[SYSVEL] Systemic velocity for each transition.

\item[SOURCE\_MODEL] Reference to an assigned component source model
table.

\item[PULSAR\_ID] An index used in the PULSAR sub-table to define
further pulsar-specific properties if the source is a pulsar.

\end{description}

\definetable{SPECTRAL\_WINDOW: Spectral window description}{
\defline{\bf Columns}
\defline{\em Data description columns}
NUM\_CHAN   &     Int  &           &        & \# spectral channels \\
\defline{\em Data}  
NAME & String & & & Spectral window name\\
REF\_FREQUENCY     &   Double       &               Hz  &   FREQUENCY&        The reference frequency.\\
CHAN\_FREQ  &   Double(NUM\_CHAN)    &           Hz    &    FREQUENCY& Center frequencies for each channel in the data matrix.\\
CHAN\_WIDTH &   Double(NUM\_CHAN)    &           Hz    &    & Channel width for each channel in the data matrix.\\
MEAS\_FREQ\_REF & Int      & & & FREQUENCY Measure ref.\\
EFFECTIVE\_BW   &     Double(NUM\_CHAN)         &      Hz      &  &
The effective noise bandwidth of each spectral channel\\
RESOLUTION   &     Double(NUM\_CHAN)         &      Hz      &  &
The effective spectral resolution of each channel\\
TOTAL\_BANDWIDTH & Double & Hz &  & total bandwidth for this window \\
NET\_SIDEBAND & Int & & & Net sideband\\
{\it (BBC\_NO)} & Int & & & Baseband converter no.\\
{\it (BBC\_SIDEBAND)} & Int & & & BBC sideband\\
IF\_CONV\_CHAIN & Int & & & The IF conversion chain \\
{\it (RECEIVER\_ID)}    & Int & & & Receiver id.\\
FREQ\_GROUP & Int & & & Frequency group\\
FREQ\_GROUP\_NAME & String & & & Freq. group name \\
{\it (DOPPLER\_ID)} & Int & & & Doppler id.\\
{\it (ASSOC\_SPW\_ID)} & Int(*) & & & Associated spw\_id.\\
{\it (ASSOC\_NATURE)} & String(*) & & & Nature of association\\
\defline{Flags}
FLAG\_ROW & Bool & & &\\

} \begin{description}

 \item[Notes:] This table describes properties for each defined
spectral window. A spectral window is both a frequency label for the
associated DATA array in MAIN, but also represents a generic frequency
conversion chain that shares joint physical properties and makes sense
to calibrate as a single entity.

\item[NUM\_CHAN] Number of spectral channels.

\item[NAME] Spectral window name; user specified.

\item[REF\_FREQUENCY] The reference frequency. A frequency
representative of this spectral window, usually the sky frequency
corresponding to the DC edge of the baseband. Used by the calibration
system if a fixed scaling frequency is required or in algorithms to
identify the observing band.

\item[CHAN\_FREQ] Center frequencies for each channel in the data
matrix. These can be frequency-dependent, to accommodate instruments
such as acousto-optical spectrometers. Note that the channel
frequencies may be in ascending or descending frequency order.

\item[CHAN\_WIDTH] Nomical channel width of each spectral
channel. Although these can be derived from CHAN\_FREQ by
differencing, it is more efficient to keep a separate reference to
this information.

\item[MEAS\_FREQ\_REF] Frequency Measure reference for
CHAN\_FREQ. This allows a row-based reference for this column in order to
optimize the choice of Measure reference when Doppler tracking is
used. Modified only by the MS access code.

\item[EFFECTIVE\_BW] The effective noise bandwidth of each spectral channel.

\item[RESOLUTION] The effective spectral resolution of each channel.

\item[TOTAL\_BANDWIDTH] The total bandwidth for this spectral window.

\item[NET\_SIDEBAND] The net sideband for this spectral window.

\item[BBC\_NO] The baseband converter number, if applicable.

\item[BBC\_SIDEBAND] The baseband converter sideband, is applicable.

\item[IF\_CONV\_CHAIN] Identification of the electronic signal
path for the case of multiple (simultaneous) IFs. (e.g. VLA: AC=0,
BD=1, ATCA: Freq1=0, Freq2=1)

\item[RECEIVER\_ID] Index used to identify the receiver associated
with the spectral window. Further state information is planned to be
stored in a RECEIVER sub-table.

\item[FREQ\_GROUP] The frequency group to which the spectral window belongs.
This is used to associate spectral windows for joint calibration purposes.

\item[FREQ\_GROUP\_NAME] The frequency group name; user specified.

\item[DOPPLER\_ID] The Doppler identifier defining frame information
for this spectral window.

\item[ASSOC\_SPW\_ID] Associated spectral windows, which are related
in some fashion (e.g. "channel-zero").

\item[ASSOC\_NATURE] Nature of the association for ASSOC\_SPW\_ID;
reserved keywords are ("CHANNEL-ZERO" - channel zero;
"EQUAL-FREQUENCY" - same frequency labels; "SUBSET" - narrow-band
subset).

\item[FLAG\_ROW] True if the row does not contain valid data.

\end{description}

\definetable{STATE: State information}{
\defline{\bf Columns}
\defline{\em Data}  
SIG   &     Bool  &           &        & Signal \\
REF   &     Bool  &           &        & Reference \\
CAL   &     Double &   K      &        & Noise calibration \\
LOAD  &     Double &   K      &        & Load temperature\\
SUB\_SCAN &  Int   &          &        & Sub-scan number\\
OBS\_MODE &  String &         &        & Observing mode\\
\defline{\em Flags}
FLAG\_ROW & Bool &            &        & Row flag\\
}
\begin{description}

\item[Notes:] This table defines the state parameters for a particular
data record as they refer to external loads, calibration sources or
references, and also characterizes the observing mode of the data
record, as an aid to defining the scheduling heuristics. It is indexed
directly via STATE\_ID in MAIN.

\item[SIG] True if the source signal is being observed.

\item[REF] True for a reference phase.

\item[CAL] Noise calibration temperature (zero if not added).

\item[LOAD] Load temperature (zero if no load).

\item[SUB\_SCAN] Sub-scan number $(\geq 0)$, relative to the
SCAN\_NUMBER in MAIN. Used to identify observing sequences.

\item[OBS\_MODE] Observing mode; defined by a set of reserved keywords
characterizing the current observing mode (e.g. "OFF-SPECTRUM"). Used
to define the schedule strategy.

\item[FLAG\_ROW] True if the row does not contain valid data. Does not
imply flagging in MAIN.

\end{description}

\definetable{SYSCAL: System calibration}{
\defline{\bf Columns}
\defline{\em Key}
ANTENNA\_ID &     Int &    &      & Antenna id\\
FEED\_ID &        Int &    &      & Feed id\\
SPECTRAL\_WINDOW\_ID &   Int &    &      & Spectral window id\\
TIME      &   Double  &      s  & EPOCH & Midpoint of time for which this set of parameters is accurate\\
INTERVAL &       Double  &      s     &   & Interval\\
\defline{\em Data}  
{\it (PHASE\_DIFF)} &  Float & rad &    & Phase difference between receptor 0 and receptor 1\\
{\it (TCAL)} &       Float (\nr)  &      K &  & Calibration temp\\
{\it (TRX)}  &       Float (\nr)  &      K &  & Receiver temperature\\
{\it (TSKY)} &       Float (\nr)  &      K &  & Sky temperature\\
{\it (TSYS)} &       Float (\nr)  &      K &  & System temp\\
{\it (TANT)} &       Float (\nr)  &      K &  & Antenna temperature\\
{\it (TANT\_TSYS)} & Float(\nr) &  & & ${{T_{ant}}\over{T_{sys}}}$\\
{\it (TCAL\_SPECTRUM)} &       Float (\nr, \nf)  &      K &  & Calibration temp\\
{\it (TRX\_SPECTRUM)}  &       Float (\nr, \nf)  &      K &  & Receiver temperature\\
{\it (TSKY\_SPECTRUM)}  &       Float (\nr, \nf)  &      K &  & Sky temperature spectrum\\
{\it (TSYS\_SPECTRUM)} &       Float (\nr, \nf)  &      K &  & System temp\\
{\it (TANT\_SPECTRUM)} &       Float (\nr, \nf)  &      K &  & Antenna temperature spectrum\\
{\it (TANT\_TSYS\_SPECTRUM)} & Float (\nr,\nf) &  & & ${{T_{ant}}\over{T_{sys}}}$ spectrum\\
\defline{\em Flags}
{\it (PHASE\_DIFF\_FLAG)} &   Bool  &    &      & Flag for PHASE\_DIFF\\
{\it (TCAL\_FLAG)} &  Bool &    &      & Flag for TCAL\\
{\it (TRX\_FLAG)} &   Bool  &    &      & Flag for TRX\\
{\it (TSKY\_FLAG)} & Bool &      &      & Flag for TSKY\\
{\it (TSYS\_FLAG)} &  Bool &    &      & Flag for TSYS\\
{\it (TANT\_FLAG)} & Bool & & & Flag for TANT\\
{\it (TANT\_TSYS\_FLAG)} & Bool & & & Flag for ${{T_{ant}}\over{T_{sys}}}$\\
}
\begin{description}

\item[Notes:] This table contains time-variable calibration
measurements for each antenna, as indexed on feed and spectral window.
Note that \nr = number of receptors, and \nf = number of
frequency channels.

\item[ANTENNA\_ID] Antenna identifier, as indexed by ANTENNA$n$ in MAIN.

\item[FEED\_ID] Feed identifier, as indexed by FEED$n$ in MAIN.

\item[SPECTRAL\_WINDOW\_ID] Spectral window identifier.

\item[TIME] Mid-point of the time interval for which the data in this
row are valid. Required to use the same TIME Measure reference as that
in MAIN.

\item[INTERVAL] Time interval.

\item[PHASE\_DIFF] Phase difference between receptor 0 and receptor 1.

\item[TCAL] Calibration temperature.

\item[TRX] Receiver temperature.

\item[TSKY] Sky temperature.

\item[TSYS] System temperature.

\item[TANT] Antenna temperature.

\item[TANT\_TSYS] Antenna temperature over system temperature.

\item[TCAL\_SPECTRUM] Calibration temperature spectrum.

\item[TRX\_SPECTRUM] Receiver temperature spectrum.

\item[TSKY\_SPECTRUM] Sky temperature spectrum.

\item[TSYS\_SPECTRUM] System temperature spectrum.

\item[TANT\_SPECTRUM] Antenna temperature spectrum.

\item[TANT\_TSYS\_SPECTRUM] Antenna temperature over system
temperature spectrum.

\item[PHASE\_DIFF\_FLAG] True if PHASE\_DIFF flagged.

\item[TCAL\_FLAG] True if TCAL flagged.

\item[TRX\_FLAG] True if TRX flagged.

\item[TSKY\_FLAG] True if TSKY flagged.

\item[TSYS\_FLAG] True if TSYS flagged.

\item[TANT\_FLAG] True if TANT flagged.

\item[TANT\_TSYS\_FLAG] True if TANT\_TSYS flagged.

\end{description}


\definetable{WEATHER: weather station information}{
\defline{\bf Columns}
\defline{\em Key}
ANTENNA\_ID &        Int &    &      & Antenna number\\
TIME &  Double  &   s  & EPOCH  & Mid-point of interval\\ 
INTERVAL &       Double &        s  &      & Interval over which data is relevant\\
\defline{\em Data}  
{\it (H2O)} &        Float    &     $m^{-2}$  &      & Average column density of water\\
{\it (IONOS\_ELECTRON)} & Float & $m^{-2}$    &      & Average column density of electrons\\
{\it (PRESSURE)} &       Float & hPa  &      & Ambient atmospheric pressure\\
{\it (REL\_HUMIDITY)} &   Float &    &      & Ambient relative humidity\\
{\it (TEMPERATURE)} &    Float &  K  &     & Ambient air temperature for an antenna\\
{\it (DEW\_POINT)} & Float & K & & Dew point \\
{\it (WIND\_DIRECTION)} & Float  &       rad &      & Average wind direction\\
{\it (WIND\_SPEED)} &  Float &  m/s  &    & Average wind speed \\
\defline{\em Flags}
{\it (H2O\_FLAG)} & Bool & & & Flag for H2O \\
{\it (IONOS\_ELECTRON\_FLAG)} & Bool & & & Flag for IONOS\_ELECTRON \\
{\it (PRESSURE\_FLAG)} & Bool & & & Flag for PRESSURE \\
{\it (REL\_HUMIDITY\_FLAG)} & Bool & & & Flag for REL\_HUMIDITY \\
{\it (TEMPERATURE\_FLAG)} & Bool & & & Flag for TEMPERATURE \\
{\it (DEW\_POINT\_FLAG)} & Bool & & & Flag for DEW\_POINT \\
{\it (WIND\_DIRECTION\_FLAG)} & Bool & & & Flag for WIND\_DIRECTION \\
{\it (WIND\_SPEED\_FLAG)} & Bool & & & Flag for WIND\_SPEED \\
}
\begin{description}

\item[Notes:] This table contains mean external atmosphere and weather
information.

\item[ANTENNA\_ID] Antenna identifier, as indexed by ANTENNA$n$ from MAIN.

\item[TIME] Mid-point of the time interval over which the data in the
row are valid. Required to use the same TIME Measure reference as in
MAIN.

\item[INTERVAL] Time interval.

\item[H2O] Average column density of water.

\item[IONOS\_ELECTRON] Average column density of electrons.

\item[PRESSURE] Ambient atmospheric pressure.

\item[REL\_HUMIDITY] Ambient relative humidity.

\item[TEMPERATURE] Ambient air temperature.

\item[DEW\_POINT] Dew point temperature.

\item[WIND\_DIRECTION] Average wind direction.

\item[WIND\_SPEED] Average wind speed.

\item[H2O\_FLAG] Flag for H2O.

\item[IONOS\_ELECTRON\_FLAG] Flag for IONOS\_ELECTRON.

\item[PRESSURE\_FLAG] Flag for PRESSURE.

\item[REL\_HUMIDITY\_FLAG] Flag for REL\_HUMIDITY.

\item[TEMPERATURE\_FLAG] Flag for TEMPERATURE.

\item[DEW\_POINT\_FLAG] Flag for DEW\_POINT.

\item[WIND\_DIRECTION\_FLAG] Flag for DEW\_POINT.

\item[WIND\_SPEED\_FLAG] Flag for DEW\_POINT.

\end{description}

\section{Correction and additions}

\begin{description}

\item{{\bf 2000/08/14:} Add expanded description of Doppler tracking to
Section 4.0.8; correctly identify SOURCE sub-table as optional; expand
definition of SPECTRAL\_WINDOW\_ID=-1 in the description of the SOURCE
sub-table; expand the definition of the REF\_FREQUENCY in the
SPECTRAL\_WINDOW sub-table description.}
\item{{\bf 2000/08/28:} Make POSITION field in SOURCE table optional.}
\item{{\bf 2000/09/12:} Add required VELDEF column in DOPPLER table.}

\end{description}

\section{References}

\noindent Date, C.J., 1986, Introduction to database systems, (Addison-Wesley).

\noindent Flatters, C., 1998, AIPS Memo. 102.

\noindent Noordam, J., 1996, aips++ Note 185.

\noindent Wieringa, M.H., and Cornwell, T., 1996, aips++ Note 191.

\end{document}

