\chapter{Code management}
\label{Code management}
\index{code!management}
\index{age!source code|see{code, management}}
\index{checkin|see{code, management}}
\index{checkout|see{code, management}}
\index{directory!management|see{code, management}}
\index{file!management|see{code, management}}
\index{major version number|see{code, management}}
\index{minor version number|see{code, management}}
\index{patch version number|see{code, management}}
\index{rcs@\rcs!code management|see{code, management}}
\index{rename|see{code, management}}
\index{search!source code|see{code, management}}
\index{update!source code!selectively|see{code, management}}
\index{verification!source code|see{code, management}}
\index{version|see{code, management}}

\aipspp\ source code management utilities \footnote{Most recent change:
$ $Id$ $}.

% ----------------------------------------------------------------------------

\section{\exe{ac}}
\label{ac}
\index{ac@\exe{ac}}
\index{master host}
\index{code!management!changelog|see{\exe{ac}}}
\index{changelog|see{\exe{ac}}}

Add an entry to an \aipspp\ changelog file.

\subsection*{Synopsis}

\begin{synopsis}
   \code{\exe{ac} [\textrm{options for} \aipsexe{ac}]}
\end{synopsis}

\subsection*{Description}

\exe{ac} adds an entry to an \aipspp\ changelog file.

\exe{ac} assumes that you are working within a private directory tree which
shadows that of the \aipspp\ master source code tree, see \exeref{mktree}.
It determines the appropriate changelog file from the current working
directory.

\exe{ac} invokes \exeref{ao} to check out the appropriate changelog
file. It also checks out another file which is used to generate a
unique id for each change. Thereafter a glish script is started
to let the user fill in a form for the change and to add the
change to the changelog file. The user has the opportunity to
abort the operation, in which case the checked out files are unlocked.
In the change is added, \exeref{ai} is invoked to check in the changed
files.

\subsection*{Options}

\begin{description}
\item[\exe{-i}]
   Do only a check in of the changed files.
   This should only be used if the normal (automatic) check in
   failed because an \exeref{exhale} has started or due to another error.
\end{description}

\subsection*{Notes}

\begin{itemize}
\item
   The \aipspp\ changelog files are meant to generate an overview
   (by \exeref{buildchangelog}) of the new and changed features
   in an \aipspp\ release.
   It means they are higher level than the RCS changelogs which are
   programmer oriented.

\item
   Almost every source directory has a changelog file to keep track
   of the changes in the \aipspp\ system.
   \\All glish subdirectories share a single changelog file in their
   root directory. The same is true for the system and the
   documentation subdirectories.

\item
   The DISPLAY variable needs to be set, because \exe{ac} uses
   a GUI to let the user fill in the change form.

\item
   If checking out fails because the file does not exist yet,
   one should create an empty file \textrm{changelog} and check it
   in using \exeref{ai}.
   In general this should only occur in new directories.

\item
   Checking in should normally succeed, but might fail because the
   an \exeref{exhale} has started or because network connection is down.
   In that case \exe{ac} should be run at a later time with the
   \textrm{-i} flag (to only do a checkin of the files).
\end{itemize}

\subsection*{Diagnostics}

Status return values
\\ \verb+   0+: success
\\ \verb+   1+: initialization error
\\ \verb+   2+: checkout failure
\\ \verb+   3+: checkin failure
\\ \verb+   4+: no change entry created by user

\subsection*{Examples}

The command

\begin{verbatim}
   cd ~/aips++/code/aips/implement/Tables
   ac
\end{verbatim}

\noindent
will let the user fill in the change form and add the completed
form to the changelog file of library module Tables.

\subsection*{Bugs}

\begin{itemize}
\item
   \exe{ac} does not detect why a checkout or checkin fails.
\end{itemize}

\subsection*{See also}

\exeref{rac} \aipspp\ ac for remote users.\\
\exeref{ai} \aipspp\ code checkin utility.\\
\exeref{ao} \aipspp\ code checkout utility.\\
\exeref{buildchangelog} process \aipspp\ changelog files.

\subsection*{Author}

Original: 2000/06/05 by Ger van Diepen, NFRA

% ----------------------------------------------------------------------------

\newpage
\section{\exe{ae}}
\label{ae}
\index{ae@\exe{ae}}
\index{master host}
\index{code!management!semaphore file deletion|see{\exe{ae}}}
\index{file!semaphore|see{\exe{ae}}}
\index{rcs@\rcs!semaphore file!deletion|see{\exe{ae}}}
\index{semaphore file deletion|see{\exe{ae}}}

Extricate an \aipspp\ source file from a half-completed checkin/out.

\subsection*{Synopsis}

\begin{synopsis}
   \code{\exe{ae} file}
\end{synopsis}

\subsection*{Description}

\exe{ae} extricates a source file in the current directory from a
half-completed \rcs\ checkin/out.  Such problems sometimes arise when the
checkin/out is interrupted part way through leaving behind an \rcs\ semaphore
file.

The presence of a defunct \rcs\ semaphore file is indicated by repeated
``in use'' errors when a checkin or checkout of a file is attempted.

\exe{ae} assumes that you are working within a private directory tree which
shadows that of the \aipspp\ master source code tree, see \exeref{mktree}.
It determines the corresponding master directory from the current working
directory.

The filename argument must be a simple filename without path or \rcs\ 
\file{,v} suffix.

\exe{ae} invokes \file{\$MSTRETCD/ae\_master}, where \file{\$AIPSMSTR} is
expected to be \textsc{nfs} mounted from the \aipspp\ master host (see
\sref{RCS directories}).  It waits till the semaphore file is at least 15
minutes old before deleting it in case a slow \rcs\ transaction is in
progress.

\subsection*{Options}

None.

\subsection*{Notes}

\begin{itemize}
\item
   The semaphore file has a name of the form \file{,file,} and resides in the
   same directory as the master \rcs\ version file (\file{file,v}).

\item
   The utility which generates updates for the \aipspp\ code distribution
   system \exeref{exhale} routinely searches for and deletes all
   semaphore files which are more than a day old.  \exe{ae} allows particular
   semaphore files to be deleted on a much shorter timescale.

\item
   \exe{ae} uses an \aipspp\ utility called \exeref{tract} to
   determine the age of the \rcs\ semaphore file.
\end{itemize}

\subsection*{Diagnostics}

Status return values
\\ \verb+   0+: success
\\ \verb+   1+: initialization error
\\ \verb+   2+: semaphore file not found
\\ \verb+   3+: failed to delete semaphore file

\subsection*{Examples}

The command

\begin{verbatim}
   cd ~/aips++/code/aips/implement/FooBar
   ae Foo.cc
\end{verbatim}

\noindent
would delete the \rcs\ semaphore file for \file{Foo.cc} if it existed.

\subsection*{Bugs}

\begin{itemize}
\item
   \exe{ae} requires that the host operating system support setuid shell
   scripts.  Some operating systems, most notably Linux and Digital
   UNIX, do not support setuid shell scripts for security reasons.
\end{itemize}

\subsection*{See also}

The manual page for \unixexe{ci}(1), the \rcs\ checkin command.\\
\aipspp\ variable names (\sref{variables}).\\
\aipspp\ code management configuration (\sref{RCS directories}).\\
\exeref{ai} \aipspp\ code checkin utility.\\
\exeref{ao} \aipspp\ code checkout utility.\\
\exeref{mktree} create \aipspp\ directory hierarchy.\\
\exeref{tract} report the age of a file or directory.

\subsection*{Author}

Original: 1995/02/23 by Mark Calabretta, ATNF.

% ----------------------------------------------------------------------------

\section{\exe{ai}}
\label{ai}
\index{ai@\exe{ai}}
\index{master host}
\index{ci@\exe{ci}|see{\exe{ai}}}
\index{code!management!checkin|see{\exe{ai}}}
\index{directory!checkin|see{\exe{ai}}}
\index{file!checkin|see{\exe{ai}}}

\aipspp\ source code checkin utility.

\subsection*{Synopsis}

\begin{synopsis}
   \code{\exe{ai} [\exe{-ae}] [\exe{-af}] [\textrm{options for} \unixexe{ci}]
      file1 [file2,...]}
\end{synopsis}

\subsection*{Description}

\exe{ai} checks in source files from the current directory into the \aipspp\ 
master \rcs\ repository.  It is a front-end for the \rcs\ \unixexe{ci}
command.

\exe{ai} assumes that you are working within a private directory tree which
shadows that of the \aipspp\ master source code tree, see \exeref{mktree}.
The filename arguments specified to \exe{ai} must be simple filenames without
path or \rcs\ \file{,v} suffix.  \exe{ai} determines the corresponding master
directory from the current working directory.

Directories may also be created on the master by ``checking them in''.  In
this case any options for \unixexe{ci} are ignored.  \file{RCS} subdirectories
are explicitly excluded from directory checkin.

If an \aipspp\ copyright notice exists in the file to be checked in \exe{ai}
will ensure that it is kept up-to-date for the current year (unless the
\exe{-oc} option is specified, see below).

If the \exe{-au} option is specified, then after checking the code into the
master, \exe{ai} copies the \rcs\ version file back to the local slave
\rcs\ repository (if it exists), giving it a file ownership defined by the
owner of \exe{ai} (\acct{aips2mgr} at the local site), which must itself be
setuid.  If this completed successfully, \exe{ai} then checks out a plain-text
copy of the code from the slave \rcs\ repository into the corresponding code
directory.

\exe{ai} updates the master \aipspp\ \rcs\ repository by executing
\file{\$MSTRETCD/ai\_master}, where \file{\$AIPSMSTR} is expected to be
\textsc{nfs} mounted from the \aipspp\ master host (see
\sref{RCS directories}).  Programmers who have an account on the \aipspp\ 
master host, \host{aips2.nrao.edu}, may instead use a related utility,
\exeref{rai}, which invokes \exe{ai} remotely on the master host via
\unixexe{ssh}.

\subsection*{Options}

\begin{description}
\item[\exe{-ae}]
   For each file to be checked in, invoke \exeref{ae} to delete any
   pre-existing \rcs\ semaphore file.

\item[\exe{-af}]
   \exe{ai} normally verifies that a file to be checked in does exist and will
   skip it if it doesn't on the assumption that the filename may have been
   misspelt.  This behaviour may be overridden via the \exe{-af} option which
   forces the creation of an empty \rcs\ version file on the master.  It is
   mainly for use by the \aipspp\ ``code cops'' to allow the true author of a
   file to check in the initial version.

\item[\exe{-au}]
   Update the slave repository after the checkin as explained above.

\item[\exe{-oc}]
   Override the automatic update of the copyright notice.  Automatic copyright
   updating can also be defeated \emph{en masse} by creating an empty file
   called ``\file{.oc}'' in the master directory.

\item[\code{\exe{-I} | \exe{-l} | \exe{-r} | \exe{-u} | \exe{-w}login}]
   These options (without revision number) are passed directly to
   \unixexe{ci}, check the manual page for \unixexe{ci} for an explanation.
   All other \unixexe{ci} options are defeated.
\end{description}

\subsection*{Notes}

\begin{itemize}
\item
   Checkins by \acct{aips2mgr} are explicitly defeated to discourage anonymous
   changes.

\item
   Code checked into the master repository will have the file ownership
   defined by the owner of \file{\$MSTRETCD/ai\_master} (\acct{aips2adm}).
   This script must have the \code{setuid} bit set (see \unixexe{chmod}).

\item
   While \exeref{mktree} creates \rcs\ symbolic links in the programmer
   directory hierarchy, these are not used by \exe{ai}.  Their main function
   is to allow \exeref{gmake} to automatically check out the \file{makefile}
   for a directory.  They also allow \rcs\ utilities such as \unixexe{rlog} to
   be applied readily to the slave \rcs\ repository.  \exe{ai} does not make
   use of these \rcs\ symbolic links but deals directly with the \aipspp\ 
   master repository by specifying absolute pathnames.

\item
   Checkins may be disabled by creating a file \file{\$MSTRETCD/ai\_disable}.
   If this file contains any text it will be printed as an informative
   message.

\item
   Checkin of new files in a particular directory can be restricted by
   creating a \file{.ai\_control} file in the master directory.  Only account
   names listed within that file will be permitted to check new files into
   that directory.

\item
   \exe{ai} uses \exeref{av} to verify the filename when a file is created and
   creates an addendum to the \aipsexe{av} filename cache for each new file
   checked in.

\item
   \exe{ai} maintains \rcs\ revision numbers and symbolic names.  It sets the
   release number on the \rcs\ \code{trunk} to coincide with the major
   \aipspp\ version number.  Further, whenever a new revision of a file is
   checked in \exe{ai} assigns it a symbolic name which matches the \aipspp\ 
   version number of the master at the time of the checkin.  Assignment of
   these symbolic names allows the version of a file current for any
   particular \aipspp\ version to be recovered easily.

\item
   Files to be checked in must have read permission for user and group.
   \exe{ai} automatically adds this.
\end{itemize}

\subsection*{Diagnostics}

Status return values
\\ \verb+   0+: success
\\ \verb+   1+: initialization error
\\ \verb+   2+: checkin failure
\\ \verb+   3+: checked in ok to master, but slave not updated
\\ \verb+   4+: code not updated
\\ \verb+   5+: directory creation failed

\subsection*{Examples}

The command

\begin{verbatim}
   ai -au Foo.cc
\end{verbatim}

\noindent
would check in \file{Foo.cc} from the current directory into the corresponding
\aipspp\ master repository, updating the slave and plain-text copies in the
process.  The copy of \file{Foo.cc} in the current directory would be deleted.
The command

\begin{verbatim}
   ai -u *.c
\end{verbatim}

\noindent
would check in all \file{.c} files in the current directory and leave
read-only copies behind.

\subsection*{Bugs}

\begin{itemize}
\item
   If the network fails part-way through a checkin, or if the checkout is
   aborted before completion, an \rcs\ semaphore file with a name of the form
   \file{,*,} may be left behind in the master \rcs\ repository.  The presence
   of this empty file will cause subsequent checkins or checkouts to report an
   ``in use'' error.  If the not deleted manually by using the \exeref{ae}
   command, or \exe{ai~-ae}, the semaphore file will eventually be cleaned out
   by \exeref{exhale} which deletes all such files which are more
   than one day old.

\item
   Since \exe{ai} operates over the internet the file transfers are subject to
   network corruption.  On a several occasions, usually when the network
   response was unusually poor, truncation of the last few characters of a
   file has been observed.  A utility, \exeref{xrcs}, is available
   to verify the internal consistency of \rcs\ version files.

\item
   \exe{ai} invokes \rcs\ commands directly on the master \rcs\ repository,
   which may be \textsc{nfs} mounted from a machine on the other side of a
   slow internet link.  A checkin involves reading the \rcs\ version file from
   the master, inserting the new revision, then writing it back.  These file
   transactions are handled more efficiently by \exeref{rai}.

\item
   \exe{ai} requires that the host operating system support setuid shell
   scripts.  Some operating systems, most notably Linux and Digital
   UNIX, do not support setuid shell scripts for security reasons.
\end{itemize}

\subsection*{See also}

The manual page for \unixexe{ci}(1), the \rcs\ checkin command.\\
The unix manual page for \unixexe{chmod}(1).\\
\aipspp\ variable names (\sref{variables}).\\
\aipspp\ code management configuration (\sref{RCS directories}).\\
\exeref{ae}, \aipspp\ \rcs\ semaphore file deletion utility.\\
\exeref{ao}, \aipspp\ code checkout utility.\\
\exeref{am}, \aipspp\ utility to change descriptive text.\\
\exeref{amv}, \aipspp\ code rename utility.\\
\exeref{asme}, execute a command with the effective uid set to the real uid.\\
\exeref{au}, \aipspp\ code update utility.\\
\exeref{av}, \aipspp\ filename validation utility.\\
\exeref{ax}, \aipspp\ code deletion utility.\\
\exeref{mktree}, create \aipspp\ directory hierarchy.\\
\exeref{rai}, invoke \exe{ai} remotely via \unixexe{ssh}(1).\\
\exeref{xrcs}, verify the internal consistency of \rcs\ version files.

\subsection*{Author}

Original: 1992/04/28 by Mark Calabretta, ATNF.

% ----------------------------------------------------------------------------

\newpage
\section{\exe{ao}}
\label{ao}
\index{ao@\exe{ao}}
\index{master host}
\index{co@\exe{co}|see{\exe{ao}}}
\index{code!management!checkout|see{\exe{ao}}}
\index{file!checkout|see{\exe{ao}}}

\aipspp\ source code checkout utility.

\subsection*{Synopsis}

\begin{synopsis}
   \code{\exe{ao} [\textrm{options for} \unixexe{co}] file1 [file2 ...]}
\end{synopsis}

\subsection*{Description}

\exe{ao} checks out source files from the \aipspp\ master \rcs\ repository
into the current directory.  It is a front-end for the \rcs\ \unixexe{co}
command.

\exe{ao} assumes that you are working within a private directory tree which
shadows that of the \aipspp\ master source code tree, see \exeref{mktree}.
The filename arguments specified to \exe{ao} must be simple filenames without
path except that an initial \file{RCS/} will be explicitly stripped off, as
will any \rcs\ \file{,v} suffix.  \exe{ao} determines the corresponding master
directory from the current working directory.

\exe{ao} checks out the source file by executing \file{\$MSTRETCD/ao\_master},
where \file{\$AIPSMSTR} is expected to be \textsc{nfs} mounted from the
\aipspp\ master host (see \sref{RCS directories}).  Programmers who have an
account on the \aipspp\ master host, \host{aips2.nrao.edu}, may instead use a
related utility, \exeref{rao}, which invokes \exe{ao} remotely on the master
host via \unixexe{ssh}.

\subsection*{Options}

\exe{ao} does not itself interpret any options but does recognize several
which it passes to \unixexe{co}.

\begin{description}
\item[\code{\exe{-I} | \exe{-l} | \exe{-r}\# | \exe{-u}\#}]
   These options are passed directly to \unixexe{co}, check the manual page
   for \unixexe{co} for an explanation.  Note that no version number may be
   specified for the \exe{-l} option, and that a \exe{-r} option with revision
   number may not be used together with \exe{-l}.  All other \unixexe{co}
   options are defeated.

   Note that the \exe{-u} option may be used to unlock a checked out file
   without replacement.
\end{description}

\subsection*{Notes}

\begin{itemize}
\item
   The \rcs\ \unixexe{co} command must have write permission on the version
   file in order to check it out of the master repository.  The file ownership
   is defined by the owner of \file{\$MSTRETCD/ao\_master} (\acct{aips2adm}).
   This script must have the \code{setuid} bit set (see \unixexe{chmod}).

\item
   While \exeref{mktree} creates \rcs\ symbolic links in the programmer
   directory hierarchy, these are not used by \exe{ao}.  Their main function
   is to allow \exeref{gmake} to check out the \file{makefile} for a directory
   automatically.  They also allow \rcs\ utilities such as \exeref{alog}
   and \unixexe{rlog} to be applied readily to the slave \rcs\ repository.
   \exe{ao} does not make use of these \rcs\ symbolic links but deals directly
   with the \aipspp\ master repository by specifying absolute pathnames.

\item
   Checkouts may be disabled by creating a file \file{\$MSTRETCD/ai\_disable}.
   If this file contains any text it will be printed as an informative
   message.
\end{itemize}

\subsection*{Diagnostics}

Status return values
\\ \verb+   0+: success
\\ \verb+   1+: initialization error, no checkout
\\ \verb+   2+: checkout failure

\subsection*{Examples}

The command

\begin{verbatim}
   ao -l Foo.cc
\end{verbatim}

\noindent
would check out \file{Foo.cc} with an exclusive lock.  Then

\begin{verbatim}
   ao -u Foo.cc
\end{verbatim}

\noindent
would remove the lock without checking the file in or deleting the working
copy.  The command

\begin{verbatim}
   ao -l RCS/*,v
\end{verbatim}

\noindent
would check out everything in the directory (that is, assuming that \file{RCS}
is a symlink into the slave directory, all files currently in the slave
directory will be checked out of the master).

\subsection*{Bugs}

\begin{itemize}
\item
   If the network fails part-way through a checkout, or if the checkout is
   aborted before completion, an \rcs\ semaphore file with a name of the
   form \file{,*,} may be left behind in the master \rcs\ repository.  The
   presence of this empty file will cause subsequent checkins or checkouts to
   report an ``in use'' error.  If the not deleted manually by using the
   \exeref{ae} command, or \aipsexee{ai~-ae}{ai}, the semaphore file will
   eventually be cleaned out by \exeref{exhale} which deletes all such files
   which are more than one day old.

\item
   Since \exe{ao} operates over the internet the file transfers are
   subject to network corruption.  On a several occasions, usually when the
   network response was unusually poor, truncation of the last few characters
   of a file has been observed.  A utility, \exeref{xrcs}, is available to
   verify the internal consistency of \rcs\ version files.

\item
   \exe{ao} invokes \rcs\ commands directly on the master \rcs\ repository,
   which may be \textsc{nfs} mounted from a machine on the other side of a
   slow internet link.  A checkout involves reading the \rcs\ version file from
   the master, inserting the locker's name, then writing it back.  These file
   transfers are handled more efficiently by \exeref{rao}.

\item
   \exe{ao} requires that the host operating system support setuid shell
   scripts.  Some operating systems, most notably Linux and Digital
   UNIX, do not support setuid shell scripts for security reasons.
\end{itemize}

\subsection*{See also}

The manual page for \unixexe{co}(1), the \rcs\ checkout command.\\
The unix manual page for \unixexe{chmod}(1).\\
\aipspp\ variable names (\sref{variables}).\\
\exeref{ae}, \aipspp\ \rcs\ semaphore file deletion utility.\\
\exeref{ai}, \aipspp\ code checkin utility.\\
\exeref{alog}, \aipspp\ change log reporting utility.\\
\exeref{amv}, \aipspp\ code rename utility.\\
\exeref{au}, \aipspp\ code update utility.\\
\exeref{ax}, \aipspp\ code deletion utility.\\
\exeref{mktree}, create \aipspp\ directory hierarchy.\\
\exeref{rao}, invoke \exe{ao} remotely via \unixexe{ssh}(1).

\subsection*{Author}

Original: 1992/04/28 by Mark Calabretta, ATNF.

% ----------------------------------------------------------------------------

\newpage
\section{\exe{alog}}
\label{alog}
\index{alog@\exe{alog}}
\index{master host}
\index{rlog@\exe{rlog}|see{\exe{alog}}}
\index{code!management!change log|see{\exe{alog}}}

\aipspp\ change log reporting utility.

\subsection*{Synopsis}

\begin{synopsis}
   \code{\exe{alog} [\exe{--archive}] \\
      \ \ \ [\exe{--file=}file | \exe{-f} file] \\
      \ \ \ [\exe{--dorescind}] \\
      \ \ \ [\exe{--listall}] \\
      \ \ \ [\exe{--master} | \exe{-m}] \\
      \ \ \ [\exe{--package}=package | \exe{-p} package] \\
      \ \ \ [\exe{--noheader}] \\
      \ \ \ [\textrm{options for} \unixexe{rlog}] \\
      \ \ \ [file1 [file2 \ldots]]}
\end{synopsis}

\subsection*{Description}
\exe{alog} prints change log messages for \aipspp\ source files.  It is a
front-end to the \rcs\ \unixexe{rlog} command.

If any files are specified then \exe{alog} simply runs \unixexe{rlog} on them
filtering the output to remove \rcs-specific information.  The filenames
may be specified via a relative or absolute pathname, with or without the
\rcs\ \file{,v} suffix.

If no files are specified on the command line, \exe{alog} executes
\unixexe{rlog} for all \aipspp\ sources, or for all sources within a
particular \aipspp\ (see \sref{Code directories}) package as specified by the
\exe{--package} option.  In this instance, apart from filtering the output of
\unixexe{rlog} to remove \rcs-specific information, \exe{alog} also removes
all references to files for which no revisions were selected.  It thus
produces a concise, formatted change log.

\exe{alog} normally consults the \rcs\ repository pointed to by the
\file{\$AIPSROOT/rcs} symbolic link, but can optionally be made to consult the
master or archive if it is available (see \sref{RCS directories}).

\subsection*{Options}

The long form \code{POSIX} style options are provided to help differentiate
options intended for \exe{alog} from those it passes to \unixexe{rlog}.

\begin{description}
\item[\exe{--archive}]
   Consult the archived \rcs\ files.  These include all \rcs\ revisions up to
   the base release of the current major version.  This option can only be
   used in Socorro.

\item[\code{\exe{--file=}file | \exe{-f} file}]
   The specified file, if it exists, contains a timestamp in \exeref{adate}
   format which \exe{alog} converts to a \exe{-d} option for
   \unixexe{rlog}.  The timestamp is then updated.  This is useful for
   generating incremental change logs.

\item[\exe{--dorescind}]
   Include rescinded files in the report (these only exist in the archive and
   master \rcs\ repositories).

\item[\exe{--listall}]
   List all files, the default behaviour is for \exe{alog} to filter out files
   without revisions unless the \exe{-R}, \exe{-h}, or \exe{-t} \unixexe{rlog}
   options are specified.

\item[\code{\exe{--master} | \exe{-m}}]
   Consult the master \rcs\ files.  The \aipspp\ master and slave
   \rcs\ files contain only the current and preceeding major versions.  The
   slave is typically a day to a week behind the master.  This option should
   only be used in Socorro.

\item[\code{\exe{--package=}package | \exe{-p} package}]
   Produce a report for the specified \aipspp\ package only.

\item[\exe{--noheader}]
   Don't add a separate header for each package.
\end{description}

Other options are passed directly to \unixexe{rlog} without verification,
check the manual page for \unixexe{rlog} for an explanation.  In particular,
the \exe{-d} option may be used to specify a range of dates, and \exe{-w} to
specify a list of users.

\subsection*{Notes}

\begin{itemize}
\item
   If file name arguments are specified, \exe{alog} will use the \rcs\ 
   symbolic links in the programmer directory created by \exeref{mktree}.
\end{itemize}

\subsection*{Diagnostics}

Status return values
\\ \verb+   0+: success
\\ \verb+   1+: initialization error

\subsection*{Examples}

The following \unixexe{cron} job is executed weekly on the \aipspp\ master
and the result e-mailed to interested parties:

\begin{verbatim}
   # Report files currently checked out.
   00 21 * * 0 (. $HOME/.profile ; alog -m -L -t) 2>&1 | \
      mail -s "AIPS++ sources currently checked out" aips2-workers
\end{verbatim}

\noindent
The \exe{-L} and \exe{-t} options are recognized by \unixexe{rlog}; \exe{-L}
causes it to ignore \rcs\ files which have no locks set, and \exe{-t} causes
it to print the header information and descriptive text only (that is,
basically everything except the change log entries).  Much of the header
information is then discarded by \exe{alog}.

\subsection*{See also}

The unix manual page for \unixexe{cron}(1).\\
The manual page for \unixexe{rlog}(1), the \rcs\ log command.\\
\aipspp\ variable names (\sref{variables}).\\
\aipspp\ code management configuration (\sref{RCS directories}).\\
\exeref{adate}, \aipspp\ time reporting utility.\\
\exeref{mktree}, create \aipspp\ directory hierarchy.

\subsection*{Author}

Original: 1993/01/22 by Mark Calabretta, ATNF.

% ----------------------------------------------------------------------------

\newpage
\section{\exe{am}}
\label{am}
\index{am@\exe{am}}
\index{master host}
\index{code!management!file description|see{\exe{am}}}
\index{file!change description|see{\exe{am}}}

Utility to change the descriptive text in an \aipspp\ \rcs\ version
file.

\subsection*{Synopsis}

\begin{synopsis}
   \code{\exe{am} [\exe{-au}] file1 [file2 ...]}
\end{synopsis}

\subsection*{Description}

\exe{am} changes the descriptive text in an existing \aipspp\ \rcs\ version
file or files.  It is a front-end for the \rcs\ \unixexe{rcs} command with
option \exe{-t}.  The descriptive text will be obtained from \file{stdin}.

\exe{am} assumes that you are working within a private directory tree which
shadows that of the \aipspp\ master source code tree, see \exeref{mktree}.
The filename arguments specified to \exe{am} must be simple filenames without
path or \rcs\ (\file{,v}) suffix.  \exe{am} determines the corresponding
master directory from the current working directory.

\exe{am} modifies the master \rcs\ version file by executing
\file{\$MSTRETCD/am\_master}, where \file{\$AIPSMSTR} is expected to be
\textsc{nfs} mounted from the \aipspp\ master host (see
\sref{RCS directories}).

\subsection*{Options}

\begin{description}
\item[\exe{-au}]
   After modifying the master \rcs\ version file, \exe{am} copies it back to
   the local slave \rcs\ repository (if it exists), giving it a file
   ownership defined by the owner of \exe{am} (\acct{aips2mgr} at the local
   site).  \exe{am} itself must be setuid to this account.
\end{description}

\subsection*{Notes}

\begin{itemize}
\item
   The \rcs\ \unixexe{rcs} command must have write permission on the version
   file in order to modify it.  The file ownership is defined by the owner of
   \file{\$MSTRETCD/am\_master} (\acct{aips2adm}).  This script must have the
   \code{setuid} bit set (see \unixexe{chmod}).

\item
   While \exeref{mktree} creates \rcs\ symbolic links in the programmer
   directory hierarchy, these are not used by \exe{am}.  Their main function
   is to allow \exeref{gmake} to automatically check out the \file{makefile}
   for a directory.  They also allow \rcs\ utilities such as \exeref{alog} and
   \unixexe{rlog} to be applied readily to the slave \rcs\ repository.
   \exe{am} does not make use of these \rcs\ symbolic links but deals directly
   with the \aipspp\ master repository by specifying absolute pathnames.

\item
   Modifications are disabled if \file{\$MSTRETCD/ai\_disable} exists, see
   \exeref{ai}.  If this file contains any text it will be printed as an
   informative message.
\end{itemize}

\subsection*{Diagnostics}

Status return values
\\ \verb+   0+: success
\\ \verb+   1+: initialization error
\\ \verb+   2+: failed to update the master
\\ \verb+   3+: updated master, but slave not updated

\subsection*{Examples}

The command

\begin{verbatim}
   cd $HOME/aips++/code/install/codemgmt
   am -au am
\end{verbatim}

\noindent
would allow the descriptive text for \exe{am} to be changed interactively.

\subsection*{Bugs}

\begin{itemize}
\item
   \exe{am} requires that the host operating system support setuid shell
   scripts.  Some operating systems, most notably Linux and Digital
   UNIX, do not support setuid shell scripts for security reasons.
\end{itemize}

\subsection*{See also}

The unix manual page for \unixexe{chmod}(1).\\
\aipspp\ variable names (\sref{variables}).\\
\exeref{ae}, \aipspp\ \rcs\ semaphore file deletion utility.\\
\exeref{ai}, \aipspp\ code checkin utility.\\
\exeref{alog}, \aipspp\ change log reporting utility.\\
\exeref{amv}, \aipspp\ code rename utility.\\
\exeref{au}, \aipspp\ code update utility.\\
\exeref{ax}, \aipspp\ code deletion utility.\\
\exeref{mktree}, create \aipspp\ directory hierarchy.

\subsection*{Author}

Original: 1995/03/17 by Mark Calabretta, ATNF.

% ----------------------------------------------------------------------------

\newpage
\section{\exe{amv}}
\label{amv}
\index{amv@\exe{amv}}
\index{master host}
\index{mv@\exe{mv}|see{\exe{amv}}}
\index{code!management!file renaming|see{\exe{amv}}}

\aipspp\ file and directory renaming utility.

\subsection*{Synopsis}

\begin{synopsis}
   \code{\exe{amv} [ \exe{-src} | \exe{-top} ] old new}
\end{synopsis}

\subsection*{Description}

\exe{amv} is used to rename \aipspp\ files and directories.  It does not do
any renaming itself but instead appends renaming commands to the action list
contained within \exereff{ax\_master}{ax}.  It then invokes
\aipsexee{ax\_master}{ax} on the master \aipspp\ \rcs\ repository which must
be available (see \sref{RCS directories}).  The renames will be applied to the
slave \rcs\ repositories and system by \exeref{inhale} and \exeref{sneeze}
respectively.  This mechanism allows \aipspp\ slave sites to follow changes
made to the master.

General users can rename directories at or below the third level below
\code{\$AIPSCODE} (see \sref{Code directories}).  In particular, this includes
module and application subdirectories.  Renaming of higher-level directories
is only allowed for members of the \acct{aips2mgr} group.  

\subsection*{Options}

\begin{description}
\item[\exe{-src}]
   Rename files or directories in the \aipspp\ \rcs\ repository and code
   subtree.  Both arguments must be specified as a pathname relative to
   \file{\$AIPSCODE} (and therefore \file{\$AIPSLAVE} and \file{\$AIPSMSTR}.
   For example,

\begin{verbatim}
   amv -src aips/implement/Foo.cc vlbi/implement/Bar.cc
\end{verbatim}

   \noindent
   would result in the following renames:

\begin{verbatim}
   $AIPSMSTR/aips/implement/Foo.cc,v -> $AIPSMSTR/vlbi/implement/Bar.cc,v
   $AIPSLAVE/aips/implement/Foo.cc,v -> $AIPSLAVE/vlbi/implement/Bar.cc,v
   $AIPSCODE/aips/implement/Foo.cc   -> $AIPSCODE/vlbi/implement/Bar.cc
\end{verbatim}

   \noindent
   the first when \exereff{ax\_master}{ax} is invoked by \exe{amv} itself, and
   the other two when invoked by \exeref{inhale}.  The following deletions
   will also result when \aipsexee{ax\_master}{ax} is invoked by
   \exeref{sneeze}:

\begin{verbatim}
   $AIPSARCH/lib/libaips.a(Foo.o)
   $AIPSARCH/libdbg/libaips.a(Foo.o)
\end{verbatim}

\item[\exe{-top}]
   Rename a file or directory beneath \file{\$AIPSROOT}.  Both arguments must
   be specified as a pathname relative to \file{\$AIPSROOT}.  For example,

\begin{verbatim}
   amv -top docs/README docs/ReadMe
\end{verbatim}

   \noindent
   would result in renaming

\begin{verbatim}
   $AIPSROOT/docs/README  -> $AIPSROOT/docs/ReadMe
\end{verbatim}

   \noindent
   when \exeref{sneeze} invokes \exereff{ax\_master}{ax}.  Note the following:

\begin{verbatim}
   amv -top master/old master/new
   amv -top slave/old slave/new
   amv -top rcs/old rcs/new
\end{verbatim}

   \noindent
   The first only renames the master \rcs\ version file, the second only
   renames the slave \rcs\ version file, and by convention the third
   renames both.
\end{description}

\subsection*{Resources}

\begin{itemize}
\item
   \code{account.manager}: Account name and group of the owner of the
   \aipspp\ source code.  \exe{amv} only allows members of the \acct{aips2mgr}
   group to delete higher-level directories.  \file{\$HOME/.aipsrc} is
   ignored.
\end{itemize}

\subsection*{Notes}

\begin{itemize}
\item
   \exe{amv} uses \exeref{av} to verify new filenames and creates addenda to
   the \aipsexe{av} filename cache for them.

\item
   It is an error for the new name to refer to a file or directory which
   already exists - use \exeref{ax} to delete any such pre-existing file or
   directory if necessary.

\item
   It is not permitted to rename a file or directory in the \aipspp\ system
   areas.  This must be done by using the \exe{-sys} option to \exeref{ax} to
   delete the file or directory and relying on a system rebuild to recreate
   any missing files.

\item
   When \exereff{ax\_master}{ax} is invoked on the master \rcs\ repositories
   it uses a temporary action list placed in \file{MSTRETCD/ax\_list} by
   \exe{amv}.  This mechanism prevents the same renames being applied by
   \aipsexee{ax\_master}{ax} twice in the same version of \aipspp.

\item
   Renames may be disabled by creating a file \file{\$MSTRETCD/amv\_disable}.
   If this file contains any text it will be printed as an informative
   message.

\item
   \exe{amv} must be setuid to \acct{aips2mgr}.
\end{itemize}

\subsection*{Diagnostics}

Status return values
\\ \verb+   0+:  success
\\ \verb+   1+:  initialization error
\\ \verb+   2+:  usage error
\\ \verb+   3+:  failed to get \aipsexee{ax\_master}{ax}

\subsection*{Examples}

To move a module subdirectory from the \code{trial} package (see \sref{Code
directories}) to the \code{aips} package use

\begin{verbatim}
   amv -src trial/implement/Module aips/implement/Module
\end{verbatim}

\noindent
It may be convenient in \unixexe{csh} and some other shells which allow it to
use a command of the form

\begin{verbatim}
   amv -src aips/implement\{,/FooMan}/Chu.cc
\end{verbatim}

\noindent
The shell would expand this to

\begin{verbatim}
   amv -src aips/implement/Chu.cc aips/implement/FooMan/Chu.cc
\end{verbatim}

\noindent
Refer also to the examples provided in the ``Options'' section above.

\subsection*{Bugs}

\begin{itemize}
\item
   \exe{amv} requires that the host operating system support setuid
   shell scripts.  Some operating systems, most notably Linux and
   Digital UNIX, do not support setuid shell scripts for security
   reasons.
\end{itemize}

\subsection*{See also}

\aipspp\ code management configuration (\sref{RCS directories}).\\
\aipspp\ variable names (\sref{variables}).\\
\exeref{inhale}, \aipspp\ code import utility.\\
\exeref{sneeze}, \aipspp\ system rebuild utility.\\
\exeref{av}, \aipspp\ filename validation utility.\\
\exeref{ax}, \aipspp\ code deletion utility.

\subsection*{Author}

Original: 1993/07/23 by Mark Calabretta, ATNF.

% ----------------------------------------------------------------------------
 
\newpage
\section{\exe{asme}}
\label{asme}
\index{asme@\exe{asme}}
\index{setuid|see{\exe{asme}}}
 
Execute a command with the effective uid set to the real uid.

\subsection*{Synopsis}
 
\begin{synopsis}
   \code{\exe{asme} [command]}
\end{synopsis}
 
\subsection*{Description}
 
\exe{asme} executes a command with the effective uid set to the real uid.  The
requirement for such a utility arises because Bourne shell offers no mechanism
within a \code{setuid} script to allow a command to be executed with the real
uid.
 
\subsection*{Options}
 
None.
 
\subsection*{Notes}
 
\begin{itemize}
\item
   The command may be omitted.
\end{itemize}
 
\subsection*{Diagnostics}
 
Status return values
\\ \verb+   -1+:  fork failed
\\ \verb+    0+:  success, value returned on \file{stdout}
\\ \verb+ else+:  exit status returned by the command
 
\subsection*{Examples}
 
\exeref{ai} runs \code{setuid} to \acct{aips2mgr}.  It uses the following
construct to ensure that the file to be checked in is readable by user and
group:

\begin{verbatim}
   asme "chmod ug+r $i"
\end{verbatim}
 
 
\subsection*{See also}
 
The unix manual page for \unixexe{chmod}(1).\\
\exeref{ai} \aipspp\ code checkin utility.
 
\subsection*{Author}
 
Original: 1996/03/28 by Mark Calabretta, ATNF.

% ----------------------------------------------------------------------------

\newpage
\section{\exe{au}}
\label{au}
\index{au@\exe{au}}
\index{master host}
\index{code!management!update|see{\exe{au}}}

\aipspp\ source code update utility.

\subsection*{Synopsis}

\begin{synopsis}
   \code{\exe{au} file1 [file2 ...]}
\end{synopsis}

\subsection*{Description}

\exe{au} updates the \aipspp\ \file{slave} and \file{code} areas with the
lastest version of a file in \file{\$AIPSMSTR}.

\exe{au} assumes that you are working within a private directory tree which
shadows that of the \aipspp\ master source code tree, see \exeref{mktree}.
The filename arguments specified to \exe{au} must be simple filenames without
path or \rcs\ \file{,v} suffix.  \exe{au} determines the corresponding master
directory from the current working directory.

\exe{au} expects the master \rcs\ repository, \file{\$AIPSMSTR}, to be
\textsc{nfs} mounted from the \aipspp\ master host (see
\sref{RCS directories}).  It copies the \rcs\ version file from the master to
the slave \rcs\ repository, giving it a file ownership defined by the owner of
\exe{au} (aips2mgr at the local site).  \exe{au} must itself have the
\code{setuid} bit set (see \unixexe{chmod}).

If successful, \exe{au} then checks out a plain-text copy of the code from the
\file{\$AIPSRCS} subdirectory into the corresponding \file{\$AIPSCODE}
subdirectory.

Newly created directories on the master may also be propagated to the local
installation.

Programmers who have an account on the \aipspp\ master host,
\host{aips2.nrao.edu}, may instead use a related utility, \exeref{rau}, which
uses \unixexe{ssh} to transfer the \rcs\ version files in compressed form.

Programmers on a Linux or Digital UNIX platform, or on any other
platform that does not support setuid shell scripts, should instead use
a related utility, \exeref{sau}.  \exe{sau} runs a non-setuid copy of
\exe{au}, \exe{sau.sh}, via a C-based setuid wrapper program.

\subsection*{Options}

None.

\subsection*{Diagnostics}

Status return values
\\ \verb+   0+: success
\\ \verb+   1+: initialization error
\\ \verb+   2+: master file not found
\\ \verb+   3+: slave not updated
\\ \verb+   4+: code not updated

\subsection*{Examples}

The command

\begin{verbatim}
   cd $HOME/aips++/code/install
   au au
\end{verbatim}

\noindent
would update the slave and plain-text copies of \exe{au}, but not the script
in the \aipspp\ \file{bin} area.

\subsection*{Bugs}
 
\begin{itemize}
\item
   \exe{au} copies \rcs\ version files directly from the master \rcs\
   repository, which may be \textsc{nfs} mounted from a machine on the other
   side of a slow internet link.  The copy is handled more efficiently by
   \exeref{rau} which compresses the files for transfer.

\item
   \exe{au} requires that the host operating system support setuid shell
   scripts.  Some operating systems, most notably Linux and Digital
   UNIX, do not support setuid shell scripts for security reasons.
\end{itemize}

\subsection*{See also}

The unix manual page for \unixexe{chmod}(1).\\
\aipspp\ variable names (\sref{variables}).\\
\exeref{ai}, \aipspp\ code checkin utility.\\
\exeref{amv}, \aipspp\ code rename utility.\\
\exeref{ao}, \aipspp\ code checkout utility.\\
\exeref{ax}, \aipspp\ code deletion utility.\\
\exeref{rau}, update \aipspp\ sources from the master via \unixexe{ssh}(1).\\
\exeref{sau}, \aipspp\ code update utility.\\
\exeref{mktree}, create \aipspp\ directory hierarchy.

\subsection*{Author}

Original: 1992/04/30 by Mark Calabretta, ATNF.

% ----------------------------------------------------------------------------

\newpage
\section{\exe{av}}
\label{av}
\index{av@\exe{av}}
\index{master host}
\index{code!management!filename validation|see{\exe{av}}}
\index{file!name validation|see{\exe{av}}}
\index{validation|see{verification}}
\index{verification!filename|see{\exe{av}}}

\aipspp\ filename validation utility.

\subsection*{Synopsis}

\begin{synopsis}
   \code{\exe{av} filename1 [filename2,...]}
\end{synopsis}

\subsection*{Description}

\exe{av} validates an \aipspp\ class implementation filename against those
already in the system to ensure uniqueness in the first 15 characters.  It
consults a cache of \aipspp\ pathnames maintained by other \aipspp\ 
code management utilities (\exeref{ai}, \exeref{amv}, \exeref{exhale}).  Any
matches found are printed on stdout.

\subsection*{Options}

None.

\subsection*{Notes}

\begin{itemize}
\item
   The pathname cache is stored in compressed form using \unixexe{gzip} to
   minimize the amount of data which must be fetched from the master during a
   checkin or rename.

\item
   If several filenames are to be validated it is more efficient to run
   \exe{av} once with multiple filename arguments than to run it separately
   for each filename.
\end{itemize}

\subsection*{Diagnostics}

Status return values
\\ \verb+   0+: success
\\ \verb+   1+: initialization error
\\ \verb+   2+: cache file not accessible

\subsection*{Examples}

The command

\begin{verbatim}
   av BaseMappedArrayIter
\end{verbatim}

\noindent
would print the string

\begin{verbatim}
   aips/implement/Tables/BaseMappedArrayEngine
\end{verbatim}

\noindent
indicating that \file{BaseMappedArrayIter} would not be acceptable as a new
file name because the first 15 characters conflict with
\file{BaseMappedArrayEngine}.

\subsection*{Files}

\begin{description}
\item[\file{\$MSTRETCD/av\_cache.gz}]
...pathname cache.
\end{description}

\subsection*{See also}

The GNU manual page for \unixexe{gzip}(1).\\
\aipspp\ variable names (\sref{variables}).\\
\exeref{ai}, \aipspp\ code checkin utility.\\
\exeref{amv}, \aipspp\ code rename utility.\\
\exeref{exhale}, \aipspp\ code export utility.

\subsection*{Author}

Original: 1994/11/11 by Mark Calabretta, ATNF.

% ----------------------------------------------------------------------------

\newpage
\section{\exe{avers}}
\label{avers}
\index{avers@\exe{avers}}
\index{master host}
\index{code!management!version|see{\exe{avers}}}

\aipspp\ version reporting utility.

\subsection*{Synopsis}

\begin{synopsis}
   \code{\exe{avers} [\exe{-b} | \exe{-l}]}
\end{synopsis}

\subsection*{Description}

\exe{avers} reports version information for an \aipspp\ installation.

The \aipspp\ version consists of three fixed-length fields of the form
\code{MM.NNN.PP}.  The first field of two digits is the major version number,
the second field of three digits is the minor version number, and the third
field of two digits is the patch level.

The major version number is incremented whenever a new base release is
produced.  A base release is the publically distributed production code.
The minor version number is incremented on each cycle of the code distribution
system for \aipspp\ development sites (see \exeref{exhale} and
\exeref{inhale}), and the patch version distinguishes different patch levels
applied to the base release for the public distribution.

The long report contains the following version and timing information:

\noindent
\verb+   Master: +Version last fetched by \exeref{inhale}.\\
\verb+    Slave: +Version and time of completion of last slave update.\\
\verb+     Code: +Version and time of completion of last code update.\\
\verb+   System: +Version and time of completion of last system rebuild.\\
\verb+   Latest: +Version currently available from the master.\\
\verb+     Time: +The current time, in \aipspp\ format (see \exeref{adate}).

\noindent
Each of the installation steps is tagged as being {\em base},
{\em incremental}, or {\em cumulative}.  The short report only includes the
version last fetched by \exeref{inhale}, and without the ``\code{Master:}''
prefix.

\subsection*{Options}

\begin{description}
\item[\exe{-b}]
   Report the version last fetched by \aipsexe{inhale} (default).

\item[\exe{-l}]
   Report the version last fetched by \aipsexe{inhale}, the time at the
   completion of each stage of its installation, the latest version available
   from the master, and the current time.
\end{description}

\subsection*{Diagnostics}

Status return values
\\ \verb+   0+: success
\\ \verb+   1+: invalid option
\\ \verb+   2+: \code{\$AIPSPATH} not defined

\subsection*{Examples}

Sample output from the command

\begin{verbatim}
   avers -l
\end{verbatim}

\noindent
appears as follows

\begin{verbatim}
   Master: 01.021.00 Wed 1992/09/16 10:30:32 GMT
    Slave: 01.021.00 Wed 1992/09/16 12:01:04 GMT  (incremental)
     Code: 01.021.00 Wed 1992/09/16 12:03:08 GMT  (incremental)
   System: 01.021.00 Wed 1992/09/16 12:07:55 GMT  (incremental)
   Latest: 01.022.00 Wed 1992/09/16 22:30:38 GMT
     Time:           Thu 1992/09/17 04:10:42 GMT
\end{verbatim}

\subsection*{See also}

\exeref{adate}, \aipspp\ time reporting utility.\\
\exeref{exhale}, \aipspp\ code export utility.\\
\exeref{inhale}, \aipspp\ code import utility.

\subsection*{Author}

Original: 1992/09/17 by Mark Calabretta, ATNF.

% ----------------------------------------------------------------------------

\newpage
\section{\exe{ax}}
\label{ax}
\index{ax@\exe{ax}}
\index{master host}
\index{ax\_master@\exe{ax\_master}|see{\exe{ax}}}
\index{code!management!deletion|see{\exe{ax}}}
\index{directory!delete|see{\exe{ax}}}
\index{file!delete|see{\exe{ax}}}
\index{rm@\exe{rm}|see{\exe{ax}}}
\index{rmdir@\exe{rmdir}|see{\exe{ax}}}

\aipspp\ deletion utility.

\subsection*{Synopsis}

\begin{synopsis}
   \code{\exe{ax} [ \exe{-src} | \exe{-sys} | \exe{-top} ] target1
      [target2 ...] [...]}
\end{synopsis}

\subsection*{Description}

\exe{ax} is used to delete \aipspp\ files and directories.  It does not delete
them itself but instead appends deletion commands to the action list contained
within \exe{ax\_master} (see below).  It then invokes \exe{ax\_master} on the
master \aipspp\ \rcs\ repository which must be available (see
\sref{RCS directories}).  The deletions will be applied to the slave \rcs\ 
repositories and system by \exeref{inhale} and \exeref{sneeze} respectively.
This mechanism allows \aipspp\ slave sites to follow changes made to the
master.
 
General users can delete directories at or below the third level below
\code{\$AIPSCODE} (see \sref{Code directories}).  In particular, this includes
module and application subdirectories.  Deletion of higher-level directories
is only allowed for members of the \acct{aips2mgr} group.

\exe{ax\_master} does not actually delete anything from the \aipspp\ master
\rcs\ repositories but instead ``rescinds'' them by renaming them so as to
hide them from the code distribution system, see \exeref{exhale}.  Rescinded
files and directories are given names of the form \file{.MM.NNN.filename},
where \file{MM.NNN} is the \aipspp\ version number at the time the file is
rescinded (see \exeref{avers}).

\subsection*{Options}

\begin{description}
\item[\exe{-src}]
   Delete a pair of files from the \aipspp\ \rcs\ repository and code subtree.
   The target must be specified as a pathname relative to \file{\$AIPSCODE}
   (and therefore \file{\$AIPSLAVE} and \file{\$AIPSMSTR}, and need not have
   an \rcs\ \file{,v} suffix.  For example,

\begin{verbatim}
   ax -src aips/apps/App1/App1.cc
\end{verbatim}

   \noindent
   would cause the following files to be deleted

\begin{verbatim}
   $AIPSMSTR/aips/apps/App1/App1.cc,v
   $AIPSLAVE/aips/apps/App1/App1.cc,v
   $AIPSCODE/aips/apps/App1/App1.cc
\end{verbatim}

   \noindent
   the first is rescinded when \exe{ax\_master} is invoked by \exe{ax} itself,
   and the other two are deleted when \exe{ax\_master} is invoked by
   \exeref{inhale}.  Note, however, that any executables associated with an
   application will not be deleted.  These may be deleted explicitly by using
   the \exe{-sys} option (see below).

   \noindent
   If the target specified is a class implementation file, the object module
   will be deleted from the relevant library when \exe{ax\_master} is invoked
   by \exeref{sneeze}.  For example,

\begin{verbatim}
   ax -src aips/implement/Class.cc
\end{verbatim}

   \noindent
   would result in the following deletions

\begin{verbatim}
   $AIPSMSTR/aips/implement/Class.cc,v
   $AIPSLAVE/aips/implement/Class.cc,v
   $AIPSCODE/aips/implement/Class.cc
   $AIPSARCH/lib/libaips.a(Class.o)
   $AIPSARCH/libdbg/libaips.a(Class.o)
\end{verbatim}

   \noindent
   If necessary, \exe{ax\_master} will also \unixexe{ranlib} the object
   libraries and rebuild any sharable objects.

   \noindent
   Directories specified as targets will be rescinded from the master and
   recursively deleted from the slave and code areas.

\item[\exe{-sys}]
   Delete a file from the \aipspp\ system area.  The target must be
   specified as a pathname relative to \file{\$AIPSARCH} - the master, slave,
   and code areas are not affected.  For example,

\begin{verbatim}
   ax -sys bin/Application bindbg/Application
\end{verbatim}

   \noindent
   would mark

\begin{verbatim}
   $AIPSARCH/bin/Application
\end{verbatim}

   and

\begin{verbatim}
   $AIPSARCH/bindbg/Application
\end{verbatim}

   \noindent
   for deletion by \exe{ax\_master} when invoked by \exeref{sneeze}.

   \noindent
   Targets specified in the form \code{x(y)} denote library modules.  For
   example,

\begin{verbatim}
   ax -sys lib/libaips.a(Class.o)
\end{verbatim}

   \noindent
   would mark \file{Class.o} for deletion from

\begin{verbatim}
   $AIPSARCH/lib/libaips.a
\end{verbatim}

   {\em and}

\begin{verbatim}
   $AIPSARCH/libdbg/libaips.a
\end{verbatim}

   \noindent
   by \exe{ax\_master} when invoked by \exeref{sneeze}.  If necessary,
   \exe{ax\_master} would then \unixexe{ranlib} the object libraries and
   rebuild any sharable objects.

   \noindent
   A directory may be specified as the target, in which case it will be
   recursively deleted.

\item[\exe{-top}]
   Delete a file beneath \file{\$AIPSROOT}.  The target must be specified as a
   pathname relative to \file{\$AIPSROOT}.  This is a catch-all target and
   should only be used when the \exe{-src} and \exe{-sys} options are
   inappropriate.

   If the target refers to a file in the master repository then it will be
   rescinded when \exe{ax\_master} is invoked by \exe{ax}.  If it refers to a
   file in the slave or code areas then it will be deleted when
   \exe{ax\_master} is invoked by \exeref{inhale}.  If it refers to a file in
   the system area then it will be deleted when \exe{ax\_master} is invoked by
   \exeref{sneeze} running on a machine of the corresponding architecture.

   For example,

\begin{verbatim}
   ax -top docs/reference/Example.ps
\end{verbatim}

   \noindent
   would mark

\begin{verbatim}
   $AIPSROOT/docs/reference/Example.ps
\end{verbatim}

   \noindent
   for deletion by \exe{ax\_master} when invoked by \exeref{inhale}.  Note the
   following:

\begin{verbatim}
   ax -top master/aips/apps/App1/App1.cc,v
   ax -top slave/aips/apps/App1/App1.cc,v
   ax -top rcs/aips/apps/App1/App1.cc,v
\end{verbatim}

   \noindent
   The first rescinds the master \rcs\ version file, the second deletes the
   slave \rcs\ version file, and by convention the third does both.  Note
   that the \rcs\ \file{,v} suffix does need to be specified here.

   \noindent
   A directory may be specified as the target, in which case it will be
   rescinded from the master and recursively deleted from the slave and code
   areas.
\end{description}

\subsection*{Resources}

\begin{itemize}
\item
   \code{account.manager}: Account name and group of the owner of the
   \aipspp\ source code.  \exe{ax} only allows members of the \acct{aips2mgr}
   group to delete higher-level directories.  \file{\$HOME/.aipsrc} is
   ignored.
\end{itemize}

\subsection*{Notes}

\begin{itemize}
\item
   Manual invokation of \exe{ax\_master} is forbidden.  However, it may be of
   interest to know that it has a mode which causes it to apply all entries in
   its action list for the current major \aipspp\ version.  This is used by
   \exeref{inhale} and \exeref{sneeze} immediately before downloading a
   cumulative update.

\item
   When \exe{ax\_master} is invoked on the master \rcs\ repositories it uses a
   temporary action list placed in \file{\$MSTRETCD/ax\_list} by \exe{ax}.
   This mechanism prevents the same file from being rescinded twice from the
   master by \exe{ax\_master} in the same version of \aipspp.

\item
   \exe{ax} may be disabled by creating a file \file{\$MSTRETCD/ax\_disable}.
   If this file contains any text it will be printed as an informative message
   to anyone trying to use \exe{ax}.

   \exe{ax} must be owned by \acct{aips2mgr} and have the \code{setuid} bit
   set.
\end{itemize}

\subsection*{Diagnostics}

Status return values
\\ \verb+   0+: success
\\ \verb+   1+: initialization error
\\ \verb+   2+: failed to get \exe{ax\_master}
\\ \verb+   3+: argument error

\subsection*{Examples}

To recursively delete module subdirectory \file{trial/implement/Junk} use
 
\begin{verbatim}
   ax -src trial/implement/Junk
\end{verbatim}
 
\noindent
Where several files are to be deleted, it may be convenient in \unixexe{csh}
and some other shells which allow it to use a command of the form

\begin{verbatim}
   cd /aips++/code
   ax -src aips/implement/\{Foo,Man,Chu}.*
\end{verbatim}

\noindent
Assuming the files existed, the shell would expand the argument to

\begin{verbatim}
   aips/implement/Foo.cc aips/implement/Foo.h \
   aips/implement/Man.cc aips/implement/Man.h \
   aips/implement/Chu.cc aips/implement/Chu.h \
\end{verbatim}

\noindent
Note that the '\code{cd /aips++/code}' provided the context for the shell's
wildcarding mechanism.

Refer also to the examples provided in the ``Options'' section above.

\subsection*{Bugs}

\begin{itemize}
\item
   \exe{ax} requires that the host operating system support setuid shell
   scripts.  Some operating systems, most notably Linux and Digital
   UNIX, do not support setuid shell scripts for security reasons.
\end{itemize}

\subsection*{See also}

\aipspp\ variable names (\sref{variables}).\\
\exeref{getrc}, query \aipspp\ resource database.\\
\exeref{exhale}, \aipspp\ code export utility.\\
\exeref{inhale}, \aipspp\ code import utility.\\
\exeref{ai}, \aipspp\ code checkin utility.\\
\exeref{amv}, \aipspp\ code rename utility.\\
\exeref{ao}, \aipspp\ code checkout utility.\\
\exeref{au}, \aipspp\ code update utility.\\
\exeref{avers}, \aipspp\ version report utility.

\subsection*{Author}

Original: 1992/05/14 by Mark Calabretta, ATNF.

% ----------------------------------------------------------------------------

\newpage
\section{\exe{buildchangelog}}
\label{buildchangelog}
\index{buildchangelog@\exe{buildchangelog}}
\index{master host}
\index{code!management!changelog|see{\exe{buildchangelog}}}
\index{changelog|see{\exe{buildchangelog}}}

Process \aipspp\ changelog files and transform them to
html files containing the selected changes.

\subsection*{Synopsis}

\begin{synopsis}
   \code{\exe{buildchangelog} [\exe{-dir[ectory]} directory]
     [\exe{-split[file]} area|package|module] [\textrm{select-options}
     <directory>}
\end{synopsis}

\subsection*{Description}

All \aipspp\ changelog files found in the given directory and its
subdirectories are processed and the selected changes
are written to html files. 
Also a few html files containing an overview of the changes are
generated with links to the detailed html files.
\\The given directory has to be a code directory; i.e. it has to
end in \file{/code}.
\\The changes can be selected in various ways using the command
options.
By default the changes in the currently developed release are
selected.
\\The output directory can be specified in the options, which defaults
to the working directory.

\exe{buildchangelog} is run by the \exeref{sneeze} script if the
appropriate \exeref{makedefs} variable is set.
In that case it builds html files for the changes in the currently
developed release.
The \aipspp\ documentation pages have links to the overview files
generated in this way, so at all times the achievements in the
currently developed release can be seen on the \aipspp\ web pages.

One can also use \exe{buildchangelog} to find, for example, all
changes made to glish in the last year.

\subsection*{Options}

\begin{description}
\item[\exe{-dir[ectory]} directory]
   The directory of the generated html files.
   Default is the working directory.

\item[\exe{-split[file]} value]
   This case insensitive option determines how many html files are written.
   \\If not given, all changes are combined in one html file.
   \\\exe{-split area} generates an html file per area
   with name \file{changelog\_<area>.html}.
   \\\exe{- split package} generates an html file per package
   with name
   \\\file{changelog\_<area>\_<package>.html}.
   \\\exe{-split module} generates an html file per module
   with name
   \\\file{changelog\_<area>\_<package>\_<module>.html}.
   \\If the split option is given (and not blank), a summary
   file per area ia generated with name
   \file{changelog\_<area>\_summary.html}.
   \\In any case an overview file is generated with name
   \file{changelog\_summary.html}.

\item[\exe{-date} ranges]
   Select changes matching the given date range(s). Multiple ranges can be
   given separated by commas. Each range is a start date optionally
   followed by a colon and an end date.
   Each date can be given as yyyy-mm-dd or as dd-mm-yyyy, but an 
   end date has to be given in the same order as its start date.
   The month can be given as a number or as a 3 letter case
   insensitive name.
   It is not necessary to give all parts of a date, so it is possible
   to select a given year or range of years (or days or months for
   that matter). Again, an end date and start date have to match
   in that respect.
   \\For example,
\begin{verbatim}
   -date 2000
   -date 1-Jan-2000:31-Dec-2000
   -date 1995,1996,1997,1998,1999,2000
   -date 1995:2000
\end{verbatim}
   The first 2 examples are equivalent; select all of the year 2000.
   \\The latter 2 examples are also equivalent; select all of years
   1995 to 2000 (inclusive).

\item[\exe{-nodate} ranges]
   Select all changes NOT matching the given date range(s).

\item[\exe{-id} ranges]
   Select all changes matching the given id range(s).
   Multiple ranges can be given separated by commas.
   Each range is a start id optionally followed by a colon and an end id.
   \\For example,
\begin{verbatim}
   -id 1:10,20,24:30
\end{verbatim}

\item[\exe{-noid} ranges]
   Select all changes NOT matching the given id range(s).

\item[\exe{-area} names]
   Select all changes matching the given area name(s).
   Multiple areas can be given separated by commas.
   Note that the names are case sensitive.
   \\For example,
\begin{verbatim}
   -area System,Glish,Tool,Library,LearnMore
\end{verbatim}
   This eaxmple shows all available areas.

\item[\exe{-noarea} names]
   Select all changes NOT matching the given area name(s).

\item[\exe{-package} names]
   Select all changes matching the given package name(s).
   Multiple packages can be given separated by commas.
   Note that the names are case sensitive.
   \\For example,
\begin{verbatim}
   -package aips
\end{verbatim}
   Note that only areas Library and Tool have multiple packages.

\item[\exe{-nopackage} packages]
   Select all changes NOT matching the given package(s).

\item[\exe{-module} names]
   Select all changes matching the given module (or tool) name(s).
   Multiple modules can be given separated by commas.
   Note that the names are case sensitive.
   \\For example,
\begin{verbatim}
   -module Tables,Arrays
\end{verbatim}
   Note that the word module is also used for the tool name, so
   a specific tool is selected in the same way.

\item[\exe{-nomodule} names]
   Select all changes NOT matching the given module name(s).

\item[\exe{-avers} ranges]
   Select all changes matching the given \aipspp\ version range(s).
   Multiple ranges can be given separated by commas.
   Each range is a start version optionally followed by a colon and an
   end version. 
   Note that the version is treated as a float number (like 1.4).
   \\By default only the currently developed \aipspp\ release is
   selected. Therefore the special (case insensitive) value
   \exe{all} can be given to indicate that all versions have to be selected.
   \\For example,
\begin{verbatim}
   -avers all
   -avers 1.4:1.9
\end{verbatim}

\item[\exe{-noavers} ranges]
   Select all changes NOT matching the given \aipspp\ version range(s).

\item[\exe{-type} types]
   Select all changes matching the given type(s).
   Multiple types can be given separated by commas.
   Note that the type names are case insensitive.
   \\For example,
\begin{verbatim}
   -type code,documentation,test,review
\end{verbatim}
   This example shows all available types.

\item[\exe{-notype} types]
   Select all changes NOT matching the given type(s).

\item[\exe{-category} categories]
   Select all changes matching the given categories.
   Multiple categories can be given separated by commas.
   Note that the category names are case insensitive.
   \\For example,
\begin{verbatim}
   -category new,change,bugfix,removed,other
\end{verbatim}
   This example shows all available categories.

\item[\exe{-nocategory} categories]
   Select all changes NOT matching the given categories.
\end{description}

\subsection*{Notes}

\begin{itemize}
\item
   The change entries have an area, package, and module component.
   There are a few areas:
   \begin{description}
     \item[Library]
       for all changes in the library code.
     \item[Tool]
       for all changes in the \aipspp\ tools (applications).
     \item[Glish]
       for all changes in glish.
     \item[System]
       for all changes in the system area (directory install).
     \item[LearnMore]
       for all changes in notes, papers, etc. (directory doc).
   \end{description}
\end{itemize}

\subsection*{Diagnostics}

Status return values
\\ \verb+   0+: success
\\ \verb+   1+: invalid option given

\subsection*{Examples}

The command

\begin{verbatim}
   buildchangelog -split module
     -dir $(AIPSROOT)/docs/project/releasenotes/changelogs
     $(AIPSROOT)/code
\end{verbatim}
\noindent
is used in the code makefile to generate the html files
when building the system.

\begin{verbatim}
   buildchangelog -avers all -area Glish /aips++/code
\end{verbatim}
\noindent
can be used to find all changes ever made to glish.
It will generate the files \file{./changelog.html} and
\file{./changelog\_summary.html}.

\subsection*{Bugs}

\begin{itemize}
\item
   \exe{buildchangelog} does not check if a change entry is filled in
   correctly. 
\end{itemize}

\subsection*{See also}

\exeref{ac} \aipspp\ changelog maintenance.

\subsection*{Author}

Original: 2000/06/05 by Ger van Diepen, NFRA

% ----------------------------------------------------------------------------

\newpage
\section{\exe{doover}}
\label{doover}
\index{doover@\exe{doover}}
\index{code!management!version.o@\file{version.o}|see{\exe{doover}}}
\index{version.o@\file{version.o}|see{\exe{doover}}}
 
Generate a function which returns \aipspp\ version information.

\begin{synopsis}
   \code{\exe{doover}}
\end{synopsis}
 
\subsection*{Description}
 
\exe{doover} generates \cplusplus\ code which records version information
found in \file{\$AIPSCODE/VERSION}.  It is intended to be used only by the
\aipspp\ \filref{makefiles}.

The \cplusplus\ code defines the following external variables

\begin{itemize}
\item
   \code{const int aips\_major\_version;} The \aipspp\ major version number
   (see \exeref{avers}).

\item
   \code{const int aips\_minor\_version;} The \aipspp\ minor version number.

\item
   \code{const int aips\_patch\_version;} The \aipspp\ patch version number.

\item
   \code{const char* aips\_version\_date;} The date on which this version was
   produced in \aipspp\ date format (see \exeref{adate}).

\item
   \code{const char* aips\_version\_info;} Any identifying version
   information.
\end{itemize}
 
\noindent
and it also defines the following global function which reports the version
information in a standard format:

\begin{verbatim}
   void report_aips_version(ostream &os)
\end{verbatim}

\noindent
Executables can access these external variables by virtue of declarations
made in \file{aips.h}.

\subsection*{Options}
 
None.
 
\subsection*{Notes}
 
\begin{itemize}
\item
   \filref{makefile.imp} uses \exe{doover} to check and if necessary generate
   \file{\$LIBDBGD/version.o} and \file{\$LIBOPTD/version.o} whenever it
   builds a system object library.  The \file{version.o} object modules are
   dependent on the \file{\$(AIPSCODE)/VERSION} file.  They are not inserted
   into the object library since that would cause all executables to be
   rebuilt.
\end{itemize}
 
\subsection*{Diagnostics}
 
Status return values
\\ \verb+   0+: success
\\ \verb+   1+: initialization error
\\ \verb+   2+: \file{VERSION} file not found
 
\subsection*{Examples}
 
Typing

\begin{verbatim}
   void report_aips_version(ostream &os)
\end{verbatim}

\noindent
shows the code containing the version information derived from the current
\file{\$(AIPSCODE)/VERSION}.
 
\subsection*{See also}
 
\aipspp\ variable names (\sref{variables}).\\
\exeref{adate}, \aipspp\ time reporting utility.\\
\exeref{avers}, \aipspp\ version report utility.\\
\filref{makefiles}, \textsc{gnu} makefiles used to rebuild \aipspp.
 
\subsection*{Author}
 
Original: 1996/02/29 by Mark Calabretta, ATNF.

% ----------------------------------------------------------------------------

\newpage
\section{\exe{mktree}}
\label{mktree}
\index{mktree@\exe{mktree}}
\index{directory!workspace|see{\exe{mktree}}}
\index{workspace!creation|see{\exe{mktree}}}

Create an \aipspp\ directory hierarchy with \rcs\ symlinks.

\subsection*{Synopsis}

\begin{synopsis}
   \code{\exe{mktree} [\exe{-d}] [\exe{-l}] [\exe{-r}] [\exe{-s}]}
\end{synopsis}

\subsection*{Description}

\exe{mktree} is an interactive utility used by \aipspp\ programmers to
create a local workspace.  It works incrementally in that it won't recreate
directories or symbolic links which already exist, but may create new ones
which have appeared since it was first run.

\exe{mktree} creates a directory tree which shadows the \file{\$AIPSCODE}
subdirectory tree (see \sref{Directories}).  In each directory it creates an
\file{RCS} symbolic link to the corresponding \rcs\ directory under
\file{\$AIPSRCS}, thereby creating a window into the local \aipspp\ code
repository.

The main function of the \file{RCS} symlinks is to provide an access route for
\exeref{gmake} to the \aipspp\ \filref{makefiles}.  Although code checkin and
checkout must be done using \exeref{ai} and \exeref{ao} which do not use the
\file{RCS} symlinks, the symlinks do allow other \rcs\ utilities such as
\unixexe{rlog} and \unixexe{rscdiff} to be used as though the \aipspp\ \rcs\ 
repositories existed in the user's own workspace.  \aipsexe{ao} also allows
\code{ao RCS/*,v} to check out everything in a directory.

\exe{mktree} also creates the \file{\$CODEINCD} directory and symbolic links
therein.  These allow include files to be specified in the form
\code{\#include~<aips/Class.h>} for example (see \sref{Code directories}).

\exe{mktree} can be used to create or update a limited portion of the
directory hierarchy.  If the current working directory is a subdirectory of
\file{*/code}, then only that portion of the tree will be created (see the
example below).

\exe{mktree} checks for defunct directories and will delete them and the files
therein after seeking confirmation (unless the \exe{-s} option was specified).

\subsection*{Options}

\begin{description}
\item[\exe{-d}]
   Delete any pre-existing symbolic links.  This can be used to force symlinks
   to be recreated in the uncommon situation where a pre-existing programmer
   tree must be updated for a changed \file{AIPSROOT}, and where the old
   \file{AIPSROOT} still remains.  (Note that if the old \file{AIPSROOT} did
   not still remain then the symlinks would be recreated without needing to
   use this option.)

\item[\exe{-l}]
   \file{RCS} symbolic links are normally created only if \file{\$AIPSRCS} is
   being shadowed (see the \exe{-r} option below), or if the
   \file{\$AIPSCODE/rcs} symlink exists.  This option forces them to be
   created regardless.

\item[\exe{-r}]
   Shadow the \file{\$AIPSRCS} tree rather than \file{\$AIPSCODE}.  This is
   used by \exeref{inhale} to construct or update the \file{\$AIPSCODE} tree
   at consortium sites.

\item[\exe{-s}]
   Don't ask for confirmation of the parent directory, {\em or when deleting
   defunct directories and the files therein}.  Use with caution!
\end{description}

A \exe{-master} option which creates \file{RCS} symlinks directly into
\file{\$AIPSMSTR} is also recognized but is only intended for use by
\exeref{exhale} when building a new public release.

\subsection*{Notes}
 
\begin{itemize}
\item
   \exe{mktree} ignores \file{tmplinst} directories, both in creating the tree
   and when deleting defunct directories.
\end{itemize}

\subsection*{Diagnostics}

Status return values
\\ \verb+   0+: success
\\ \verb+   1+: initialization error

\subsection*{Examples}

To construct an \aipspp\ workspace from scratch or update an existing one

\begin{verbatim}
   mkdir $HOME/aips++
   cd $HOME/aips++
   mktree
\end{verbatim}

\noindent
This is equivalent to

\begin{verbatim}
   mkdir $HOME/aips++/code
   cd $HOME/aips++/code
   mktree
\end{verbatim}

\noindent
To just create or update the \file{aips} package (see \sref{Code directories})

\begin{verbatim}
   cd $HOME/aips++/code/aips
   mktree
\end{verbatim}

\subsection*{See also}

The manual page for \unixexe{rlog}(1).\\
The manual page for \unixexe{rcsdiff}(1).\\
\aipspp\ variable names (\sref{variables}).\\
\exeref{ai}, \aipspp\ code checkin utility.\\
\exeref{alog}, \aipspp\ change log reporting utility.\\
\exeref{amv}, \aipspp\ code rename utility.\\
\exeref{ao}, \aipspp\ code checkout utility.\\
\exeref{au}, \aipspp\ code update utility.\\
\exeref{ax}, \aipspp\ code deletion utility.

\subsection*{Author}

Original: 1992/03/07 by Peter Teuben, BIMA.
 
% ----------------------------------------------------------------------------
 
\newpage
\section{\exe{rac}}
\label{rac}
\index{rac@\exe{rac}}
\index{master host}
\index{code!management!changelog|see{\exe{rac}}}
\index{changelog|see{\exe{rac}}}
 
\exe{rac} is effectively the same as \exeref{ac}. The only difference
is that it is using \exeref{rao} and \exeref{rai} instead of
\exeref{ao} and \exeref{ai} to be able to update the changelog
from a remote site.

\subsection*{Synopsis}
 
\begin{synopsis}
   \code{\exe{rac} [\textrm{options for} \aipsexe{rac}]}
\end{synopsis}
 
\subsection*{Description}
You are referred to the description of \exeref{ac}.

\subsection*{See also}

\exeref{rai}, \aipspp\ code checkin utility.\\
\exeref{rao}, \aipspp\ code checkout utility.\\
\exeref{buildchangelog} process \aipspp\ changelog files.
 
\subsection*{Author}
 
Original: 2000/06/05 by Ger van Diepen, NFRA.

% ----------------------------------------------------------------------------
 
\newpage
\section{\exe{rai}}
\label{rai}
\index{rai@\exe{rai}}
\index{master host}
\index{code!management!checkin|see{@\exe{rai}}}
 
Check in \aipspp\ sources by invoking \exeref{ai} remotely via \unixexe{ssh}.

\subsection*{Synopsis}
 
\begin{synopsis}
   \code{\exe{rai} [\textrm{options for} \aipsexe{ai}]}
\end{synopsis}
 
\subsection*{Description}
 
\exe{rai} is an \unixexe{ssh}-based front-end to \exeref{ai}.  It checks in
files by transmitting them to the \aipspp\ master host via \unixexe{ssh} as a
\unixexe{gzip}'d tar file and running \aipsexe{ai} there.  This is provided as
an alternative to invoking \aipsexe{ai} directly on a machine on which the
\aipspp\ master directory is remotely \textsc{nfs} mounted.

In order to use \exe{rai} a programmer must have an account on
\host{aips2.nrao.edu} and it must be set up for \unixexe{ssh} access from the
local host via a suitable entry in the \file{.ssh/authorized\_keys} file
(\file{.ssh/authorized\_keys2} for ssh2 based access).
The programmer
must also have an up-to-date \aipspp\ workspace on \host{aips2.nrao.edu} and
it \emph{must} be  rooted at \file{\$HOME/aips++} (see \exeref{mktree}).
Visit the \htmladdnormallink{Using SSH page}{../../programmer/usingssh.html}
for more details on using ssh in \aipspp.

\subsection*{Options}
 
Refer to the options for \exeref{ai}.  \exe{rai} itself has no options.

\subsection*{Notes}
 
\begin{itemize}
\item
   \exe{rai} handles directories as well as files.

\item
   \exe{rai} is faster than \aipsexe{ai} because

   \begin{itemize}
   \item
      It only has to do one file transfer rather than two.

   \item
      It transfers files in compressed form.

   \item
      Although it needs to invoke \unixexe{ssh} two or three times the
      overhead involved in this is less than that of one automount of a
      remote filesystem.
   \end{itemize}

\item
   When accessing \host{aips2.nrao.edu} via \unixexe{ssh} the programmer's
   login initialization scripts (e.g. \file{.bashrc}, \file{.cshrc},
   \file{.esrc}, \file{.login}, \file{.profile}, \file{.rcrc}, \file{.tcshrc})
   must not consume \file{stdin} or write to \file{stdout} or \file{stderr}.

\item
   Where the account name on the \host{aips2.nrao.edu} differs from the local
   account name an environment variable may be used to translate between them:

\begin{verbatim}
   AUID="<remotename>"
\end{verbatim}

   \noindent

   The \file{.ssh/authorized\_keys}  (\file{.ssh/authorized\_keys2} for ssh2)
   file for the \acct{remotename} account on
   \host{aips2.nrao.edu} should have an entry for each machine you may want
   to use remotely.

   \noindent
   you will need to run ssh-keygen and put the generated key in the 
   \file{.ssh/authorized\_keys} on \host{aips2.nrao.edu}

\item
   \exe{rai} is not failsafe in the way that \aipsexe{ai} is failsafe.  It has
   no way of determining whether a checkin succeeded and so provides no
   guarantees one way or the other.  This arises because of the difficulty in
   recovering status returns from complicated command sequences executed by
   \unixexe{ssh}.

   If a checkin fails the error message from \unixexe{ci} will be visible on
   \file{stderr} but \exe{rai} will not catch an error return.  The only
   course of action would be to log in to \host{aips2.nrao.edu} and clean up
   manually.

   For this reason, \exe{rai} \emph{never deletes files from the local
   workspace} but simply removes the write permission from them (unless
   invoked with the \exe{-l} option) - \emph{even if the checkin failed}.  It
   is left to the programmer to decide whether the checkin succeeded and
   delete them explicitly.

   \exe{rai} does delete files from the \host{aips2.nrao.edu} workspace if the
   checkin succeeded.

\item
   \exe{rai} exhibits the following differencees from \aipsexe{ai} by virtue
   of the fact that it runs \aipsexe{ai} remotely through an \unixexe{ssh}:

   \begin{itemize}
   \item
      When \unixexe{ci} says to

\begin{verbatim}
   enter description, terminated with single '.' or end of file:
\end{verbatim}

      you must terminate with a single ``\code{.}'' and not use \verb+^D+.
      The reason is simply that \verb+^D+ means ``end-of-file'' to the
      \unixexe{ssh} as well as \unixexe{ci}, so if you use \verb+^D+ when
      checking in multiple files \unixexe{ci} will not accept change messages
      for succeeding files.

   \item
      There may be noticable delays in entering the change message.
   \end{itemize}

\item
   \unixexe{ssh} is used as the transfer agent rather than \unixexe{ftp} since
   the latter requires a password to be supplied for each remote access. 
\end{itemize}

\subsection*{Diagnostics}
 
Status return values
\\ \verb+   0+:  success (returned from \aipsexee{ai\_master}{ai})
\\ \verb+   1+:  initialization error, no checkin\\
 
\subsection*{Examples}
 
The command
 
\begin{verbatim}
   rai -au Foo.cc
\end{verbatim}
 
\noindent
would check in \file{Foo.cc} from the current directory into the corresponding
\aipspp\ master repository, updating the slave and plain-text copies in the
process.  The copy of \file{Foo.cc} in the current directory would \emph{not}
be deleted but it's write permission would be removed.  If the checkin
succeeded it would be appropriate to manually delete the file

\begin{verbatim}
   rm Foo.cc
\end{verbatim}
 
\noindent
If the checkin failed then it would be necessary to log in to
\host{aips2.nrao.edu} and recover manually.

\subsection*{Bugs}

\begin{itemize}
\item
   Not all of the \exeref{ai} options are supported yet due to setuid
   issues.
\end{itemize}

\subsection*{See also}

The manual page for \unixexe{ci}(1), the \rcs\ checkin command.\\
The GNU manual page for \unixexe{gzip}(1).\\
The unix manual page for \unixexe{ftp}(1).\\
The unix manual page for \unixexe{ping}(1).\\
The unix manual page for \unixexe{ssh}(1).\\
\exeref{ai}, \aipspp\ code checkin utility.\\
\exeref{ao}, \aipspp\ code checkout utility.\\
\exeref{au}, \aipspp\ code update utility.\\
\exeref{mktree} create \aipspp\ directory hierarchy.\\
\exeref{rao}, invoke \exe{ao} remotely via \unixexe{ssh}(1).\\
\exeref{rau}, update \aipspp\ sources from the master via \unixexe{ssh}(1).
 
\subsection*{Author}
 
Original: 1996/08/06 by Mark Calabretta, ATNF.

% ----------------------------------------------------------------------------
 
\newpage
\section{\exe{rao}}
\label{rao}
\index{rao@\exe{rao}}
\index{master host}
\index{code!management!checkout|see{@\exe{rao}}}
 
Check out \aipspp\ sources by invoking \exeref{ao} remotely via \unixexe{ssh}.

\subsection*{Synopsis}
 
\begin{synopsis}
   \code{\exe{rao} [\textrm{options for} \aipsexe{ao}]}
\end{synopsis}
 
\subsection*{Description}
 
\exe{rao} is an \unixexe{ssh}-based front-end to \exeref{ao}.  It checks out
files by running \aipsexe{ao} on the \aipspp\ master host and transmitting
them to the local host via \unixexe{ssh} as a \unixexe{gzip}'d tar file.  This
is provided as an alternative to invoking \aipsexe{ao} directly on a machine
on which the \aipspp\ master directory is remotely \textsc{nfs} mounted.
 
In order to use \exe{rao} a programmer must have an account on
\host{aips2.nrao.edu} and it must be set up for \unixexe{ssh} access from the
local host via a suitable entry in the \file{.ssh/authorized\_keys} file
(\file{.ssh/authorized\_keys2} for ssh2). The programmer
must also have an up-to-date \aipspp\ workspace on \host{aips2.nrao.edu} and
it \emph{must} be  rooted at \file{\$HOME/aips++} (see \exeref{mktree}). 
Visit the \htmladdnormallink{Using SSH page}{../../programmer/usingssh.html}
for more details on using ssh in \aipspp.
%$ 
\subsection*{Options}
 
Refer to the options for \exeref{ao}.  \exe{rao} itself has no options.
 
\subsection*{Notes}
 
\begin{itemize}
\item
   \exe{rao} handles directories as well as files.
 
\item
   \exe{rao} is faster than \aipsexe{ao} because
 
   \begin{itemize}
   \item
      It only has to do one file transfer rather than two.
 
   \item
      It transfers files in compressed form.
 
   \item
      The overhead involved in the invokation of \unixexe{ssh} is much less
      than that of one automount of a remote filesystem.
   \end{itemize}
 
\item
   When accessing \host{aips2.nrao.edu} via \unixexe{ssh} the programmer's
   login initialization scripts (e.g. \file{.bashrc}, \file{.cshrc},
   \file{.esrc}, \file{.login}, \file{.profile}, \file{.rcrc}, \file{.tcshrc})
   must not consume \file{stdin} or write to \file{stdout} or \file{stderr}.
 
\item
   Where the account name on the \host{aips2.nrao.edu} differs from the local
   account name an environment variable may be used to translate between them:
 
\begin{verbatim}
   AUID="<remotename>"
\end{verbatim}
 
   \noindent
   The \file{.ssh/authorized\_keys} file   (\file{.ssh/authorized\_keys2} for ssh2)
   for the \acct{remotename} account on
   \host{aips2.nrao.edu} should have an entry for each machine you may want
   to use remotely.

   \noindent
   you will need to run ssh-keygen and put the generated key in the 
   \file{.ssh/authorized\_keys} on \host{aips2.nrao.edu}
 
\item
   \unixexe{ssh} is used as the transfer agent rather than \unixexe{ftp} since
   the latter requires a password to be supplied for each remote access.

\item
   \exe{rao} will prompt the user if writable copies of the files being
   checked out exist, allowing the user to choose between replacing them
   and aborting the check-out.  Each file will be prompted for
   separately.

\item
   \exe{rao} will print a warning message when it removes
   locally-unwritable (read-only) copies of files in order to replace
   them with the checked-out copies.
\end{itemize}
 
\subsection*{Diagnostics}
 
Status return values
\\ \verb+   0+:  success (returned from \aipsexee{ao\_master}{ao})
\\ \verb+   1+:  initialization error, no checkout

\subsection*{Examples}
 
The command
 
\begin{verbatim}
   rao -l Foo.cc
\end{verbatim}
 
\noindent
would check out \file{Foo.cc} with an exclusive lock.  Then
 
\begin{verbatim}
   rao -u Foo.cc
\end{verbatim}
 
\noindent
would remove the lock without checking the file in or deleting the working
copy.  The command
 
\begin{verbatim}
   rao -l RCS/*,v
\end{verbatim}
 
\noindent
would check out everything in the directory (that is, assuming that \file{RCS}
is a symlink into the slave directory, all files currently in the slave
directory will be checked out of the master).

\subsection*{See also}
 
The manual page for \unixexe{co}(1), the \rcs\ checkout command.\\
The GNU manual page for \unixexe{gzip}(1).\\
The unix manual page for \unixexe{ftp}(1).\\
The unix manual page for \unixexe{ping}(1).\\
The unix manual page for \unixexe{ssh}(1).\\
\exeref{ai}, \aipspp\ code checkin utility.\\
\exeref{ao}, \aipspp\ code checkout utility.\\
\exeref{au}, \aipspp\ code update utility.\\
\exeref{mktree} create \aipspp\ directory hierarchy.\\
\exeref{rai}, invoke \exe{ai} remotely via \unixexe{ssh}(1).\\
\exeref{rau}, update \aipspp\ sources from the master via \unixexe{ssh}(1).

\subsection*{Author}
 
Original: 1996/08/06 by Mark Calabretta, ATNF.

% ----------------------------------------------------------------------------
 
\newpage
\section{\exe{rau}}
\label{rau}
\index{rau@\exe{rau}}
\index{master host}
\index{code!management!update|see{@\exe{rau}}}
 
Update \aipspp\ sources from the master via \unixexe{ssh}.

\subsection*{Synopsis}
 
\begin{synopsis}
   \code{\exe{rau} file1 [file2 ...]}
\end{synopsis}
 
\subsection*{Description}
 
\exe{rau} updates the \aipspp\ \file{slave} and \file{code} areas with the
lastest version of a file in \file{\$AIPSMSTR}.  It is provided as an
%$
alternative to \aipsexe{au} which operates on machines on which the \aipspp\
master directory is remotely \textsc{nfs} mounted.

In order to use \exe{rau} a programmer must have an account on
\host{aips2.nrao.edu} and it must be set up for \unixexe{ssh} access from the
local host via a suitable entry in the \file{.ssh/authorized\_keys} file
(\file{.ssh/authorized\_keys} for ssh2).  The programmer
must also have an up-to-date \aipspp\ workspace on \host{aips2.nrao.edu} and
it \emph{must} be  rooted at \file{\$HOME/aips++}.
%$
\exe{rau} assumes that you are working within a private directory tree which
shadows that of the \aipspp\ master source code tree, see \exeref{mktree}.
The filename arguments specified to \exe{rau} must be simple filenames without
path or \rcs\ \file{,v} suffix.  \exe{rau} determines the corresponding master
directory from the current working directory.

\exe{rau} transmits files from the master host to the local host via
\unixexe{ssh} as a \unixexe{gzip}'d tar file.  It gives them a file ownesship
defined by the owner of \exe{rau} (aips2mgr at the local site).  \exe{rau}
must itself have the \code{setuid} bit set (see \unixexe{chmod}).
 
If successful, \exe{au} then checks out a plain-text copy of the code from the
\file{\$AIPSRCS} subdirectory into the corresponding \file{\$AIPSCODE}
subdirectory.
 
Newly created directories on the master may also be propagated to the local
installation.

Programmers on a Linux or Digital UNIX platform, or on any other
platform that does not support setuid shell scripts, should instead use
a related utility, \exeref{sau}.  \exe{sau} runs a non-setuid copy of
\exe{au}, \exe{sau.sh}, via a C-based setuid wrapper program.
 
\subsection*{Options}
 
None.

\subsection*{Notes}
 
\begin{itemize}
\item
   \exe{rau} handles directories as well as files.
 
\item
   \exe{rau} is faster than \aipsexe{au} because
 
   \begin{itemize}
   \item
      It transfers files in compressed form.
 
   \item
      The overhead involved in the invokation of \unixexe{ssh} is much less
      than that of one automount of a remote filesystem.
   \end{itemize}
 
\item
   When accessing \host{aips2.nrao.edu} via \unixexe{ssh} the programmer's
   login initialization scripts (e.g. \file{.bashrc}, \file{.cshrc},
   \file{.esrc}, \file{.login}, \file{.profile}, \file{.rcrc}, \file{.tcshrc})
   must not consume \file{stdin} or write to \file{stdout} or \file{stderr}.
 
\item
   Where the account name on the \host{aips2.nrao.edu} differs from the local
   account name an environment variable may be used to translate between them:
 
\begin{verbatim}
   AUID="<remotename>"
\end{verbatim}
 
   \noindent
   The \file{.ssh/authorized\_keys} file   (\file{.ssh/authorized\_keys2} for ssh2)
   for the \acct{remotename} account on
   \host{aips2.nrao.edu} should have an entry for each machine you may want
   to use remotely.

   \noindent
   you will need to run ssh-keygen and put the generated key in the 
   \file{.ssh/authorized\_keys} on \host{aips2.nrao.edu}
 
\item
   \unixexe{ssh} is used as the transfer agent rather than \unixexe{ftp} since
   the latter requires a password to be supplied for each remote access.
\end{itemize}

\subsection*{Diagnostics}
 
Status return values
\\ \verb+   0+:  initialization succeeded
\\ \verb+   1+:  initialization error
 
\subsection*{Examples}
 
The command
 
\begin{verbatim}
   cd $HOME/aips++/code/install
   rau rau
\end{verbatim}
 
\noindent
would update the slave and plain-text copies of \exe{rau}, but not the script
in the \aipspp\ \file{bin} area.

\subsection*{Bugs}
\begin{itemize}
\item
   \exe{rau} requires that the host operating system support setuid
   shell scripts.  Some operating systems, most notably Linux and
   Digital UNIX, do not support setuid shell scripts for security
   reasons.
\end{itemize}

\subsection*{See also}

The manual page for \unixexe{co}(1), the \rcs\ checkout command.\\
The GNU manual page for \unixexe{gzip}(1).\\
The unix manual page for \unixexe{ftp}(1).\\
The unix manual page for \unixexe{ping}(1).\\
The unix manual page for \unixexe{ssh}(1).\\
\exeref{ai}, \aipspp\ code checkin utility.\\
\exeref{ao}, \aipspp\ code checkout utility.\\
\exeref{au}, \aipspp\ code update utility.\\
\exeref{mktree} create \aipspp\ directory hierarchy.\\
\exeref{rai}, invoke \exe{ai} remotely via \unixexe{ssh}(1).\\
\exeref{rao}, invoke \exe{ao} remotely via \unixexe{ssh}(1).\\
\exeref{sau}, \aipspp\ code update utility.
 
\subsection*{Author}

Original: 1996/08/16 by Mark Calabretta, ATNF.

% ----------------------------------------------------------------------------
 
\newpage
\section{\exe{rcscat}}
\label{rcscat}
\index{rcscat@\exe{rcscat}}
\index{rcs@\rcs!concatenate|see{\exe{rcscat}}}
 
Concatenate two \rcs\ version files.

\subsection*{Synopsis}
 
\begin{synopsis}
   \code{\exe{rcscat} rcsfile rcsarkv}
\end{synopsis}
 
\subsection*{Description}
 
\exe{rcscat} is a special-purpose utility which concatenates the archived
\rcs\ version file to the current \rcs\ version file for an \aipspp\ source.
It is used by \exeref{exhale} to update the \aipspp\ \rcs\ archive when
creating a new base release.

The first file argument specifies the master \rcs\ file which \exeref{exhale}
has just ``upreved'' to the new base release, \code{\textit{m}.0}.  It
contains all revisions from \code{\textit{l}.0} to \code{\textit{m}.0},
where \textit{l}~=~\textit{m}~-~1.

The second file argument specifies the archive \rcs\ version file which
contains all revisions from \code{1.0} to \code{\textit{l}.0}.  

The concatenated \rcs\ version file which \exe{rcscat} writes to \file{stdout}
contains all revisions from \code{1.0} to \code{\textit{m}.0}.
\exeref{exhale} makes this the new archive \rcs\ file.
 
\subsection*{Options}
 
None.

\subsection*{Notes}
 
\begin{itemize}
\item
   \exe{rcscat} is not intended for general use.

\item
   The process of producing a new base release is described in detail in the
   entry for \exeref{exhale}.

\item
   \exe{rcscat} is implemented as a \textsc{c} program.  It manipulates the
   internal contents of the \rcs\ version files, relying on the fact that
   revision \code{\textit{l}.0} is identical to the preceding revision.

\item
   \exe{rcscat} can handle binary as well as plain-text sources.
\end{itemize}

\subsection*{Diagnostics}
 
Status return values
\\ \verb+   0+: success
\\ \verb+   1+: usage error
\\ \verb+   2+: file access error
\\ \verb+   3+: \rcs\ file format error
 
\subsection*{See also}
 
The manual page for \unixexe{rcsfile}(5), \rcs\ version file format.\\
\exeref{exhale}, \aipspp\ code export utility.

 
\subsection*{Author}
 
Original: 1995/07/20 by Mark Calabretta, ATNF

% ----------------------------------------------------------------------------

\newpage
\section{\exe{sau}}
\label{sau}
\index{sau@\exe{sau}}
\index{master host}
\index{code!management!update|see{\exe{sau}}}

\aipspp\ source code update utility.

\subsection*{Synopsis}

\begin{synopsis}
   \code{\exe{sau} file1 [file2 ...]}
\end{synopsis}

\subsection*{Description}

\exe{sau} updates the \aipspp\ \file{slave} and \file{code} areas with the
lastest version of a file in \file{\$AIPSMSTR}.

\exe{sau} is a replacement for \exeref{au} (and \exeref{rau})
on operating systems, most notably Linux and Digital UNIX, that do not
support setuid shell scripts.  \exe{sau} is a C-based setuid wrapper
that runs a non-setuid copy of \exeref{au}, \exe{sau.sh}.

Refer to the \exeref{au} documentation for further information.

\subsection*{Options}

None.

\subsection*{Diagnostics}

Status return values
\\ \verb+   0+: success
\\ \verb+  -1+: exec of \exe{sau.sh} failed.
\\ \verb+     +Refer to the \exeref{au} documentation for other return values.

\subsection*{Examples}

The command

\begin{verbatim}
   cd $HOME/aips++/code/install
   sau au
\end{verbatim}

\noindent
would update the slave and plain-text copies of \exe{au}, but not the
script in the \aipspp\ \file{bin} area.

\subsection*{See also}

\exeref{au}, \aipspp\ code update utility.

\subsection*{Author}

Original: 1997/09/08 by Jeff Uphoff, NRAO

% ----------------------------------------------------------------------------

\newpage
\section{\exe{squiz}}
\label{squiz}
\index{squiz@\exe{squiz}}
\index{code!management!source search|see{\exe{squiz}}}

Search for and examine \aipspp\ sources.

\subsection*{Synopsis}

\begin{synopsis}
   \code{\exe{squiz} [\exe{-c} category] [\exe{-e}|\exe{-E} expression]
      [\exe{-l}] [\exe{-F}] [\exe{-p} package] [file]}
\end{synopsis}

\subsection*{Description}

\exe{squiz} searches for \aipspp\ source files and optionally searches for
strings inside the files it finds.  Wildcarding is allowed in the file
specification and the search is case insensitive by default.

If no file is specified the default is to search for (i.e. list) all files.

\subsection*{Options}

\begin{description}
\item[\exe{-c} category]
   Restrict the search to a certain category:
   \begin{itemize}
      \item
      \code{application}, \code{apps}, \code{app}: Application source files.

      \item
      \code{fortran}, \code{for}, \code{ftn}: \textsc{fortran} sources.

      \item
      \code{implement}, \code{imp}: Class implementation files.

      \item
      \code{test}, \code{tst}: Test programs.
   \end{itemize}
   The default is to search everything.

\item[\exe{-e} expression]
   Perform a case insensitive search for the specified expression in all
   selected files.  \unixexe{grep}-type regular expressions are allowed.
   See also the \exe{-l} option.

\item[\exe{-E} expression]
   Case sensitive version of the \exe{-e} option.

\item[\exe{-F}]
   Do a case sensitive search for the files named.

\item[\exe{-l}]
   Only list the names of files that contain a match to the expression 
   as specified in \exe{-e}.

\item[\exe{-p} package]
   Restrict the search to a particular package (see \sref{Code directories}).
   The default is to search all packages.
\end{description}

\noindent
Whitespace is allowed between short-form options and their arguments.

\subsection*{Diagnostics}

Status return values
\\ \verb+   0+:  success
\\ \verb+   1+:  initialization error

\subsection*{Examples}

The following command will search for all files in and below the
\file{implement} directory of the \file{aips} package (that is
\file{code/aips/implement/...}) whose name contains the string ``aips'' (case
insensitive) and will search inside each file that it finds looking for the
string ``debug'' (case insensitive).

\begin{verbatim}
   squiz -c imp -e debug -p aips "*aips*"
\end{verbatim}

\noindent
The following simply counts the number of files in the \file{doc} package:

\begin{verbatim}
   squiz -p doc | wc -l
\end{verbatim}

\noindent
Finally, this example will list out all files that contain the expression
``plonk'':

\begin{verbatim}
   squiz -l -e plonk
\end{verbatim}

\subsection*{See also}

The unix manual page for \unixexe{grep}(1).\\
\sref{Code directories}, \aipspp\ directory structure.

\subsection*{Author}

Original: 1994/07/28 by Mark Calabretta, ATNF\\
Modified: 1997/05/30 by Pat Murphy, NRAO

% ----------------------------------------------------------------------------

\newpage
\section{\exe{tract}}
\label{tract}
\index{tract@\exe{tract}}
\index{code!management!file age|see{\exe{tract}}}
\index{file!age|see{\exe{tract}}}
\index{directory!age|see{\exe{tract}}}

Report the age of a file or directory.

\subsection*{Synopsis}

\begin{synopsis}
   \code{\exe{tract} [\exe{-a}|\exe{-c}|\exe{-m}] [\exe{-q}\#] [\exe{-s}]
      <file|directory>}
\end{synopsis}

\subsection*{Description}

\exe{tract} reports the age of a file or directory; by default it reports the
time in seconds since the file or directory was last modified.  In query mode
it can be used to determine if a file or directory is older than a specified
number of seconds.

\subsection*{Options}

\begin{description}
\item[\exe{-a}]
   Report time since last access.

\item[\exe{-c}]
   Report time since last status change.

\item[\exe{-m}]
   Report time since last modification (default).

\item[\code{\exe{-q}\#}]
   Return a successful exit status if the file is older than the specified
   number of seconds.  Nothing is reported on \file{stdout}.

\item[\exe{-s}]
   Report the time in sexagesimal (h:m:s) format (default is seconds).
\end{description}

\subsection*{Diagnostics}

Status return values
\\ \verb+  -1+:  In query mode, the file is younger than the specified timespan.
\\ \verb+   0+:  Success.
\\ \verb+   1+:  Usage error.
\\ \verb+   2+:  File access error.

\subsection*{Examples}

The following reports the elapsed time in h:m:s format since tract was
installed:

\begin{verbatim}
   tract -s `which tract`
\end{verbatim}

\subsection*{See also}

The unix manual page for \unixexe{find}(1).

\subsection*{Author}

Original: 1995/02/21 by Mark Calabretta, ATNF

% ----------------------------------------------------------------------------

\newpage
\section{\exe{xrcs}}
\label{xrcs}
\index{xrcs@\exe{xrcs}}
\index{code!management!verification|see{\exe{xrcs}}}

Verify the internal consistency of \rcs\ version files.

\subsection*{Synopsis}

\begin{synopsis}
   \code{\exe{xrcs} [\exe{-q}] [file(s) | directory]}
\end{synopsis}

\subsection*{Description}

\exe{xrcs} exercises \rcs\ version files.  If a list of files is specified
then \exe{xrcs} will verify each; if a directory then all \rcs\ version files
in it and in all subdirectories will be verified.  Files and directories may
not be mixed.

File names may be specified without the \file{,v} suffix and without an
\file{RCS} subdirectory specification.  If \file{<file>} is the filename the
search order is:

\begin{verbatim}
   <file>,v
   RCS/<file>,v
   <file>
   RCS/<file>
\end{verbatim}

\exe{xrcs} verifies each version file simply by checking out the lowest
revision.  Since this requires \unixexe{co} to reconstruct every revision (on
the trunk) it can only succeed if the whole version file is synactically
correct.

If no file or directory is specified then \file{\$AIPSROOT/slave} is assumed.

\subsection*{Options}

\begin{description}
\item[\exe{-q}]
   Quiet mode, if a list of files was specified then don't report a successful
   verification.  If a directory was specified, don't report the name and
   revision of each file in a directory as it is processed.  A summary of any
   bad files found is always produced.
\end{description}

\subsection*{Notes}
 
\begin{itemize}
\item
   \exe{xrcs} is primarily used by \exeref{exhale} when producing a new base
   release.
\end{itemize}

\subsection*{Diagnostics}

Status return values
\\ \verb+   0+: success
\\ \verb+   1+: initialization error
\\ \verb+   2+: a bad \rcs\ version file was found

\subsection*{Examples}

To verify the \rcs\ version file for \exe{xrcs} itself:

\begin{verbatim}
   cd ~/aips++/code/install/codemgmt
   xrcs xrcs
\end{verbatim}

To verify the \aipspp\ slave repository in its entirety:

\begin{verbatim}
   xrcs
\end{verbatim}

\subsection*{See also}

The unix manual page for \unixexe{co}(1).\\
The unix manual page for \unixexe{rcs}(1).\\
\aipspp\ variable names (\sref{variables}).\\
\exeref{exhale}, \aipspp\ code export utility.

\subsection*{Author}

Original: 1994/08/10 by Mark Calabretta, ATNF.
