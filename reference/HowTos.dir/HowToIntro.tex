
\chapter{Where to Look\label{HowTo.Intro}}
\textit{You are in a maze of twisty passages --- all alike}. This famous line
from the game "Adventure" is also a common reaction to people who
attempt to find things, documentation especially, in the \textsc{aips++}
system.
In the belief that the
best way to learn is to do, we've assembled this "How To .*" document
to help you (and us too).
\section{Who Can Help Me?}
If you just plain get stuck, don't despair. There are a lot of
knowledgable, friendly people in the \textsc{aips++} project. If you cannot
find the help locally, one of the following people will help your or 
direct you to someone who can help you:

\begin{description} 
\item[Tim Cornwell(\htmladdnormallink{tcornwel@nrao.edu}{mailto:tcornwel@nrao.edu})]
Astronomical analysis and design, questions about AIPS++.
\item[Athol Kemball (\htmladdnormallink{akemball@nrao.edu}{mailto:akemball@nrao.edu})]
Class libraries and coding, Synthesis development.
\item[Darrell Schiebel (\htmladdnormallink{drs@nrao.edu}{mailto:drs@nrao.edu})]
Glish programming, System installation and utilities.
\item[Wes Young (\htmladdnormallink{wyoung@nrao.edu}{mailto:wyoung@nrao.edu})]
Documentation.
\end{description}

\section{Directories}

There are several major directory areas you should understand if you
wish to program or write documentation for \textsc{aips++}. You may wish
%This link ought to be an htmlref into the system manual.
to refer to the \htmladdnormallink{AIPS++ System Manual}{../System/System.html}
for further details.

\begin{enumerate}
\item 
There are the \emph{master} source directories, which contain the
sources of code and documentation \footnote{For historical reasons these
are all referred to as \emph{code}, although \emph{source} would be a
more accurate name}. Since these are \emph{master} sources, you cannot
directly modify them, rather you must modify them by checking them into
your \emph{personal} (see below) directory, modify them, and then check
them back.
\item
You modify source (program or documentation) in your \emph{personal}
directories. These directories shadow the \emph{master} directories.
Because new directories are created or deleted on occasion, you will
sometimes need to run the \texttt{mktree} program to reflect any directory
additions or deletions in the \emph{master} directories.
\item
\emph{Processed documentation} directories. Documentation which has been
processed (normally by being turned into PostScript and HTML) are placed into
common directory tree. During development of the documents you will be
working in your \textit{personal} directories.
\item
\emph{Binary} directories. These are where common (shared among all
programmers) programs and libraries go. These are architecture
dependent. During class and program development you will be working in
your \textit{personal} directories.
\end{enumerate}

The exact location of these directories will depend on how your local
system was set up, but will be of the following form:

\begin{description} 
\item[\emph{master}]
\texttt{/aips++/code}
\item[\emph{personal}]
\texttt{~/aips++/code}
\item[\emph{processed documentation}]
\texttt{/aips++/docs}
\item[\emph{binary}]
\texttt{/aips++/sun4sol\_gnu} or whatever architecture(s) you have at your
installation.
\end{description}

Sometimes instead of \texttt{/aips++} your common directories will be
under \texttt{/aips2/aips++} or \texttt{/u/aips++} or some other location.
You will need to ask locally.

Similarly, your personal directory need not be directly under your home
directory --- you may put it anywhere you have Unix "write" permission.

The directory location of a source file is often written as something like
\texttt{code/doc/reference/Coding.latex}. If you are working on it in your
\texttt{personal} directory you would prepend this path with
\texttt{~/aips++} and if you wanted to look at the master version you
would prepend the path with \texttt{/aips++} (or wherever your common
installation is).

\section{Documentation}

There are several types of documentation in the \textsc{aips++} system.
\begin{description}
\item[\emph{Memos and Notes}]
Various pieces of documentation for \textsc{aips++}. The distinction is
that \htmladdnormallink{\emph{Memos}}{../../memos/memos/memos.html} are meant
for external distribution and \htmladdnormallink{\emph{Notes}}{../../notes/notes/notes.html} are more internal. \index{notes and memos} 
These are under \texttt{code/doc/memos} and \texttt{code/doc/notes}
respectively.
\item[\emph{Specifications}]
User requirements and specifications for \textsc{aips++}. These are under
\texttt{code/doc/specs}.
\item[\emph{Papers}]
Various papers that have been written (\textit{e.g.}, for conference
proceedings) are available under \texttt{code/doc/papers}.
\item[\emph{Reference}]
Several reference documents exists
\begin{itemize}
\item \textbf{\htmladdnormallink{The AIPS++ Programming Manual}{../Coding/Coding.html}} is found in \texttt{code/doc/reference/Coding.latex}.
\item \textbf{\htmladdnormallink{AIPS++ System Manual}{../System/System.html}} is found in \texttt{code/doc/reference/System.latex}.
\item \textbf{How to .* for AIPS++ Programmers} (this document) is found in \texttt{code/doc/reference/HowTos.latex}
\end{itemize}
\end{description}


\label{Doc Sources}
\subsection{Multi-file Documents}

A simple convention is used to locate and place the individual files
used in a multi-source document. Basically, the "included" files are
placed in a subdirectory with a \texttt{.dir} extension, however the
including file is kept at the outer level. This scheme has the advantage
so that if a single-source file later becomes split up, the "including" file
itself does not move so you always find it in the same place.

So, as an example, we have a \textbf{\htmladdnormallink{A Guide to Synthesis Processing
in AIPS++}{../../user/Synthesis/}}
under the
\texttt{code/doc/user} directory. We call it
\texttt{synthesis.latex}.
At a later point we might want to split this document into
multiple \texttt{.tex} files for convenience, or we might have some
\texttt{.ps} files. In this code we would make a directory
\texttt{code/doc/reference/synthisis.dir} and place those files within it,
however synthisis.latex would remain in \texttt{code/doc/user}.

\label{Processed Documentation}
\subsection{Processed Documentation}

Processed source files (PostScript or plain text, depending upon the
source document) are all available under \texttt{/aips++/docs} (assuming
your common aips++ area is /aips++). The directory structure of
processed documents follows that of the source documents.  So, for
example, the postscript version of the "Coding" volume (\textbf{
\htmladdnormallink{The AIPS++ Programming Manual}{../Coding/Coding.html}})
would be \texttt{/aips++/docs/reference/Coding.ps.gz}.

\label{Online Documentation}
\subsection{Online Documentation}

All of AIPS++ documentation is available via the WWW.  Some is in HTML,
some in postscript, and some both in postscript and HTML.  

For an AIPS++ note or memo to appear in the AIPS++ web, the documenter must
add a pointer to the document in either 
the \texttt{code/doc/notes/notes.dir/index.tex} or
\texttt{code/doc/memos/memos/dir/index.tex} files.  To generate HTML from
\latex source, the documenter must add the document root name to the makefile
file with the desired \texttt{latex2html} options (typically, \texttt{-split 1}).

\chapter{Getting Help}

The AIPS++ team has developed many useful tools, classes, and programming
techniques.  These series of "How To's"
will hopefully get you started on the road through the \textit{twisty little
passages}. 

\section{Where to get Help}
Help today is "Hypertext". There are many avenues to get to the help
you need.  For programmers, you might want to start with the 
\htmladdnormallink{"programmer documentation"}{../../programmer/programmer.html}
http://aips2.nrao.edu/docs/programmer/programmer.html.
This web page has many links to useful documentation. Users should check
out \textbf{\htmladdnormallink{Getting Started in AIPS++}{../../user/gettingstarted/gettingstarted.html}}.  If you're primarily interested in
Glish Programming you will want to take a look at the \textbf{\htmladdnormallink{The 
Glish User's Manual}{http://www.cv.nrao.edu/aips++/glish/manual/}} and Rick Fisher's \textbf{\htmladdnormallink{Guide to Getting Started with Glish}{../../notes/195/195.html}}.

Most documents have postscript version which you may print out.  There are
a few "web only" which will require you to print the postscript from your
web browser.  

You may also \htmladdnormallink{search}{../../search/search.html} the
AIPS++ documentation tree from the URL \htmladdnormallink{
http://aips2.nrao.edu/aips++/doc/search/search.html}
{../../search/search.html}
\section{What Kind of Help}
We think of three types of programmers who will program in AIPS++,
the library builder, the applications, and the glish programmer.
All programmers need to pay particular attention to the coding
style, documentation requirements, and proper code checking.

If you're a glish programmer you probably want to look at the \htmladdnormallink{glish}{../../notes/195/195.html}
tutorial written by Rick Fisher or the \textbf{\htmladdnormallink{Glish user's manual}{http://www.cv.nrao.edu/aips++/glish/manual/}}.

If you're writting library tools, look at the modules inside the
\htmladdnormallink{aips package}{../../aips/aips.html}.

If you're an applications programmer, you may want to start with
\textbf{\htmladdnormallink{AIPS++ Programmer Reference}{../Coding/Coding.html}}
or from the 
\htmladdnormallink{"On-line programmer documentation"}{../../programmer/programmer.html},
http://aips2.nrao.edu/docs/reference/online/online.latex web page.
