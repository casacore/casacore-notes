\chapter{Getting Started \label{HowTo.GettingStarted}}
\section{Setting up your account to use \textsc{aips++}}
\index{setting up}
\label{Users}
\subsection{\textsc{aips++} user setup}
\index{setup - user}

The \textsc{aips++} environment is defined via a once-only modification of your
shell startup script.

Assuming that \textsc{aips++} has been installed under \textsl{/aips++}, users of
Bourne-like shells (\textit{sh}, \textit{ksh}, \textit{bash}) must add the following line to
their \textsl{.profile} file at a point \emph{after} \texttt{PATH} (and
\texttt{MANPATH}) are defined:

\begin{verbatim}
[ -f /aips++/aipsinit.sh ] && . /aips++/aipsinit.sh
\end{verbatim}

\noindent
The equivalent entry in \textsl{.login} for C-like shells (\textit{csh}, \textit{tcsh})
is:

\begin{verbatim}
if (-f /aips++/aipsinit.csh) source /aips++/aipsinit.csh
\end{verbatim}

The \textit{aipsinit} scripts define a single environment variable called
\texttt{AIPSPATH}\index{AIPSPATH}, and prepend the \textsc{aips++} \textsl{bin} area to the
\texttt{PATH} environment variable, and the \textsc{aips++} \textsl{doc} directory to
the \texttt{MANPATH} environment variable (There are no UNIX style man pages
available, \textsc{aips++} documentation is available on-line via the
world-wide-web).

For more detailed information, see \hyperref{aipsinit}{aipsinit(see AIPS++ System Manual, Section }{for more details)}{aipsinit}
\index{aipsinit}
including an explanation of \texttt{AIPSPATH} and a mechanism
for controlling the point where the \textsc{aips++} \textsl{bin} areas are added to
\texttt{PATH}.



\label{Programmers}
\subsection{\textsc{AIPS++} programmer setup}
\index{setup - programmer}
\paragraph{\textsc{AIPS++} Environment Variables}
Normally, \textsc{aips++} programmers must belong to the \textsc{aips++} programmer
group (in the unix sense) in order to have permission to write to the
\textsc{aips++} source directories.  The conventional name for this group is
\texttt{aips2prg}, but it may be set up differently at your site.  Consult your
local \textsc{aips++} manager.

\textsc{aips++} programmers must invoke the \textit{aipsinit} scripts from their
\textsl{.profile} or \textsl{.login} files as do users.  However, programming
is normally done on the \texttt{gnu} version of \textsc{aips++}. The required
incantation of \textit{aipsinit} for Bourne-like shells is

\begin{verbatim}
if [ -f /aips++/aipsinit.sh ]
then
   aips_ext=gnu
   . /aips++/aipsinit.sh
fi
\end{verbatim}

\noindent
and for C-like shells

\begin{verbatim}
if (-f /aips++/aipsinit.csh) then
   set aips_ext = gnu
   source /aips++/aipsinit.csh
endif
\end{verbatim}

\noindent
Note that the \textit{aipsinit} scripts explicitly unsets \texttt{aips\_ext}
In Socorro, the default aips\_ext is gnu, so there is no need to set it.

\paragraph{\textsc{AIPS++} Directory Tree}
Programmers also need to create a shadow copy of the \textsc{aips++} \texttt{code}
directory tree to serve as their \textsc{aips++} workspace.  The \textit{mktree}
utility does this automatically:

\begin{verbatim}
mkdir $HOME/aips++
cd $HOME/aips++
mktree
\end{verbatim}

\noindent
Apart from creating a shadow copy of the \textsc{aips++} \textsl{code} directory
tree, \textit{mktree} creates symbolic links into the local \textsc{aips++} \textsl{rcs}
\index{mktree}\index{rcs}
directory tree thereby linking the programmer's workspace directly to the
local copy of the \textsc{aips++} \textsc{RCS} repositories.  However, do not use
\textit{ci} or \textit{co} to check code in or out, see \hyperref{Checkout}{Checkout (see AIPS++ System Manual, Section }{ for more details)}{Checkout} for instructions
on how to do that.

\textit{mktree} works incrementally so that if any workspace directories or
\textsc{RCS} symbolic links are accidently deleted, or if new \textsc{aips++}
directories are created, \textit{mktree} will recreate only what is necessary.
For a more detailed description, see \hyperref{mktree}{mktree (see AIPS++ System Manual, Section }{ for more details)}{mktree}


\label{Programming}
\section{Getting started on \textsc{aips++} programming}

\label{Checkout}
\subsection{Checking source code in and out}
\index{code management}
\index{code check-in/out}
\paragraph{Code Management Background}
The first thing to realize about \textsc{aips++} code management is that, on
account of the distributed nature of \textsc{aips++} code development, a master
copy of the \textsc{aips++} sources resides in Socorro, and some consortium
sites have slave copies which are updated automatically at least once a day
by the \textsc{aips++} code distribution system (if all goes well that is -
and it usually does).

RCS is used for source code management in \textsc{aips++}, refer to the unix
manual page for \textit{rcsintro} for more information.  On both master and slave
systems the \textsc{aips++} sources reside within dual directory trees which exist
under the \textsc{aips++} root directory.   One of these, \textsl{~aips++/rcs},
contains the RCS source code repositories, and the other, \textsl{~aips++/code},
contains plain-text sources.  Symbolic links named \texttt{RCS} exist in each
subdirectory of \textsl{~aips++/code} and point into the corresponding
subdirectory of \textsl{~aips++/rcs}, thereby tying the two directory trees
together for checkout of sources.  The code distribution system updates the
\textsl{rcs} tree, and the \textsl{code} tree is then updated from it.  The
sources under \textsl{~aips++/code} are used to build the \textsc{aips++} system
locally, including object libraries, executables, documentation, system
databases and more.

When it comes to modifying or adding new \textsc{aips++} sources, the RCS checkin
and checkout operations must be performed directly on the master RCS
repositories in Socorro.  A valid but tedious way of doing this would
be to \textit{rlogin} to Socorro, checkout the sources you're interested
in, \textit{ftp} them back to your home machine, modify them, \textit{ftp} them back to
Socorro, and then check them back in.  Alternatively, \textsc{aips++}
sources can be checked in and out, updated, or deleted via the \textit{ai}, \textit{ao},
\textit{au}, and \textit{ax} utilities running on your local machine (\textit{ax} may only be
run in Socorro).  These commands work directly on the master
\textsc{aips++} RCS repositories via an NFS automount running across the internet
which was (or should have been) set up when \textsc{aips++} was first installed
at your site.  For this reason the commands may appear to execute slowly, or
in some cases you may get a message to the effect that the network is down and
to try again later (you will also get this message if the NFS automount was
not set up properly).

Note carefully that although the programmer workspace created by \textit{mktree}
contains \textsl{RCS} symlinks into the local RCS slave repositories, you should
never use \textit{co -l} to check code out with a lock, or \textit{ci} to check it back
in again.  This would only affect the local slave repository, your changes
would not be propagated back to the master in Socorro or the other
consortium sites.  The \texttt{RCS} symlinks in programmer workspaces are
provided so that you can check out sources without a lock, or use the \textit{rlog}
or \textit{rcsdiff} commands (provided you don't mind that the local RCS repository
may be out-of-date by up to a day).

\paragraph{Checking out source}
\index{checking out source}
\index{ao}
When you check out a file using \textit{ao} it is checked out from the master in
Socorro.  The \textit{ao} script is really a front-end for the ordinary
RCS \textit{co} command but with built-in knowledge of the \textsc{aips++} code
management system.  \textit{ao} takes all the same command line options as \textit{co}.
In particular, you can use \textit{ao -l filename.ext} to check out a file
named \textit{filename.ext}, locking others out of the capability to modify
the file while you are working with it; and you can use \textit{ao -u} to
cause a checkout to be forgotten.
Check the unix manual page for \textit{co} for more information.

\paragraph{Checking in source}
\index{checking in source}
\index{ai}
The checkin script, \textit{ai}, operates in much the same way as \textit{ao}.  It is a
front-end to the RCS \textit{ci} command and takes the same command line options.
However, \textit{ai} does more.  After checking the code back into the master in
Socorro, it updates the local RCS slave repository by copying the
master version back, and then updates the plain-text file in the \textsc{aips++}
\textsl{code} directory.  That way the local slave copy of the \textsc{aips++}
sources is updated immediately, and you don't have to wait for the automatic
update delivered by the code distribution system.

\paragraph{When to check in new code}
\index{checking in new code}
\index{ai}

Since AIPS++ is a production system, we are very concerned about the
stability of the system. As a result, one should be very careful about
checking in new features. The overall rules are:

\begin{itemize}
\item Maintaining the stability of daily is a high priority for
everyone in the project. Breaking daily affects the work of a large
number of people.
\item Maintaining the stability of weekly is essential. Breaking
weekly is disastrous for the work of a large number of people. Our
goal should be that every weekly can be marked as stable.
\item Checkins of changes to existing code should be made with
the above two goals in mind.
\item Changes to code that is used by others must be well-tested
on the project compiler (egcs 1.1.2) before checkin. Checkin of
purely developmental code is less crucial but still good habits 
should be followed. 
\item All checkins except simple ones, and bug fixes should be
done early in the week. Early means with two more of your inhales 
to get it right before the final weekly.
\item If you have a *really* substantial change, alert others
in the project via your weekly report beforehand.
\item All development sites should be running a daily using the
project compiler.
\end{itemize}

\paragraph{Other code management utilities}
\index{code management utilities}
\index{au}
\index{ax}
On occasion it may happen that someone from another consortium site checks in
some source code and you want your local slave repository to be updated
immediately.  The \textit{au} command does this for you by copying the RCS version
file from the master and then updating the plain-text copy.

Deletion of sources is handled by the \textit{ax} utility.  Since all deletions
must be propagated to the slave copies, and also because of certain timing
questions, \textit{ax} can only be run on the master sources in Socorro.

For a more detailed description of the various utilities discussed, see
\hyperref{Code management}{Code management (see AIPS++ System Manual, Section }{for more details)}{Code management}


\label{Compilation}
\subsection{Compiling your source code}
\index{make system}\index{compilation}

\paragraph{The make system}
System generation in \textsc{aips++} is handled by a hierarchy of GNU makefiles.
These makefiles are quite complex because of the complex job they have to
perform, building \textsc{aips++} on various platforms at any number of sites at
which \textsc{aips++} has been installed.

For a programmer, however, the main thing to recognize about the makefiles is
their two-sided nature.  Apart from rebuilding the \textsc{aips++} system from
installed sources, they also support programmer workspaces.  Where a
programmer has created new sources, or checked out and modified old ones, they
must recompile them locally and leave the results in the programmer's private
workspace.  Furthermore, programmers are not required to check out associated
sources, such as header files, just for the sake of the compilation.  After
searching unsuccessfully for a file in the programmers own workspace, the
makefiles carry the search into the standard \textsc{aips++} source code
directories.

\paragraph{System-targets}
\index{targets, system}
In accordance with the dual role of the makefiles their targets are of two
kinds, system-oriented and programmer-oriented.  The default target of every
\textsc{aips++} makefile, \texttt{all}, is programmer-oriented, and recompiles
everything in the programmer's workspace which is out-of-date.  It is usually
sufficient for programmers to invoke \textit{gmake} without arguments to recompile
their sources.  The corresponding system-oriented target is \texttt{allsys} and,
like all system-oriented targets, this recompiles any new sources in the
standard installation without any reference whatsoever to anything in the
programmer's workspace.

\paragraph{Programmer-targets}
\index{targets, programmer}
The programmer-oriented targets in the makefile for class implementation files
and applications deserve special mention.  They allow programmers to specify
private include file directories via the \texttt{MYINCL} variable which can be
defined on the command line

\begin{verbatim}
   gmake MYINCL=/home/bloggs/include MyClass.o
\end{verbatim}

\noindent
Likewise, the applications makefile provides a \texttt{MYLIBS} variable which
allows for private libraries.  If you wanted to compile an \textsc{aips++}
application and link it to class implementation files which you had modified
but not checked in, you would need to build a private library by using the
programmer-oriented \texttt{mylib} target in the implementation makefile.  You
could then link your application to this using the \texttt{MYLIBS} variable.

The \textsc{aips++} makefiles and the standard target names are discussed at
length in \hyperref{makefiles}{makefiles (see AIPS++ System Manual, Section }{ for more details)}{makefiles}































