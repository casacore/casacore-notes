\emph{(updated \today) with assistance from
  Mark Calabretta, Mark Wieringa, Shelby Yang, Brian Glendenning 
  \& Wim Brouw}

\section{Motivation}
The main reason I had for using the g++ compiler was that the template
instantiation bug in the Sun native compiler was really getting to me. It
meant for me that the time to compile nearly identical pieces of code could
vary from $\sim40$s to $\sim10$mins, depending on whether the compiler
(for unknown reasons) decided to re-instantiate everything or not. Sun is
aware of this bug and a recent patch has alleviated this problem somewhat.

In g++ the template instantiation is done by hand (with assistance from the
aips++ system) so this problem cannot occur. On a line by line basis the g++
compiler is slower than the Sun native compiler. In practice (my experience,
and in the time taken for system rebuilds) it appears to be faster,
because it does not do excessive recompilations.

Because of these problems with the standard Sun compiler and to aid
porting of AIPS++ to other architectures, g++ has become the standard
C++ compiler for AIPS++.

This note describes details on compiling AIPS++ code that are specific
to using g++ and the ``do it yourself'' template instantiation
mechanism that is used to specify which templates are created.

Code compiled and optimised under g++ is in my experience about 30\%
slower than code compiled and optimised under the Sun native compiler.
Currently we are investigating using the using the ``do it yourself''
template instantiation mechanism in conjunction with the latest
release of the Sun compiler and other compilers.

\section{g++ system setup}
These notes assume you have the g++ compilers installed and aips++
setup to use it. I know nothing about this and you should contact (in
order) either, Pat Murphy or Brian Glendenning (bglenden@nrao.edu) on
how to do this. The g++ compiler is now available at all of the
consortium sites, and version 2.7.2 has been extensively used at
Socorro. These notes relate to that version.

\section{Switching to g++}
Most sites are setup to have the gnu compiler as the default
compiler. 

If not you can switch to to the g++ compiler using:
\begin{verbatim}
   aipsinit gnu
\end{verbatim}
This assumes that you have already run aipsinit to get the normal
AIPS++ environment variables set.

\section{Setting up your workspace}

Ensure that the following directories exist (otherwise create them)
\begin{verbatim}
   ${MYAIPS2ROOT}/sun4sol_gnu
   ${MYAIPS2ROOT}/sun4sol_gnu/aux
   ${MYAIPS2ROOT}/sun4sol_gnu/bin
   ${MYAIPS2ROOT}/sun4sol_gnu/lib
   ${MYAIPS2ROOT}/sun4sol_gnu/bindbg
   ${MYAIPS2ROOT}/sun4sol_gnu/libdbg
\end{verbatim}
where \texttt{\${MYAIPS2ROOT}} is the root of your workspace. 
(eg. \texttt{\$HOME/aips++}). 

At Socorro where the code is on one partition and the executables are on
another (which is not backed up), you should substitute for
\texttt{\$MYAIPS2ROOT} the path \texttt{/tarzan2/\$USERNAME}, 
and create a symbolic link pointing from
\texttt{/tarzan/\$USERNAME/sun4sol\_gnu}  to
\texttt{/tarzan2/\$USERNAME/sun4sol\_gnu}
 
\section{Compiling}
Compiling should be done in the normal way (ie. using gmake in the
source directory), and with all the normal flags (eg. OPT=1). Using
gmake, in my experience, with the OPT=1 flag results in faster
compilation as well as optimised code. But the fastest way to
recompile your code is to use the OPTLIB=1 option. This links your
code (compiled with debug flags) against the optimised libraries, and
compiles about 2-3 times faster than not using any flags. It has the
added advantage that you can use the debugger to step through your
code (but not the system code). Using gmake without any flags also
results in executables that are about ten times bigger than the
optimised version (ie. they are huge).

Debugging should be done using gdb (use ddd or xxgdb for a graphical
interface), rather than the normal system debuggers (ie. dbx and
friends). Note that the commands differ from the standard ones (e.g.
no 'stop' but rather 'break').  Help can be obtained by typing 'help'
or 'help subject', while information is available with the 'info'
command (see 'help info').

If gmake dumps core it may be because you are using a very old version
that exists in \texttt{/opt/local1/bin} (Socorro only).
   
\section{Upcasting Bug\label{sec:upcast}}
The g++ compiler cannot do a few things the Sun native compiler can.
In particular if a global function with templated arguments of some
base class is called with a class derived from the base class it will
not find a match. (i.e. it cannot do upcasts)

This problem is particularly endemic in the Arrays module and for
these classes a particular solution has been devised. The Array class
(and hence derived classes) have an upcasting member function called
\texttt{arrayCast()} (or \texttt{ac()} for short).

For example in the aips/Arrays/ArrayMath.h class there is a function
\begin{verbatim}
   T max(const Array<T> &a) 
\end{verbatim}
which will not compile when called with a vector as an argument. ie.
\begin{verbatim}
   Vector<Float> vec(4);
   cout << max(vec) << endl; // error flagged on this line!
\end{verbatim}
but will compile if rewritten as:
\begin{verbatim}
   cout << max(vec.arrayCast()) << endl; 
\end{verbatim}

This solution gets particularly useful when doing arithmetic on
Vectors because g++ will complain about:
\begin{verbatim}
   Vector<Float> v1(4), v2(4);
   v1 += v2;
\end{verbatim}
And this should be changed to
\begin{verbatim}
   v1.arrayCast() += v2.arrayCast(); // make the cast explicit
   or 
   v1.ac() += v2.ac(); // emphasize the arithmetic not the cast
\end{verbatim}

This problem has also been recently seen in the Lattice classes and a
corresponding set of member functions called \texttt{latticeCast()}, or
\texttt{lc()} for short, have been written. 

On rare occasions you may need to selectively do an upcast when a templated
argument can be either a builtin data type (Int/Float etc.) or an Array
(Vector/Matrix etc.). For these occasions the functions \texttt{ at\_c()} \&
\texttt{ at\_cc()} exist. These global functions only upcast Arrays and
leave the builtin data types alone. They are defined in the
Array, Vector, Matrix \& Cube classes\footnote{you need to look at the .h
  files to see the definition as they do not show up in the .html files},
and in aips.h for the builtin data types (and in String.h also). 

The difference between the two functions is that \texttt{ at\_c()} is takes
a non-constant input arguement while \texttt{ at\_cc()} takes a constant
input arguement. Two function names are used rather than overloading just
one because of what we think is a bug in the g++ compiler. 

An example of this function in use in is the Measures module where a Quantum
has standard templates of \texttt{Double} and \texttt{Vector<Double>}. Then
the floor function is defined in QMath.cc as:
\begin{verbatim}
template <class Qtype>
Quantum<Qtype> floor(const Quantum<Qtype> &left) {
    Qtype tmp = left.getValue(); 
    Qtype ret;
    at_c(ret) = floor(at_c(tmp));
    return (Quantum<Qtype>(ret,left));
}
\end{verbatim}

If you see this problem in other class hierarchies, you can do an explicit
type cast but are instead urged to create an upcasting member function in
the base class. The advantage of using member functions, rather than
explicit casts, is that if g++ is ever able to do upcasts in the future it
will then be easier to search for the upcasting functions and remove
them. Please tell me if you do this, so I can add it to the documentation. 

It is expected that this bug will disappear in g++ version 2.8.0 (private
communication between Brian and one of the g++ developers).

\section{Typedef Bug\label{sec:typedef} }   
The g++ compiler sometimes complains it cannot
find a type definition even when it has definitely been included. For
example I found that it was complaining that \texttt{ Vector<Int>} was
undefined even though \texttt{ aips/Arrays/Vector.h} was included. To
solve this add the line:
\begin{verbatim}
   typedef Vector<Int> gpp_vector_int;
\end{verbatim}
near the top of the .cc file. the type \texttt{ gpp\_vector\_int} is
arbitrary but this seems like a good convention (Brian's convention).
   
This appears to occur when a specific templated types used in the context
of a general template definition. eg.
\begin{verbatim}
   template <class T> const Lattice<Bool> & PagedImage<T>::mask() const
\end{verbatim}
will trigger this bug because requiring a typedef for \texttt{Lattice<Bool>}

I enclose this change and others that are specific to the gnu compiler in
ifdefs's to avoid getting other compilers off side. eg.
\begin{verbatim}
   #ifdef __GNUG__
      typedef Vector<Int> gpp_vector_int;
   #endif
\end{verbatim}
It is not necessary to do this when using the array upcasting functions
(arrayCast() etc.) described in section \ref{sec:upcast}.

The error messages that indicate you are being affected by this bug can be
quite obscure and some are listed in section~\ref{sec:error-messages}

\section{Do it yourself templates (creating a templates file)}
When linking the aips++ system will have a large number of template
instantiations already done for you. However there may be some
template instantiations you will have to create for yourself. This is
done with a personal template instantiation list. To do this create a
file called \texttt{templates}. This should be in the directory
containing the test program for the class you are currently working on
(eg., \texttt{.../code/trial/implement/MyModule/test/templates}) or in
the applications directory (eg.,
\texttt{.../code/trial/apps/MyApp/templates}). gmake will see this
file and ``knows'' how to create the templates you have specified. You
should \emph{never} create a templates file in a code module directory
(eg.  \texttt{.../code/trial/implement/MyModule/templates} as the
templates in this file will overwrite the system templates (when you
check the templates file in).

The format of the template entries is described in detail in Chapter 9
of the AIPS++ system manual (the mkinst section).  
I will only outline the basics  here.

The templates file should contain entries like:
\begin{verbatim}
1010 trial/Arrays/Convolver.cc 
     template class Convolver<Float>
1000 aips/Arrays/ArrayMath.cc aips/Arrays/Array.h
     template Double min(Array<Double> const &)
\end{verbatim}

The entries are:
\begin{description}
\item[Identification number:] The exact number is arbitrary but it
  should be a unique number for each .cc file, and should be (by
  convention) a 4 digit number. An value of four zeros will be
  converted to a unique number by ``reident'' (described below).
\item[template file(s):] The .cc file containing the template. If using
  templates like \texttt{ myClass< Array<Float> >} you should also
  include a .h file containing the definitions of the
  template arguments. In this example \\
  \texttt{ trial/Utilities/myClass.cc aips/Arrays/Array.h}
\item[template:] A keyword separating the filenames from the required
  template. This usually goes in a separate line although it is not
  manditory to do this.
\item[required template:] This states which specific template is required
  to be instantiated. It can either be a class or a global
  function. An example of each is shown above. 
\end{description}
See the system template instantiation list in: \\
\texttt{ .../code/\{aips,trial\}/implement/\_ReposFiller/templates} 
for further examples and a listing of the system generated 
templates. These templates are the ones necessary to link the aips++
applications.

Templates that are used by the test programs should not be put in the
system file (unless there is an overriding reason), but stored in
separate templates files that exist in the test directories. This keeps
the aips++ libraries at a manageable size and avoids excessive link
times. 

If your templates file is getting long and unmanageable there is a
utility called \texttt{reident} to sort/uniq, re-sequence and
``canonicalise''the templates file.

In the default mode\footnote{There is an old mode enabled using the
  \texttt{ -z} option which adds identification numbers starting at
  1000 for each .cc file and increments by 10 to allow sufficient
  space for inserting new entries. You should rebuild your object
  library from scratch after running reident using this mode} reident
will do the following:
\begin{itemize}
\item Adjust the identification number on the template definitions so
  that they are unique. It will also add identification numbers if they
  are missing and replace the identification numbers for all templates
  below 1000.
\item Convert builtin types to AIPS++ standard definitions. eg., float
  will be converted to Float.
\item Remove unnecessary white space.
\item Warn if the const keywords are incorrectly positioned
\item Warn if global functions do not have a return type
\item Check that \verb!#ifdef! and \verb!#else! and \verb!#endif! are 
  properly sequenced.
\item Comment out bad templates entries.
\end{itemize}

Its output is always to the file called templates (and input and output
\emph{can} be the same file) so I usually just type
\begin{verbatim}
reident
\end{verbatim}
as the default input file is also called templates and I have already
appended my changes tothis file.

reident can also merge multiple template files together so that
\begin{verbatim}
reident templates templates.additions
\end{verbatim}
will merge both input files into the output file. 

The shell comment character \texttt{\#} can be used to comment entries in
your templates file and the preprocessor directives \verb!#if!,
\verb!#if defined(symbol)!, \verb!#if !defined(symbol)!, \verb!#ifdef!,
\verb!#ifndef!, \verb!#else!, \verb!#endif! can also be used. 

Each entry in the templates file gets compiled into an object file
called myClass\_Ident.o where the myClass is the name of the first .cc
file in the template line and Ident is the indentification number.
Normally this object file contains only one instantiation. But often
templates instaitions are required in groups. For example to use a
\verb!CountedPtr<SkyCompRep>! requires you instantiate four other
classes, namely: \verb!CountedConstPtr<SkyCompRep>!,
\verb!PtrRep<SkyCompRep>!, \verb!SimpleCountedConstPtr<SkyCompRep>! \&
\verb!SimpleCountedPtr<SkyCompRep>!. In this case it is better to
group all five instantiations in one object file. This is done by
adding multiple template lines as shown below. 
\begin{verbatim}
1000 aips/Utilities/CountedPtr.cc trial/ComponentModels/SkyCompRep.h 
     template class CountedConstPtr<SkyCompRep> 
     template class CountedPtr<SkyCompRep> 
     template class PtrRep<SkyCompRep> 
     template class SimpleCountedConstPtr<SkyCompRep> 
     template class SimpleCountedPtr<SkyCompRep> 
\end{verbatim}

\section{Demangling the template names}
When linking the names of the missing class/function instantiations may
be printed in a mangled form. 
To demangle them save the g++ output in a
file and use \texttt{ /usr/local/gnu/bin/c++filt}. \\
An example (Using csh) is:
\begin{verbatim}
   gmake test.cc >&! g++.out
   /usr/local/gnu/bin/c++filt < g++.out
\end{verbatim}
or using sh 
\begin{verbatim}
gmake test.cc 2>&1 | /usr/local/gnu/bin/c++filt | tee missing-templates
\end{verbatim}

The c++filt command will translate the following mangled name
\begin{verbatim}
getArray__Ct11MaskedArray1Z14PointComponen
\end{verbatim}
to
\begin{verbatim}
MaskedArray<PointComponent>::getArray(void) 
\end{verbatim}
and because the getArray function is a member function of the MaskedArray
class we need to instantiate the whole class 
(\texttt{MaskedArray<PointComponent>})

\section{Checking in your templates}

When code is checked in, the templates entries should be added to the
relevant system/test template instantiation file:
\begin{verbatim}
   .../code/{aips,trial}/implement/\_ReposFiller/templates
\end{verbatim}
if the templates are associated with an application or
\begin{verbatim}
   .../code/{aips,trial}/implement/yourModule/test/templates
\end{verbatim}
if the templates are associated with a test program. \footnote{If you are
  modifying an existing class and you find you need to add additional
  templates to the test program it is likely that these same templates will
  be used by other programs using this class. So you will need to go and
  find them and if they are applications then just put the templates in the
  system (\_Reposfiller) templates file, otherwise you will need to modify
  the templates file of other test programs that use the modified class}

Be very careful when adding your templates to the system file as any
duplication of the indent. number between your added templates
and the already defined ones will cause the next system rebuild to
fail (a sure way to become popular). Mark Calabretta suggests:

``When you add idents to the system templates file you should always run
'reident' before checking it back in.  A good strategy is to give your
new entries a 0000 ident.''

A question that is often asked is ``when do I put a template in the aips
package and when do I put it in the trial package?''. If any template
required for an application uses code (.cc or .h files) only from the aips
package it should go in the \texttt{aips/implement/\_Reposfiller/template}
file. If it uses anything from the trial package it should go in the
\texttt{trial/implement/\_Reposfiller/template} template file.  This avoids
the problem of having undefined symbols when some purely aips templates are
defined in the trial library (at the expense of having a larger than
necessary aips library).

Pure aips templates that are needed for test programs in the aips package
should still be in \texttt{aips/implement/yourModule/test/templates}. The
aips test programs should not rely on anything in trial, they are not linked
against the trial libraries, and should not use any templates from the trial
package.

Any templates (pure aips or involving trial) that are needed for test
programs in the trial package should still be in the
\texttt{trial/implement/yourModule/test/templates} file. This may mean that
you may have to use the \texttt{EXTRA\_PGMRLIBS} flag when compiling these
test programs, although we will attempt to automate this.

You should also ensure that the templates you are adding to the system are
not defined alsewhere. Duplicate templates lead to
link time errors on many systems (eg. Dec Alpha). The policy for checking
for duplicates is:
\begin{itemize}
\item If checking new templates into
  \texttt{aips/implement/\_Reposfiller/templates} then you \emph{must}
  ensure that the same template is not duplicated in any of the following
  template files:
  \begin{itemize}
  \item \texttt{aips/implement/anyModule/test/templates} 
    (remove templates from test if any found)
  \item \texttt{trial/implement/anyModule/test/templates} 
    (remove templates from test if any found)
  \end{itemize}
\item If checking new templates into
  \texttt{aips/implement/anyModule/test/templates} then you \emph{must}
  ensure that the same template is not duplicated in
  \begin{itemize}
  \item \texttt{aips/implement/\_Reposfiller/templates}.  (template in
    test should not be necessary)
  \end{itemize}
\item If checking new templates into
  \texttt{trial/implement/\_Reposfiller/templates} you need to check
  \begin{itemize}
  \item \texttt{trial/implement/anyModule/test/templates} (remove templates
    from test if any found)
  \end{itemize}
\item If checking new templates into
  \texttt{trial/implement/anyModule/test/templates} then you need to check:
  \begin{itemize}
  \item \texttt{aips/implement/\_Reposfiller/templates} (template in test
    should not be necessary)
  \item \texttt{trial/implement/\_Reposfiller/templates} (template in test
    should not be necessary)
  \end{itemize}
  
\end{itemize}

Brian Glendenning has a tool that can automatically check for template
duplicates. It is checked into the system and information on how to use it
can be obtained by typing \texttt{duplicates -help}

\section{Error Messages\label{sec:error-messages}}
The gnu compiler can generate many different error messages that are in
fact caused by the problems described in section \ref{sec:upcast} and
\ref{sec:typedef}. Here I intend to list the error messages reported for
each of the problems as an aid to programmers. Please forward new error
messages to me (rmarson@nrao.edu).

\subsection{Upcasting Bug (section \ref{sec:upcast})}
\begin{itemize}
  \item type unification failed for function template
  \item too few arguments for function
\end{itemize}

The upcasting bug has also been known to cause an exception to be
thrown. This occured in the following piece of code:
\begin{verbatim}
template <class Qtype>
ostream& operator<< (ostream &os, const Quantum<Qtype> &ku) {
    os << ku.qVal;
    return os;
}
\end{verbatim}
where ku.qVal could be a \texttt{Double} or a \texttt{Vector<Double>} (but
the \texttt{operator<<} function is defined for Arrays not Vectors). The
compiler did not detect that there was an upcasting problem and examination
with a debugger revealed it was getting into an infinite constructor loop!
Further investigation revealed that the compiler did not find a
function match using an upcast but was able to find another unintended
match, that was further resolved to the original function!

\subsection{Missing Typedef Bug (section \ref{sec:typedef})}
\begin{itemize}
  \item invalid use of undefined type  
  \item parse error at null character
\end{itemize}

\subsection{Other Gotcha's}
You will probably have problems linking when you modify a class and then
create a new set of templates (all in your private workspace), if the same
templates also exist in the system library. This is because the newly
modified templates may not be required until after your private library has
been searched and the linker is using the system library. By this stage it
will not know to use the templates in your personal library and instead
pickup the system ones. The solution to this problem is.
\begin{itemize}
\item Wait for a system update. This may be overnight or take a week.
\item Copy the system library into your own workspace prior to compiling the
  templates. 
\end{itemize}
