\newcommand{\thisdocURL}{http://aips2.nrao.edu/aips++/docs/notes/202/202.html}
\newcommand{\searchURL}{../../html/aips2search.html}
\newcommand{\aipsviewURL}{../../user/aipsview/aipsview.html}
\newcommand{\GlishmanualURL}{../../reference/Glish/Glish.html}
\newcommand{\GlishtutorialURL}{../../notes/195/195.html}
\newcommand{\synthesisURL}{../../user/synthesis/synthesis.html}
\newcommand{\gsURL}{../../user/gettingstarted/gettingstarted.html}
\newcommand{\userrefmanualURL}{../../user/Refman/Refman.html}
\newcommand{\specsURL}{../../specs/specs.html}
\newcommand{\betaURL}{../beta/beta.html}
\newcommand{\consortiumURL}{../consortium.html}
%
% URL to Help system
%
\externallabels{../../user/Refman}{../../user/Refman/labels.pl}

% Add home page navigation button
%

\htmladdtonavigation{\htmladdnormallink
  {\htmladdimg{../../gif/home.gif}}{{../../html/aips++.html}}}

\section{Purpose}

The purpose of this document is to suggest interactive demonstrations
of AIPS++.

\section{General reminders}

\begin{itemize}
\item Set DISPLAY
\item Use a stable version of AIPS++, {\em e.g.} /aips++/beta
\item Ensure that the netscape is local to the machine being used or
can see the same disks
\item Work in a clean directory. Know what is in the active .aipsrc
and .glishrc files. I recommend that you put the following in your
.aipsrc file:
\begin{verbatim}
objectcatalog.default:	gui
\end{verbatim}
\item Do a trial run!
\end{itemize}

\section{General capabilities}

\begin{itemize}
\item Say what AIPS++ is:
\begin{itemize}
\item Astronomical Information Processing System
\item C++, Scripting, GUIs, libraries, toolkits, and applications
\item Designed by a team of astronomers and programmers
\item Developed by an international consortium of radio observatories
\end{itemize}
\item Start aips++
\item Talk about logger windows, clients. 
\begin{itemize}
\item Show options on File and Options menu. 
\item Show how help works
\begin{verbatim}
help()
help('aips')
help('display')
web()
\end{verbatim}
\item Show {\em about AIPS++} window, talk about configuration of
AIPS++ (aipsrc variables)
\item Show Object Catalog GUI. Talk about objects in AIPS++.
\end{itemize}
\item Talk about Glish. Show capabilities of Glish:
\begin{itemize}
\item Simple vectors: 
\begin{verbatim}
a:=1:1000
b:=sin(pi*(a/360)^2)
print a[1:10] , b[1:10]
\end{verbatim}
\item Plotting: 
\begin{verbatim}
mypg:=pgplotter()
mypg.plotxy(a, b, "Chirp")
\end{verbatim}
\item Servers and FFTs. Do a help first:
\begin{verbatim}
- help('*fft*')      

There are 10 matches. Please choose from:
  aips.mathematics.fftserverdemo
  aips.mathematics.fftservertest
  aips.mathematics.fftserver
  aips.mathematics.fftserver.complexfft
  aips.mathematics.fftserver.realtocomplexfft
  aips.mathematics.fftserver.convolve
  aips.mathematics.fftserver.crosscorr
  aips.mathematics.fftserver.autocorr
  aips.mathematics.fftserver.shift
  synthesis.synthesisdos.imagesolver.setfft
F 
- help('*fftserver')

 fftserver -- Object  -- aips.mathematics

  FFTs and related operations

 Methods
   complexfft   Full complex-to-complex in-place FFT of an array
   realtocomplexfft     Real to Complex FFT of an array
   convolve     Convolve a model with a psf
   crosscorr    Cross-correlate two real arrays
   autocorr     Auto-correlate an array
   shift        Shift an array some number of pixels with an FFT

You may find more information in the on-line documentation available
via your web browser.  Type the command

   web()

to view more about aips.mathematics.fftserver.

F 
- web()
\end{verbatim}
Now use it:
\begin{verbatim}
myfft:=fftserver()
c:=myfft.realtocomplexfft(b)
mypg.clear()
mypg.plotxy(a, abs(c), "FT of Chirp")
\end{verbatim}
\item Point out that {\tt myfft} and {\tt mypg} show up in the Object
Catalog. Use the Object Catalog to delete {\tt myfft}, since we no longer
need it.
Zoom on plot to show detail.
\item Talk about the plotter. Show the plotting capabilities given by the
PGPLOT tk widget (courtesy of Martin Shepherd and Tim Pearson!)
as bound to Glish. To do this, go to the menu of the plotter and
under Options, use the Add Commands option. Scroll until you find
demo, then select it and go.
\item Aipsview. Make and display an array:
\begin{verbatim}
a:=array(1:1000, 100, 100)
dd.array(a)
\end{verbatim}
\end{itemize}
\end{itemize}


\section{Synthesis}

\begin{itemize}
\item Start the File Catalog to see the files as they are generated:
either use the Object Catalog to Show dc or enter by hand:
\begin{verbatim}
dc.gui()
\end{verbatim}
\item Set up an {\tt imager} using the {\tt imagertester} constructor. This will
create a table 3C273XC1.ms and set the state correspondingly.
\item Refresh the File Catalog and Browse 3C273XC1.ms. Remember to close the table. Talk about
data access in AIPS++: you can always get at it and edit it if you like.
\item First deconvolve an an image using the {\tt clean} method. The
default inputs should be ok.
\item Second, restore the image using the {\tt restore} method.
\item Refresh the File catalog and View the image clean.restored
\item Set levels on aipsview (-0.03, 0.05 works). Go through all
planes. Show the axes.
\item An option is to use {\tt simpleimage} to show wizard-like
processing. You'll be prompted for answers to a few questions.
\begin{verbatim}
simpleimage()
\end{verbatim}
\end{itemize}

\section{Measures GUI}

\begin{itemize}
\item Find Local Apparent Sidereal Time from Greenwich Mean Time. First start 
measures gui by using the Object Catalog Show button after selecting dm.
\begin{itemize}
\item Select Position gui
\item Under OBS, select your observatory e.g. GB
\item Select output as IRTF
\item Press $->$Convert to convert to IRTF
\item Press $->$Frame to define this as a frame
\item Select Epoch gui under main measures gui
\item Enter the string 'today' (no quotes) in the slot on
the left
\item Select output as LAST
\item Press $->$Convert to convert to LAST (presto)
\end{itemize}
\end{itemize}

\section{Single Dish}

\begin{itemize}
\item Start the single dish environment, dish
\begin{verbatim}
include 'dish.g'
\end{verbatim}
If someone has previously used dish from the same account, then dish
will start up in the same state as it was when last used.  This can
be pointed out after the dish GUI appears.  If you want to guarantee
a pristine, default state then you must remove the directory which
holds that state - \$HOME/aips++/dishstate.  Note that if you do start with
some previously set state which was arrived at after following this
demo, you may be able to skip the steps involving opening data sets
(i.e. check first to see if it is already opened).  Also, it is probably
a good idea to clean out the Results Manager periodically if giving
this demo repeatedly so that it doesn't become too cluttered with the
results of past demonstrations.  Just select what you want to forget
and press the ``Delete'' button in the Results Manager.
\item Create the demo data tables for reading
\begin{verbatim}
include 'dishdemo.g'
dishdemo()
\end{verbatim}
This will create the demo data in the current directory.  If the demo data 
already exists this will do nothing. It is quite safe to repeat it
if you aren't sure the demo data exists.  It takes a while.  It sends
messages to the logger as it processes each SDFITS file
with fits2table.

The current demo data consists of the following tables:
\begin{itemize}
\item {\tt dishdemo1} and {\tt dishdemo1}.  These are simulated data sets which are
especially useful for demonstrating the averaging operations.
\item {\tt dishparkes}.  This is some sample data from Parkes.  This is useful
for showing how the dish plotter displays multi-polarization data.
\item {\tt dishmopra}.  This is some sample Mopra data.  This is used in
the command line portion of this demonstration.  It consists of 4 scans
of position switched total power data.  The first two scans are ``on''
scans at two separate frequencies and the last two scans are the
corresponding ``off'' scans at those same frequencies.
\item {\tt dishspecproc}.  This is some GBT Spectral Processor data taken using
the 140' earlier this year.  It is useful for showing real data.  It can
be used in the selection and averaging part of this demonstration but
because it was used without the benefit of all of the GBT Monitor and
Control devices and software it lacks several important header parameters
(no telescope position, no system temperature information, and no
velocity frame and rest frequency information).
\end{itemize}
\item Load the Parkes table ({\tt dishparkes}) using the File menu (select Open
and Read Only)
\item Browse it
\begin{itemize}
\item Select it in the Results Manager
\item Press the ``Browse'' button.
\item Once the browser has come up, you can move around in the browser
by selecting different records, by using the up and down arrow keys as well
as the Home and End keys.
\item Look at a number of records and see how the plot changes
\item The first and fourth records have two polarizations.
\item Play with plotting options in DISHPLOT (Styles, Coordinates,
Statistics, and Options are the most useful).
\item Play with the browser options.
\item Select ``Browse record'' in the browser.
\begin{itemize}
\item Change to a different coordinate system for ``Direction'' by pushing
the button labeled ``B1950'' and selecting ``J2000'' (button label should
change).
\item Change to a different format for the Time display by pushing the
button labeled ``Time'' and selecting ``dmy''.
\item Change to a different time system by pushing the button labeled
``UTC'' and selecting ``GAST'' for example.  Note that some conversions
will not be possible with this data set because the telescope position is
not in the header (e.g., a conversion to LST is impossible).
\item Experiment with the other buttons.
\item Display the data values by pressing the ``Browse arr'' button.
\item Show the history by pressing that button (scroll down first).
\end{itemize}
\item Click on any number of other records in the ``SD Working Set Browser''
window to see them update the record browser window and the plotter.
\end{itemize}
\item Dismiss the browser windows.
\item Open the {\tt dishspecproc} data set.
\item Select two operations: Selection and Averaging (Operations menu).
\begin{itemize}
\item Press some of the ``All'' buttons on the Selection window to show all 
of the unique values for those quantities in this data set.  Make sure you
push the ``Object'' field's ``All'' button.
\item In the Object entry enter ``1328*'' to select both 1328+305 and 
1328+305SBT
\item Turn on the Object selector by pressing the box between the combobox down
pointer and the ``All'' button.
\item Press the ``Apply'' button on the Selection window.  This produces a
new working set (in the Results Manager as well as in the Selection window).
\item In the Averaging window, turn off the ``Make selection before averaging'' option
since we have just made the selection (it turns out that it doesn't
matter in this case since the act of making the selection clears the
selectors and when the selectors are clear, no selection happens and the
entire current working set is used - which is what we want).
\item Press the ``Apply'' button in the averager and look at the result
(the result will appear in the plotter and be selected in the result manager).
You may want to manually adjust the Y axis range on the plotter.
\begin{itemize}
\item Under the Options menu select Scaling and then Manual
\item Enter the appropriate limits for the Y axis.  For this example,
MinY of 0.4 and MaxY of 0.6 seem to work well.
\item Press return in the MaxY or MinY entry fields or press the Redraw
button to have these ranges take effect.  Be sure to point out during
this operation that cursor-driven zoom is expected very soon.
\end{itemize}
\item Browse (this just brings up the individual record browser when the focus
is on an individual record).  Look at the history of this result.
\item Note that because of the limitations of this data set discussed
above, if you want to play with other alignment options it is best to use
one of the fake data sets (dishdemo1 and dishdemo2).  For this data set,
you really can only choose no alignment or align by x-axis.  You can
not choose tsys weighting because there are no tsys values.
\item You can browse the working set created as a result of
the selection operation.  You can browse more than one thing at a time.
\end{itemize}
\item Any of these results can be examined at the glish prompt.
For example, type ``print average1.hist'' at the glish prompt.  This shows
the history.  Type ``print average1.data.desc''.  This shows the 
axis description.
\item You can rename any variable in the results manager by typing in a
new name and hitting return.
\item Now do some smoothing (turn ``off'' the other operation GUIs to
conserve real estate on your desktop by pressing their ``Dismiss'' buttons).
\begin{itemize}
\item Select a record (either the result of a previous operation, or
through the browser on a working set).  This operation works on the
most recently displayed record (as do all operations which work on single
records).
\item Try different types of smoothing.
\item Try Hanning or Boxcar with and without ``decimation''
\item Do a Gaussian smooth.
\begin{itemize}
\item Select ``Gaussian''
\item Make sure that the displayed units are ``Channels'' and not
``x-axis units'' (press that menu to select Channels).
\item Enter a width in channels
\item Convert it to x-axis units by pushing the ``Convert'' button.
\item Enter a different width.
\item Oops, you meant that to be in channels, push the menu button
labeled ``X-axis Units'' and select ``Channels'' (the value you just
typed in has not changed, but the label has).
\item Actually do the smooth.
\end{itemize}
\end{itemize}
\item Fit some baselines.
\begin{itemize}
\item Select ``Polynomial'' 
(``Sinusoidal'' also works, but, as with most sinusoidal fitters, its
fairly sensitive to the initial conditions.  It tends to work best if
the initial guess at a period is longer than the suspected period in the data.)
\item Choose an order.
\item Chose the x-axis ranges to include in the fit, using the plotter.
\begin{itemize}
\item click on ``Cursor active'' at the bottom of the Baselines GUI.
\item Mark some ranges with the cursor - left mouse to start and end, right
mouse to cancel.
\item The height of each range box is 2x the RMS within that box, the 
center Y of a range box is the local mean.
\item Notice that these ranges are also displayed in the Baselines GUI.
\item You can convert to X-axis units in the Baselines GUI in DISH.
\end{itemize}
\item With the ``Recalculate'' and ``Show'' options selected, press ``Apply''.
\item Note the RMS in the GUI, the fit overlaying the data on the 
plotter and the original record still selected.  The RMS value is the
RMS of the data with respect to the fit over the baseline regions.
\item Change the order and/or the ranges
\item Ranges can be edited in the Baselines GUI entry - these changes will
not appear on the plotter until the ``Apply'' button is pressed or you hit
the Enter/Return key.  The contents of the ranges entry are used ``as 
is'' whenever ``Apply'' is pressed (even if you haven't previous hit return).
\item When you are satisfied that this is a baseline you want to remove,
turn off the ``Recalculate'' button and select the ``Subtract'' option
and press ``Apply''.  Note that if manual scaling is in effect as recommended
above then you will need to re-scale at this point,  -0.1 to 0.1 seems to
work well.
\end{itemize}
\item Function on data - with a record selected and plotted try the following
(comments appear after the \verb_#_), you might want to return to the original
record before each application of a function on the data.
\begin{itemize}
\item {\tt ARR*ARR}  \verb_#_ square of the data array, press ``Apply''
\item {\tt ARR * 10}  \verb_#_ scale the data by 10.0
\item {\tt sqrt(ARR)} \verb_#_ take the sqrt of the data
\item Any valid glish can be used here. {\tt ARR} is replaced by the data 
array, {\tt HEADER} by
the {\tt header} record, {\tt NS\_HEADER} by the {\tt ns\_header} 
record, {\tt DESC} by the 
{\tt desc} record, and {\tt DATA} by the {\tt data} record.  
The operations must return
an array of the same shape as the input data array.  Any global glish
variable can be used here (including functions).
\end{itemize}
\item Turn on the ``Trace'' option and try some of these operations again.
\item Select ``Open Eval Window'' from the ``Tools'' menu.
\begin{itemize}
\item Type any valid glish in the eval window.
\item Press ``Eval Contents''
\item marvel at the result which appears in the glish window where you
started from.
\end{itemize}
\item Try the Multi-Operation Operation.
\begin{itemize}
\item Clicking
on one of the ``Add:'' buttons adds that operation to the operation
cache.  Clicking on ``Do multi-op sequence'' executes this sequence,
doing those operations in order using the parameters as currently set
in each operation window (even those which aren't currently displayed).
\item Add the operations ``Average'', ``Baseline'', and ``Smoothing'' to
the cache and do that sequence.
\item Store that sequence by giving it a name by which it will be known 
in the ``Store Cache'' entry and press the ``Store Cache'' button.
\item Reverse the order of the baseline and smoothing operations by
first deleting baseline and then adding it back in.
\item Do this sequence and save it with a different name.
\item Restore the first sequence by selecting it (clicking on it) in the
list of stored caches.
\item Our current thinking is that once the user gets various parameters
set up that DISH will be often ``driven'' from this multi-operation
operation menu.
\end{itemize}
\item A simple command-line example.
In this example you will be taking two total power scans, one on-source
and one off-source, constructing the difference spectra and
placing the result back in the results manager where GUI operations
on it could continue.
\begin{itemize}
\item Open the {\tt dishmopra} data set
\item At the glish command line, type the following (the comments are not
necessary, obviously)
\begin{verbatim}
dishmopra.setlocation(1)  # point the iterator at record 1
on := dishmopra.get()     # get the ``on source'' data
dishmopra.setlocation(3)  # point the iterator at record 3
off := dishmopra.get()    # get the ``off source'' data
result := on              # initialize the result to have
                          # the same structure and header
                          # values as on
result.data.arr :=
   (on.data.arr - off.data.arr)/off.data
                          # the difference spectra
dish.rm().add('result'    # add it to the results manager by name
              'Difference of rows 1 and 3'   # description
              result      # the value
              'SDRECORD') # its type - dish knows how to plot these
\end{verbatim}
\item In the Results Manager you should see a new variable named
``result1''.  Select it.  It should now be displayed.
\item Some manual scaling is necessary - the values at the end are
unreliable due to the overall bandpass shape of on and off.
A range of -0.01 to 0.01 works well here.
\end{itemize}
\end{itemize}

\section{Utilities}

\begin{itemize}
\item Show system information:
\begin{verbatim}
sysinfo()
sysinfo().numcpu()
sysinfo().host()
\end{verbatim}
\item GUI stuff: each one of these lines does something interesting:
\begin{verbatim}
f:=infowindow('Eat at Bodos', 'Dining recommendations') # Informative only
choicewin('Type of bagel',"plain wheat sourdough")      # Choice returned to Glish
g:=guiframework()					# Framework of GUIs
\end{verbatim}
\end{itemize}

