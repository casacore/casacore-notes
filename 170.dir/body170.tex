%%-----------------------------------------------------------------------------
\section{Introduction} The AIPS++ project adopted a formal procedure
for quality checking code in the late summer of 1994.  The
procedure requires that all new core\footnote{{\it aips, dish, synthesis} 
and {\it vlbi} packages} source code be submitted to the ``code
cop'' ({\it email: aips2-cop@nrao.edu}) before it can become sanctioned
AIPS++ code.  Once received, the cop assigns a reviewer, who judges
the code against the {\bf AIPS++ Standards and Guidelines}
(note 167).  A dialog between the programmer and the reviewer may 
follow, eventually leading to approval of the code, and its formal
inclusion into AIPS++.   

In this note, I discuss the review procedure
\begin{itemize}
  \item in general
  \item from the perspective of the programmer submitting new code
  \item from the perspective of the reviewer
  \item with special attention to test programs
\end{itemize}

%%-----------------------------------------------------------------------------
\section{General}
There are eight documents, in addition to this one, which are
essential to the code preparation and review process.  Both programmer
and reviewer should study them carefully, and base their work upon them.

\begin{itemize}
  \item {\em code/install/codedevl/template-class-h:}  how to construct a
      class header file.
  \item {\em code/aips/implement/Lattices/Slicer.h:} good example of a
      class header.
  \item {\em code/install/codedevl/template-module-h:} how to construct a module 
     headers file.
  \item {\em code/aips/implement/Tables.h:} good example of a module header.
  \item {\em code/install/codedevl/template-global-functions-h:}  how to 
      construct a  header file for global functions.
  \item {\em code/aips/implement/Arrays/ArrayIO.h:} good example of a
      global function header file.
  \item {\em code/doc/notes/167.latex:} Standards and Guidelines Checklist.
  \item {\em http://www.cv.nrao.edu/aips++/docs/html/cxx2html.html} An online
      guide to the documentation extractor, with good descriptions of the
      markup tags to be used in creating header files.
\end{itemize}

There are three main goals for the code review, all of which are equally 
important:
\begin{itemize}
  \item Guarantee that the fundamental design of the class, as exposed
        in the public interface, is sensible and reasonably complete.  
        We are more concerned about overall design than implementation
        details, because we assume that the programmer is competent; that the 
        implementation may be too complicated to analyze in a reasonable
        amount of time; and that test programs and early usage will indicate 
        errors and gross inefficiencies.  The public interface, though, is
        everyone's  concern, it is a good indicator of overall design, and
         it is painful to change once a class has been checked in.
  \item Make sure that the class header file is a sufficient guide to
        the class for client programmers.  This is a very high standard,
        and demands a great deal of work from the programmer and the 
        reviewer.  ``Sufficient'' means that a programmer,
        previously unacquainted with the class, can read the header file
        and quickly reach a point where he or she can write code that uses 
        the class.  This standard of explanation is often met by well-written 
        programming books, and is rarely  met by programmers in their own 
        internal documentation.  It is, nonetheless, the standard for all 
        AIPS++ source code.
  \item Use the test program(s) to make sure that memory leaks are 
        discovered and removed. See that code coverage is about 90\% 
        complete\footnote {Tools like ``Purify'' and ``TestCenter''
        should be used if they are available.}.  
\end{itemize}

A great deal can be learned from reading through a class, and
running test and demo programs.  But no review is complete until
other programmers use the new code.  For this reason, we have a 
package called {\bf trial}.\footnote
{{\it (your-workspace)/code/trial} -- at the same logical level as
{\it aips, dish, vlbi, synthesis} and all of the site-specific packages}
This package is automatically distributed by the AIPS++ code distribution
system.  If you ask the code cop to install your code in the trial
directory, it will be available for everyone to examine and use.  (All
of these possible users will understand that {\it trial} code is subject
to change with little or no notice.)
%%-----------------------------------------------------------------------------
\section{For the Programmer}
\begin{itemize}
  \item When you are within a week of submitting code, send the cop email   
        briefly explaining the purpose of your class, and when it will
        arrive.  This lead time will allow the cop to begin the search
        for a suitable reviewer.

  \item If your class is called, for example {\it ``NewClass''}, then you 
        need to submit the following files:
         \begin{enumerate}
           \item  {\it NewClass.h}: the class header
           \item  {\it NewClass.cc}: the class implementation
           \item  {\it tNewClass.cc}\footnote {If there is more than one
                  test, exec, output or demo program, name subsequent files,
                  for example {\it tNewClass1.cc, tNewClass2.cc} -- 
                  the first program  is implicitly  number zero.}: 
                  a test program -- see section devoted to this topic below
           \item  {\it dNewClass.cc} (optional):  a short, heavily commented
            and well-focused program demonstrating how to use {\it NewClass}.
         \end{enumerate}
  \item Your code (header, source, test, and demo programs) should be
        combined into a standard unix ``shar'' file and emailed to
        {\it aips2-cop@nrao.edu}\footnote{If the class source code
        amounts to many thousands of lines of text, explain this to the cop --
        it may be possible for the reviewer to examine the code ``in place'',
        on your machine, or checked in to the {\it trial} directory.}.
         This shar file should preserve the
        directory structure needed to build the code once it is unshar'd.
  \item The cop will pass the shar file onto the reviewer, and tell you
        who the reviewer is.
  \item All subsequent email exchange between you and the reviewer should
        be CC'd to the cop.
  \item If you have test coverage and memory leak statistics ({\em strongly}
        recommended for all programmers who have access to the right tools)
        please send them to the cop in a separate email message.
\end{itemize}
%%-----------------------------------------------------------------------------
\section{For the Reviewer}
Your job is a difficult one and requires a lot of time and effort.
It may also require  imagination and diplomacy.  Imagination is
needed because you must cultivate the perspective of the non-expert
client programmer, intentionally forgetting things that you know, and
scrutinizing the code (the class header in particular) to make sure
that it explains everything the client programmer will need to know.
You must also be able to take on the domain expert's point of view,
and carefully examine the new class for the quality of its overall
design.  You may need considerable diplomacy to persuade a
recalcitrant author that they need to make changes in their code and
documentation.

Your responsibililites include these specific actions:

\begin{itemize}
  \item Let the cop and the author know how long it will take to
        make the first review of the code, and how well you are
        sticking to that schedule as the time goes by.
  \item All email between you and the author which is related to the
        new class should be CC'd to the code cop.
  \item Be prepared for the cop to suggest changes beyond those you
        require.
  \item Base most of your evaluation on the header file for each 
        class.  A quick reading of the implementation file 
        will allow you to judge the adequacy of its documentation,
        and detect any troubling features.
\end{itemize}
%%-----------------------------------------------------------------------------
\section{Test Programs}
The programmer must provide one or more programs that thorougly test
the newly submitted code.  These programs should be very simple to
run, and should exercise all aspects of the new code.\footnote{A handy
way to construct a test program for a new class is to copy the class
header file into a new file -- which will become the test program --
and edit the member function declarations into member function calls.
This simple approach guarantees that all of the member functions are
called.  The next step would be to intentionally call all of the
member functions in such a way that all the exceptions are thrown:
if you trigger and catch all of the exceptions, then you have gone
a long way towards exercising all of the code.  ``Domain'' coverage should
be attempted too:  exercise the member functions by calling them with a 
range of inputs, paying particular attention to the boundaries of the domain.}

To indicate a successful test, it will often suffice for the test
program to simply print``OK'' to standard output, and return {\it 0}
as the value of the main program.  In the test program fails, it
should print a descriptive error message, and return some non-zero
value to the environment.  Some classes and global functions
(especially those which employ numerical methods) can only be
adequately tested by comparing a list of expected results against
actual results, making allowance for tolerable roundoff
differences\footnote {We hope that, in time, the AIPS++ project will
find or develop a ``diff'' tool that is smart enough to compare
numbers, and distinguish between allowable differences (due to
word-length round-off) and unacceptable numerical error.}.

The programmer may use two strategies here:

\begin{enumerate}
   \item Create tMyClass.out, which contains the {\it expected} results, and
         tMyClass.exec, a shell script which runs tMyClass, redirects
         the actual results to a file, and compares them to the 
         contents of tMyClass.out.
   \item Alternatively, the tMyClass program {\em itself} creates the new 
         results file, and runs the comparison, using a {\it system} call;  
         when the comparison of results is complete, tMyClass prints ``OK'' 
         and returns {\it 0} to the environment if it is successful, just as
         the simpler test programs do.
\end{enumerate}
%%-----------------------------------------------------------------------------
