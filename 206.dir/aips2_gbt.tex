%
% jmcmulli
% aips++_gbt.tex,aips++_gbt.ps
% psiz=77K
%
% April 11, 1997
% AIPS++ in GBT
% AIPS++ in GBT : Varium et mutabile semper AIPS++
% Joseph P. McMullin
% Green Bank
% West Virginia
%
\newcommand{\thisdocURL}{http://aips2.nrao.edu/aips++/docs/notes/206/206.html}
\newcommand{\searchURL}{../../html/aips2search.html}
\newcommand{\aipsviewURL}{../../user/aipsview/aipsview.html}
\newcommand{\GlishmanualURL}{../../reference/Glish/Glish.html}
\newcommand{\GlishtutorialURL}{../../notes/195/195.html}
\newcommand{\synthesisURL}{../../user/synthesis/synthesis.html}
\newcommand{\gsURL}{../../user/gettingstarted/gettingstarted.html}
\newcommand{\userrefmanualURL}{../../user/Refman/Refman.html}
\newcommand{\specsURL}{../../specs/specs.html}
\newcommand{\betaURL}{../beta/beta.html}
\newcommand{\consortiumURL}{../consortium.html}
\def\drakoshomelink{http://cbl.leeds.ac.uk/nilos/personal.html}
\def\thisdocURL{http://aips2.nrao.edu/aips++/docs/notes/206/206.html}
\def\gbthomelink{http://www.gb.nrao.edu/GBT/GBT.html}
\def\vlahomelink{http://www.nrao.edu/doc/vla/html/VLAhome.shtml}
\def\cvhomelink{http://www.cv.nrao.edu/cv-home.html}

%
% URL to Help system
%
\externallabels{../../user/Refman}{../../user/Refman/labels.pl}

% Add home page navigation button
%

\htmladdtonavigation{\htmladdnormallink
  {\htmladdimg{../../gif/home.gif}}{{../../html/aips++.html}}}

\section{Purpose}

This document is for beginning users of AIPS++ in Green Bank. Although the
installation is meant to be mostly homogeneous from site to site, there
are local variations which may ease/hinder your AIPS++ experience. 
This note will guide the user to the appropriate existing documentation.

\bigskip
\section{Introduction}

AIPS++ (Astronomical Information Processing System, written in the C++ 
language) is an ambitious project to standardize the
processing of (primarily radio) astronomical data, both interferometric
and filled aperture. Currently under development by an international
consortium of observatories, AIPS++ is now under beta release, with
several other 'experimental' releases available to consortium organizations
including NRAO-Green Bank.

\bigskip
\subsection{Computing in GB}

Most staff computers in Green Bank fulfill the minimum requirements for 
running AIPS++ but fall short of the canonical system. Currently there
is only one such system in Green Bank, rigel, which is designated as the
visitor/observer machine in Room 105 of the new wing of the Jansky Lab.

One can setup to run AIPS++ on rigel remotely by doing the following:

\begin{description}
\item{$>$} xhost +rigel.gb.nrao.edu (from local machine)
\item{$>$} rsh rigel (from local machine xterm or terminal)

For csh,tcsh:

\item{$>$} setenv DISPLAY mymachine.gb.nrao.edu:0 (from rigel)

For Bourne-type shells:

\item{$>$} export DISPLAY mymachine.gb.nrao.edu:0.0 (from rigel)
\end{description}


\bigskip
\subsection{Versions of AIPS++}

There are currently four different versions of AIPS++ available in
Green Bank; these are labeled: 
1)beta, 2) test, 3) new, and 4) old.

The beta release is a limited package targeted to consortium sites and a
few friendly astronomers. It contains only the basic environment, some
tools and some synthesis applications. It is intended to provide early
exposure and feedback for AIPS++. 

The test version is the package currently under development by all of the AIPS++
programmers. As such, it is a highly volatile environment which contains
all of the latest tools along with the newest bugs and problems.
The new version is designated as the most up to date, stable version of the
software. This is currently the version used (linked with) the Monitor and
Control software in Green Bank.
The old version is intended as a fall back, in case something goes wrong
with new (particularly during an update).

The test version of AIPS++ changes weekly (on Fridays at 21:30 lasting until
approximately Saturday 06:00). If you expect to be using AIPS++ during this critical time
please contact Joe McMullin (jmcmulli@nrao.edu) to suspend the crontab
for that week. The evolution of test$\rightarrow$new and
new$\rightarrow$old occurs less periodically, typically after substantial
changes have stabilized.
Currently, the new version is from Nov-1996 and the old version is from
the Pre-Cambrian Period.

Each of these versions lives under the /aips++ directory (e.g. /aips++/beta,
/aips++/test, etc.).


\bigskip
\subsection{Setting up your environment}


Setting up your environment is covered more thoroughly in the "Getting
Started in AIPS++" document. We elaborate here on a specific example
appropriate to Green Bank.

In order to setup the environment (paths, etc) for executing AIPS++,
only a single command need be invoked.

For C-like shells (csh,tcsh):

$>$ source /aips++/version/aipsinit.csh, where version is beta, test, new,
	or old.

For Bourne-like shells (sh, bash, ksh, zsh):

. /aips++/version/aipsinit.sh, where version is again, beta, test, new or old.

At this point, you can enter the AIPS++ environment by typing "aips++" 
(NOTE: This applies only to the beta and test versions currently).

In addition, there are two standard configuration files which can
be modified to suit your needs: .aipsrc and .glishrc. These files
should live in your home directory.

\begin{description}
\item{.glishrc} This file can be used to customize the Glish interface by
	setting search paths for Glish scripts, setting default precision for
	output, starting particular packages, etc. It can contain any valid
	Glish command. For example, an engineering .glishrc file in 
	Green Bank might contain the following:
\begin{verbatim}
	system.path.include:=". /aips++/engr-scripts /aips++/gb-astro-scripts 
	/aips++/test/sun4sol\_gnu/libexec";
	system.print.precision:=9
	include "rtools3.g"
\end{verbatim}
This sets up the path to include several directories where useful scripts
might be found (specifically, in Green Bank, the directories 
/aips++/engr-scripts and /aips++/gb-astro-scripts, are repositories of
local scripts), sets the print precision to 9, and includes the file
"rtools3.g" which contains a number of useful functions (rtools3.g will
be discussed in more detail later).
However, be aware that your personal .glishrc file may override any scripts
executed prior to entering AIPS++. For the example above, if you wanted to
use the new version of AIPS++ and executed the appropriate aipsinit script
for that version, AIPS++ would still then execute your .glishrc script which
would tell it to read from the test directory 
(/aips++/test/sun4sol\_gnu/libexec). It is best to avoid adding a version
specific directory to this path; the path is set to be correct for the
current installation, through the setup scripts.

\item{.aipsrc} This file customizes the AIPS++ interface. For example:
\begin{verbatim}
	logger.default: gui
\end{verbatim}
Designates the logger GUI as the default output for logger messages.
\end{description}
%which file to source (csh/tcsh) or . (bash) to setup environment
%.aipsrc file
%.glishrc file
\bigskip

\section{Green Bank Tasks and Tools}
 
This section will be evolving rapidly. Please continue to check this
page regularly.

Currently, there are several script collections available. 

\begin{description}

\item{}DCR utility functions. 

\item{}Tipping, cross correlation, rms calculation, baseline removal, gaussian fitting, obtain pointing offsets. 

\item{}Contour Maps. 

\item{}This will perform a primitive contour map from a DCR scan. 

\item{}Plotting. 

\item{}Performs low-level plotting function. 

\item{}Pulsar timing. 

\item{}IF test routine. 

\item{}Pulsar timing. 

\end{description}

\section{Documentation}

Currently, there is existing documentation treating:

\begin{description}
\item{}\htmladdnormallink{Getting Started in AIPS++}{http://aips2.nrao.edu/aips++/docs/user/gettingstarted/gettingstarted.html}.
\item{}\htmladdnormallink{AIPS++ Reference Manual}{http://aips2.nrao.edu/aips++/docs/user/Refman/Refman.html}.
\item{}\htmladdnormallink{Glish Tutorial}{http://aips2.nrao.edu/aips++/docs/notes/195/195.html}.
\item{}\htmladdnormallink{Glish User Manual}{http://aips2.nrao.edu/aips++/docs/reference/Glish/Glish.html}.
\item{}\htmladdnormallink{Synthesis Processing}{http://aips2.nrao.edu/aips++/docs/user/Synthesis/Synthesis.html}.
\item{}\htmladdnormallink{AipsView}{http://aips2.nrao.edu/aips++/docs/user/Aipsview/Aipsview.html}.
\end{description}

In addition, there are a number of programming manuals, along with
collections of notes, papers and memos on AIPS++.
 
All of which is available on the web at the AIPS++ home page:
 
http://aips2.nrao.edu/aips++/docs/aips++.html

Changes to the environment in Green Bank will also be updated regularly
here (specific url).
 
Information can also be obtained through the help facility (again see the
"Getting Started in AIPS++" document for more details). A brief synopsis
follows.

Help can be obtained in a number of:
\begin{description}
\item{1.} - help() : type help() at the aips++ prompt. This will review your
	help options. 
\item{2.} - help('modulename') : Note: the modulename (e.g. table,plotter,etc)
		must be in single quotes.
\item{3.} - web() : This will drive your WWW browser to the AIPS++ Reference
	Manual (if you haven't recently tried help on a modulename) or to
	the modulename's help page (if you have executed a help('modulename')
	command.
\end{description}

\subsection{Reporting Problems}

There is a formalized method of reporting bugs encountered in AIPS++. 
Within AIPS++, the 'bug()' command will pop up a form which allows
description of the problem and will then send the report to the appropriate
people. In addition, the AIPS++ home page also provides a standard form;
point your browser to:

\begin{description}
\item{}\htmladdnormallink{http://aips2.nrao.edu/aips++/docs/contactus/reportbug.htm}{http://aips2.nrao.edu/aips++/docs/contactus/reportbug.html}.
\end{description}

In addition, the person responsible for AIPS++ in Green Bank is Joe McMullin.
He may be contacted at any time regarding problems (Office: 
304-456-2236/Home:304-456-5369/E-mail:jmcmulli@nrao.edu).

\end{document}
